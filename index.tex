% Options for packages loaded elsewhere
\PassOptionsToPackage{unicode}{hyperref}
\PassOptionsToPackage{hyphens}{url}
\PassOptionsToPackage{dvipsnames,svgnames,x11names}{xcolor}
%
\documentclass[
  letterpaper,
  DIV=11,
  numbers=noendperiod]{scrreprt}

\usepackage{amsmath,amssymb}
\usepackage{iftex}
\ifPDFTeX
  \usepackage[T1]{fontenc}
  \usepackage[utf8]{inputenc}
  \usepackage{textcomp} % provide euro and other symbols
\else % if luatex or xetex
  \usepackage{unicode-math}
  \defaultfontfeatures{Scale=MatchLowercase}
  \defaultfontfeatures[\rmfamily]{Ligatures=TeX,Scale=1}
\fi
\usepackage{lmodern}
\ifPDFTeX\else  
    % xetex/luatex font selection
\fi
% Use upquote if available, for straight quotes in verbatim environments
\IfFileExists{upquote.sty}{\usepackage{upquote}}{}
\IfFileExists{microtype.sty}{% use microtype if available
  \usepackage[]{microtype}
  \UseMicrotypeSet[protrusion]{basicmath} % disable protrusion for tt fonts
}{}
\makeatletter
\@ifundefined{KOMAClassName}{% if non-KOMA class
  \IfFileExists{parskip.sty}{%
    \usepackage{parskip}
  }{% else
    \setlength{\parindent}{0pt}
    \setlength{\parskip}{6pt plus 2pt minus 1pt}}
}{% if KOMA class
  \KOMAoptions{parskip=half}}
\makeatother
\usepackage{xcolor}
\ifLuaTeX
  \usepackage{luacolor}
  \usepackage[soul]{lua-ul}
\else
  \usepackage{soul}
  
\fi
\setlength{\emergencystretch}{3em} % prevent overfull lines
\setcounter{secnumdepth}{-\maxdimen} % remove section numbering
% Make \paragraph and \subparagraph free-standing
\makeatletter
\ifx\paragraph\undefined\else
  \let\oldparagraph\paragraph
  \renewcommand{\paragraph}{
    \@ifstar
      \xxxParagraphStar
      \xxxParagraphNoStar
  }
  \newcommand{\xxxParagraphStar}[1]{\oldparagraph*{#1}\mbox{}}
  \newcommand{\xxxParagraphNoStar}[1]{\oldparagraph{#1}\mbox{}}
\fi
\ifx\subparagraph\undefined\else
  \let\oldsubparagraph\subparagraph
  \renewcommand{\subparagraph}{
    \@ifstar
      \xxxSubParagraphStar
      \xxxSubParagraphNoStar
  }
  \newcommand{\xxxSubParagraphStar}[1]{\oldsubparagraph*{#1}\mbox{}}
  \newcommand{\xxxSubParagraphNoStar}[1]{\oldsubparagraph{#1}\mbox{}}
\fi
\makeatother

\usepackage{color}
\usepackage{fancyvrb}
\newcommand{\VerbBar}{|}
\newcommand{\VERB}{\Verb[commandchars=\\\{\}]}
\DefineVerbatimEnvironment{Highlighting}{Verbatim}{commandchars=\\\{\}}
% Add ',fontsize=\small' for more characters per line
\usepackage{framed}
\definecolor{shadecolor}{RGB}{241,243,245}
\newenvironment{Shaded}{\begin{snugshade}}{\end{snugshade}}
\newcommand{\AlertTok}[1]{\textcolor[rgb]{0.68,0.00,0.00}{#1}}
\newcommand{\AnnotationTok}[1]{\textcolor[rgb]{0.37,0.37,0.37}{#1}}
\newcommand{\AttributeTok}[1]{\textcolor[rgb]{0.40,0.45,0.13}{#1}}
\newcommand{\BaseNTok}[1]{\textcolor[rgb]{0.68,0.00,0.00}{#1}}
\newcommand{\BuiltInTok}[1]{\textcolor[rgb]{0.00,0.23,0.31}{#1}}
\newcommand{\CharTok}[1]{\textcolor[rgb]{0.13,0.47,0.30}{#1}}
\newcommand{\CommentTok}[1]{\textcolor[rgb]{0.37,0.37,0.37}{#1}}
\newcommand{\CommentVarTok}[1]{\textcolor[rgb]{0.37,0.37,0.37}{\textit{#1}}}
\newcommand{\ConstantTok}[1]{\textcolor[rgb]{0.56,0.35,0.01}{#1}}
\newcommand{\ControlFlowTok}[1]{\textcolor[rgb]{0.00,0.23,0.31}{\textbf{#1}}}
\newcommand{\DataTypeTok}[1]{\textcolor[rgb]{0.68,0.00,0.00}{#1}}
\newcommand{\DecValTok}[1]{\textcolor[rgb]{0.68,0.00,0.00}{#1}}
\newcommand{\DocumentationTok}[1]{\textcolor[rgb]{0.37,0.37,0.37}{\textit{#1}}}
\newcommand{\ErrorTok}[1]{\textcolor[rgb]{0.68,0.00,0.00}{#1}}
\newcommand{\ExtensionTok}[1]{\textcolor[rgb]{0.00,0.23,0.31}{#1}}
\newcommand{\FloatTok}[1]{\textcolor[rgb]{0.68,0.00,0.00}{#1}}
\newcommand{\FunctionTok}[1]{\textcolor[rgb]{0.28,0.35,0.67}{#1}}
\newcommand{\ImportTok}[1]{\textcolor[rgb]{0.00,0.46,0.62}{#1}}
\newcommand{\InformationTok}[1]{\textcolor[rgb]{0.37,0.37,0.37}{#1}}
\newcommand{\KeywordTok}[1]{\textcolor[rgb]{0.00,0.23,0.31}{\textbf{#1}}}
\newcommand{\NormalTok}[1]{\textcolor[rgb]{0.00,0.23,0.31}{#1}}
\newcommand{\OperatorTok}[1]{\textcolor[rgb]{0.37,0.37,0.37}{#1}}
\newcommand{\OtherTok}[1]{\textcolor[rgb]{0.00,0.23,0.31}{#1}}
\newcommand{\PreprocessorTok}[1]{\textcolor[rgb]{0.68,0.00,0.00}{#1}}
\newcommand{\RegionMarkerTok}[1]{\textcolor[rgb]{0.00,0.23,0.31}{#1}}
\newcommand{\SpecialCharTok}[1]{\textcolor[rgb]{0.37,0.37,0.37}{#1}}
\newcommand{\SpecialStringTok}[1]{\textcolor[rgb]{0.13,0.47,0.30}{#1}}
\newcommand{\StringTok}[1]{\textcolor[rgb]{0.13,0.47,0.30}{#1}}
\newcommand{\VariableTok}[1]{\textcolor[rgb]{0.07,0.07,0.07}{#1}}
\newcommand{\VerbatimStringTok}[1]{\textcolor[rgb]{0.13,0.47,0.30}{#1}}
\newcommand{\WarningTok}[1]{\textcolor[rgb]{0.37,0.37,0.37}{\textit{#1}}}

\providecommand{\tightlist}{%
  \setlength{\itemsep}{0pt}\setlength{\parskip}{0pt}}\usepackage{longtable,booktabs,array}
\usepackage{calc} % for calculating minipage widths
% Correct order of tables after \paragraph or \subparagraph
\usepackage{etoolbox}
\makeatletter
\patchcmd\longtable{\par}{\if@noskipsec\mbox{}\fi\par}{}{}
\makeatother
% Allow footnotes in longtable head/foot
\IfFileExists{footnotehyper.sty}{\usepackage{footnotehyper}}{\usepackage{footnote}}
\makesavenoteenv{longtable}
\usepackage{graphicx}
\makeatletter
\def\maxwidth{\ifdim\Gin@nat@width>\linewidth\linewidth\else\Gin@nat@width\fi}
\def\maxheight{\ifdim\Gin@nat@height>\textheight\textheight\else\Gin@nat@height\fi}
\makeatother
% Scale images if necessary, so that they will not overflow the page
% margins by default, and it is still possible to overwrite the defaults
% using explicit options in \includegraphics[width, height, ...]{}
\setkeys{Gin}{width=\maxwidth,height=\maxheight,keepaspectratio}
% Set default figure placement to htbp
\makeatletter
\def\fps@figure{htbp}
\makeatother

\KOMAoption{captions}{tableheading}
\makeatletter
\@ifpackageloaded{tcolorbox}{}{\usepackage[skins,breakable]{tcolorbox}}
\@ifpackageloaded{fontawesome5}{}{\usepackage{fontawesome5}}
\definecolor{quarto-callout-color}{HTML}{909090}
\definecolor{quarto-callout-note-color}{HTML}{0758E5}
\definecolor{quarto-callout-important-color}{HTML}{CC1914}
\definecolor{quarto-callout-warning-color}{HTML}{EB9113}
\definecolor{quarto-callout-tip-color}{HTML}{00A047}
\definecolor{quarto-callout-caution-color}{HTML}{FC5300}
\definecolor{quarto-callout-color-frame}{HTML}{acacac}
\definecolor{quarto-callout-note-color-frame}{HTML}{4582ec}
\definecolor{quarto-callout-important-color-frame}{HTML}{d9534f}
\definecolor{quarto-callout-warning-color-frame}{HTML}{f0ad4e}
\definecolor{quarto-callout-tip-color-frame}{HTML}{02b875}
\definecolor{quarto-callout-caution-color-frame}{HTML}{fd7e14}
\makeatother
\makeatletter
\@ifpackageloaded{bookmark}{}{\usepackage{bookmark}}
\makeatother
\makeatletter
\@ifpackageloaded{caption}{}{\usepackage{caption}}
\AtBeginDocument{%
\ifdefined\contentsname
  \renewcommand*\contentsname{Table of contents}
\else
  \newcommand\contentsname{Table of contents}
\fi
\ifdefined\listfigurename
  \renewcommand*\listfigurename{List of Figures}
\else
  \newcommand\listfigurename{List of Figures}
\fi
\ifdefined\listtablename
  \renewcommand*\listtablename{List of Tables}
\else
  \newcommand\listtablename{List of Tables}
\fi
\ifdefined\figurename
  \renewcommand*\figurename{Figure}
\else
  \newcommand\figurename{Figure}
\fi
\ifdefined\tablename
  \renewcommand*\tablename{Table}
\else
  \newcommand\tablename{Table}
\fi
}
\@ifpackageloaded{float}{}{\usepackage{float}}
\floatstyle{ruled}
\@ifundefined{c@chapter}{\newfloat{codelisting}{h}{lop}}{\newfloat{codelisting}{h}{lop}[chapter]}
\floatname{codelisting}{Listing}
\newcommand*\listoflistings{\listof{codelisting}{List of Listings}}
\makeatother
\makeatletter
\makeatother
\makeatletter
\@ifpackageloaded{caption}{}{\usepackage{caption}}
\@ifpackageloaded{subcaption}{}{\usepackage{subcaption}}
\makeatother

\ifLuaTeX
  \usepackage{selnolig}  % disable illegal ligatures
\fi
\usepackage{bookmark}

\IfFileExists{xurl.sty}{\usepackage{xurl}}{} % add URL line breaks if available
\urlstyle{same} % disable monospaced font for URLs
\hypersetup{
  pdftitle={Course Notes for DS 101: In-Depth Introduction to Data Science},
  pdfauthor={Hung Tong},
  colorlinks=true,
  linkcolor={blue},
  filecolor={Maroon},
  citecolor={Blue},
  urlcolor={Blue},
  pdfcreator={LaTeX via pandoc}}


\title{Course Notes for DS 101: In-Depth Introduction to Data Science}
\author{Hung Tong}
\date{2026-01-01}

\begin{document}
\maketitle

\renewcommand*\contentsname{Table of contents}
{
\hypersetup{linkcolor=}
\setcounter{tocdepth}{2}
\tableofcontents
}

\bookmarksetup{startatroot}

\chapter*{Preface}\label{preface}
\addcontentsline{toc}{chapter}{Preface}

\markboth{Preface}{Preface}

This text contains my lecture notes for DS 101: In-Depth Introduction to
Data Science at Rowan University.

\part{Base R programming}

\chapter{1: R Basics}\label{r-basics}

\section{Reading}\label{reading}

From \textbf{R Coding Basics: An Introduction to the Basics of Coding in
R} by Dr.~Gaston Sanchez:

\begin{itemize}
\item
  \href{https://www.gastonsanchez.com/R-ice-breaker/1-04-first-contact.html}{First
  Contact with R}
\item
  \href{https://www.gastonsanchez.com/R-ice-breaker/1-05-help-documentation.html}{Getting
  Help}
\end{itemize}

\section{Topics}\label{topics}

\begin{itemize}
\item
  Mathematical operations
\item
  Operation precedence
\item
  Comments
\item
  Objects, variables, and assignments
\item
  Variable names and reserved words
\item
  Functions
\item
  Help Documentation
\item
  Packages
\end{itemize}

\section{Mathematical Operations}\label{mathematical-operations}

\begin{itemize}
\tightlist
\item
  Common mathematical operations in R include:
\end{itemize}

\begin{Shaded}
\begin{Highlighting}[]
\DecValTok{2} \SpecialCharTok{+} \DecValTok{2}     \CommentTok{\# Add}

\DecValTok{3} \SpecialCharTok{{-}} \DecValTok{4}     \CommentTok{\# Subtract}

\DecValTok{5} \SpecialCharTok{*} \DecValTok{2}     \CommentTok{\# Multiply}

\DecValTok{4} \SpecialCharTok{/} \DecValTok{2}     \CommentTok{\# Divide}

\DecValTok{2} \SpecialCharTok{\^{}} \DecValTok{3}     \CommentTok{\# Exponent}
\DecValTok{2} \SpecialCharTok{**} \DecValTok{3}    \CommentTok{\# Exponent as well}
\end{Highlighting}
\end{Shaded}

\section{Operation Precedence}\label{operation-precedence}

\begin{itemize}
\tightlist
\item
  In R, operator precedence (from highest to lowest) is:
\end{itemize}

\begin{Shaded}
\begin{Highlighting}[]
\CommentTok{\# Parenthesis}
\NormalTok{(}\DecValTok{2} \SpecialCharTok{+} \DecValTok{3}\NormalTok{) }\SpecialCharTok{*} \DecValTok{4}

\CommentTok{\# Exponent}
\DecValTok{2} \SpecialCharTok{\^{}} \DecValTok{3}
\DecValTok{2} \SpecialCharTok{**} \DecValTok{3}

\CommentTok{\# Multiply and divide (from left to right)}
\DecValTok{5} \SpecialCharTok{*} \DecValTok{2}
\DecValTok{4} \SpecialCharTok{/} \DecValTok{2}

\CommentTok{\# Add and subtract}
\DecValTok{2} \SpecialCharTok{+} \DecValTok{2}
\DecValTok{3} \SpecialCharTok{{-}} \DecValTok{4}
\end{Highlighting}
\end{Shaded}

💻 \textbf{Hands-On}

Evaluate the following expressions in R:

\begin{itemize}
\item
  \(19 - 3 \times 6\)
\item
  \(20 - 2 \times 3^2 + 11\)
\item
  \(3 + 8 \div 2 \times 2\)
\item
  \(7 - (5 \times 3 + 2^3)\)
\end{itemize}

\begin{tcolorbox}[enhanced jigsaw, colframe=quarto-callout-tip-color-frame, coltitle=black, left=2mm, rightrule=.15mm, colback=white, opacityback=0, toprule=.15mm, bottomtitle=1mm, colbacktitle=quarto-callout-tip-color!10!white, breakable, titlerule=0mm, title=\textcolor{quarto-callout-tip-color}{\faLightbulb}\hspace{0.5em}{Answer}, toptitle=1mm, arc=.35mm, bottomrule=.15mm, leftrule=.75mm, opacitybacktitle=0.6]

\begin{Shaded}
\begin{Highlighting}[]
\DecValTok{19} \SpecialCharTok{{-}} \DecValTok{3} \SpecialCharTok{*} \DecValTok{6}
\end{Highlighting}
\end{Shaded}

\begin{verbatim}
[1] 1
\end{verbatim}

\begin{Shaded}
\begin{Highlighting}[]
\DecValTok{20} \SpecialCharTok{{-}} \DecValTok{2} \SpecialCharTok{*} \DecValTok{3}\SpecialCharTok{\^{}}\DecValTok{2} \SpecialCharTok{+} \DecValTok{11}
\end{Highlighting}
\end{Shaded}

\begin{verbatim}
[1] 13
\end{verbatim}

\begin{Shaded}
\begin{Highlighting}[]
\DecValTok{3} \SpecialCharTok{+} \DecValTok{8} \SpecialCharTok{/} \DecValTok{2} \SpecialCharTok{*} \DecValTok{2}
\end{Highlighting}
\end{Shaded}

\begin{verbatim}
[1] 11
\end{verbatim}

\begin{Shaded}
\begin{Highlighting}[]
\DecValTok{7} \SpecialCharTok{{-}}\NormalTok{ (}\DecValTok{5} \SpecialCharTok{*} \DecValTok{3} \SpecialCharTok{+} \DecValTok{2}\SpecialCharTok{\^{}}\DecValTok{3}\NormalTok{)}
\end{Highlighting}
\end{Shaded}

\begin{verbatim}
[1] -16
\end{verbatim}

\end{tcolorbox}

\section{Comments}\label{comments}

\begin{itemize}
\item
  Comments are notes to explain what the code does.
\item
  In R, comments are created using a hash \texttt{\#}.
\item
  On a line, everything after a hash will be ignored. In other words,
  even if the comments contain some code, it will not be executed.
\item
  The keyboard short cut to comment out blocks of code is
  \texttt{Command} or \texttt{Ctrl} + \texttt{Shift} + \texttt{c}
\end{itemize}

\begin{Shaded}
\begin{Highlighting}[]
\NormalTok{(}\DecValTok{2} \SpecialCharTok{+} \DecValTok{3}\NormalTok{) }\SpecialCharTok{*} \DecValTok{4}    \CommentTok{\# 2 + 3 will be evaluated first, then multiplied by 4}
\end{Highlighting}
\end{Shaded}

💻 \textbf{Hands-On}

Run the following code in the console and add a comment describing the
output.

\begin{Shaded}
\begin{Highlighting}[]
\DecValTok{5} \SpecialCharTok{*}\NormalTok{ (}\DecValTok{2} \SpecialCharTok{+} \DecValTok{4}\NormalTok{) }\SpecialCharTok{/} \DecValTok{2} 

\NormalTok{(}\DecValTok{5} \SpecialCharTok{*} \DecValTok{2} \SpecialCharTok{+} \DecValTok{4}\NormalTok{) }\SpecialCharTok{/} \DecValTok{2}

\DecValTok{5} \SpecialCharTok{*}\NormalTok{ (}\DecValTok{2} \SpecialCharTok{+} \DecValTok{4} \SpecialCharTok{/} \DecValTok{2}\NormalTok{)}
\end{Highlighting}
\end{Shaded}

\begin{tcolorbox}[enhanced jigsaw, colframe=quarto-callout-tip-color-frame, coltitle=black, left=2mm, rightrule=.15mm, colback=white, opacityback=0, toprule=.15mm, bottomtitle=1mm, colbacktitle=quarto-callout-tip-color!10!white, breakable, titlerule=0mm, title=\textcolor{quarto-callout-tip-color}{\faLightbulb}\hspace{0.5em}{Answer}, toptitle=1mm, arc=.35mm, bottomrule=.15mm, leftrule=.75mm, opacitybacktitle=0.6]

\begin{Shaded}
\begin{Highlighting}[]
\DecValTok{5} \SpecialCharTok{*}\NormalTok{ (}\DecValTok{2} \SpecialCharTok{+} \DecValTok{4}\NormalTok{) }\SpecialCharTok{/} \DecValTok{2}    \CommentTok{\# 15}
\end{Highlighting}
\end{Shaded}

\begin{verbatim}
[1] 15
\end{verbatim}

\begin{Shaded}
\begin{Highlighting}[]
\NormalTok{(}\DecValTok{5} \SpecialCharTok{*} \DecValTok{2} \SpecialCharTok{+} \DecValTok{4}\NormalTok{) }\SpecialCharTok{/} \DecValTok{2}    \CommentTok{\# 7}
\end{Highlighting}
\end{Shaded}

\begin{verbatim}
[1] 7
\end{verbatim}

\begin{Shaded}
\begin{Highlighting}[]
\DecValTok{5} \SpecialCharTok{*}\NormalTok{ (}\DecValTok{2} \SpecialCharTok{+} \DecValTok{4} \SpecialCharTok{/} \DecValTok{2}\NormalTok{)    }\CommentTok{\# 20}
\end{Highlighting}
\end{Shaded}

\begin{verbatim}
[1] 20
\end{verbatim}

\end{tcolorbox}

\section{Object, Variable, and
Assignment}\label{object-variable-and-assignment}

\begin{itemize}
\item
  An \textbf{object} is anything that stores values or instructions.
\item
  A \textbf{variable} is a name associated to an object so that we can
  access the object in memory.
\item
  A variable can be created using the assignment operator
  \texttt{\textless{}-} (left arrow) or the equal sign \texttt{=}.
\end{itemize}

\begin{Shaded}
\begin{Highlighting}[]
\CommentTok{\# 100 is an object itself}
\DecValTok{100}

\CommentTok{\# We name this object \textquotesingle{}eagles\textquotesingle{}}
\NormalTok{eagles }\OtherTok{\textless{}{-}} \DecValTok{100}

\CommentTok{\# The variable \textquotesingle{}eagles\textquotesingle{} allows us to access the object 100}
\NormalTok{eagles}
\end{Highlighting}
\end{Shaded}

💻 \textbf{Hands-On}

Create the following variables:

\begin{itemize}
\item
  \texttt{year} that stores the value \texttt{2026}
\item
  \texttt{price} that stores the value \texttt{5.5}
\item
  \texttt{score} that stores the value \texttt{93}
\end{itemize}

\begin{tcolorbox}[enhanced jigsaw, colframe=quarto-callout-tip-color-frame, coltitle=black, left=2mm, rightrule=.15mm, colback=white, opacityback=0, toprule=.15mm, bottomtitle=1mm, colbacktitle=quarto-callout-tip-color!10!white, breakable, titlerule=0mm, title=\textcolor{quarto-callout-tip-color}{\faLightbulb}\hspace{0.5em}{Answer}, toptitle=1mm, arc=.35mm, bottomrule=.15mm, leftrule=.75mm, opacitybacktitle=0.6]

\begin{Shaded}
\begin{Highlighting}[]
\NormalTok{year }\OtherTok{\textless{}{-}} \DecValTok{2026}

\NormalTok{price }\OtherTok{\textless{}{-}} \FloatTok{5.5}

\NormalTok{score }\OtherTok{\textless{}{-}} \FloatTok{9.3}
\end{Highlighting}
\end{Shaded}

\end{tcolorbox}

\section{Variable Names}\label{variable-names}

\begin{itemize}
\tightlist
\item
  Variable names are \textbf{case-sensitive}.
\end{itemize}

\begin{Shaded}
\begin{Highlighting}[]
\CommentTok{\# These names are NOT the same}
\NormalTok{eagles}

\NormalTok{Eagles}

\NormalTok{EAGLES}

\NormalTok{eaGleS}

\NormalTok{EagLEs}

\NormalTok{EaGlEs}
\end{Highlighting}
\end{Shaded}

💻 \textbf{Hands-On}

Run the following code in the console. What does it return?

\begin{Shaded}
\begin{Highlighting}[]
\NormalTok{EAGLES }\OtherTok{\textless{}{-}} \DecValTok{123}

\NormalTok{Eagles}
\end{Highlighting}
\end{Shaded}

\begin{tcolorbox}[enhanced jigsaw, colframe=quarto-callout-tip-color-frame, coltitle=black, left=2mm, rightrule=.15mm, colback=white, opacityback=0, toprule=.15mm, bottomtitle=1mm, colbacktitle=quarto-callout-tip-color!10!white, breakable, titlerule=0mm, title=\textcolor{quarto-callout-tip-color}{\faLightbulb}\hspace{0.5em}{Answer}, toptitle=1mm, arc=.35mm, bottomrule=.15mm, leftrule=.75mm, opacitybacktitle=0.6]

Since variable names are case-sensitive, \texttt{EAGLES} and
\texttt{Eagles} are different. In this case, we created \texttt{EAGLES}
but not \texttt{Eagles} so R will throw an error.

\begin{Shaded}
\begin{Highlighting}[]
\NormalTok{EAGLES }\OtherTok{\textless{}{-}} \DecValTok{123}

\NormalTok{Eagles}
\end{Highlighting}
\end{Shaded}

\begin{verbatim}
Error: object 'Eagles' not found
\end{verbatim}

\end{tcolorbox}

\section{Variable Names}\label{variable-names-1}

\begin{itemize}
\item
  A variable name must start with a letter or a period (not
  recommended). The rest can only consist of letters \texttt{a-z,\ A-Z},
  numbers \texttt{0-9}, underscores \texttt{\_}, and periods \texttt{.}
\item
  A variable name cannot be a reserved word (\texttt{TRUE},
  \texttt{FALSE}, \texttt{if}, \texttt{for}, \texttt{while}, etc.)
\end{itemize}

\begin{Shaded}
\begin{Highlighting}[]
\CommentTok{\# Valid variable names}
\NormalTok{age            }
\NormalTok{height\_cm1    }
\NormalTok{weight2.kg    }

\CommentTok{\# Valid but not recommended}
\NormalTok{.data       }
\NormalTok{True}
\end{Highlighting}
\end{Shaded}

\section{Reserved Words}\label{reserved-words}

\begin{itemize}
\tightlist
\item
  \textbf{Reserved words} are words with special meaning and \ul{cannot}
  be used as variable names.
\end{itemize}

\begin{Shaded}
\begin{Highlighting}[]
\ControlFlowTok{if}\NormalTok{, }\ControlFlowTok{else}\NormalTok{, }\ControlFlowTok{for}\NormalTok{, }\ControlFlowTok{while}\NormalTok{, }\ControlFlowTok{function} 

\ControlFlowTok{break}\NormalTok{, stop, continue, }\ControlFlowTok{next}

\ConstantTok{TRUE}\NormalTok{, }\ConstantTok{FALSE}

\ConstantTok{NULL}\NormalTok{, }\ConstantTok{NA}\NormalTok{, }\ConstantTok{Inf}\NormalTok{, }\ConstantTok{NaN}
\end{Highlighting}
\end{Shaded}

💻 \textbf{Hands-On}

The following R code attempts to create variables. Run each line in the
console and explain why the variable name is invalid.

\begin{Shaded}
\begin{Highlighting}[]
\DecValTok{9}\NormalTok{null }\OtherTok{\textless{}{-}} \DecValTok{5}

\NormalTok{null\%}\DecValTok{9} \OtherTok{\textless{}{-}} \DecValTok{5}

\NormalTok{null }\DecValTok{9} \OtherTok{\textless{}{-}} \DecValTok{5}

\NormalTok{\_null9 }\OtherTok{\textless{}{-}} \DecValTok{5} 

\NormalTok{null,}\DecValTok{9} \OtherTok{\textless{}{-}} \DecValTok{5}

\ConstantTok{NULL} \OtherTok{\textless{}{-}} \DecValTok{5}
\end{Highlighting}
\end{Shaded}

\begin{tcolorbox}[enhanced jigsaw, colframe=quarto-callout-tip-color-frame, coltitle=black, left=2mm, rightrule=.15mm, colback=white, opacityback=0, toprule=.15mm, bottomtitle=1mm, colbacktitle=quarto-callout-tip-color!10!white, breakable, titlerule=0mm, title=\textcolor{quarto-callout-tip-color}{\faLightbulb}\hspace{0.5em}{Answer}, toptitle=1mm, arc=.35mm, bottomrule=.15mm, leftrule=.75mm, opacitybacktitle=0.6]

Note that all given variable names are invalid. See below for the
explanation.

\begin{Shaded}
\begin{Highlighting}[]
\DecValTok{9}\NormalTok{null }\OtherTok{\textless{}{-}} \DecValTok{5}     \CommentTok{\# variable name cannot start with a number}

\NormalTok{null\%}\DecValTok{9} \OtherTok{\textless{}{-}} \DecValTok{5}    \CommentTok{\# variable name cannot contain any percentage}

\NormalTok{null }\DecValTok{9} \OtherTok{\textless{}{-}} \DecValTok{5}    \CommentTok{\# variable name cannot contain any space}

\NormalTok{\_null9 }\OtherTok{\textless{}{-}} \DecValTok{5}    \CommentTok{\# variable name cannot start with an underscore}

\NormalTok{null,}\DecValTok{9} \OtherTok{\textless{}{-}} \DecValTok{5}    \CommentTok{\# variable name cannot contain any comma}

\ConstantTok{NULL} \OtherTok{\textless{}{-}} \DecValTok{5}      \CommentTok{\# variable name cannot be a reserved word}
\end{Highlighting}
\end{Shaded}

\end{tcolorbox}

\section{Naming Styles}\label{naming-styles}

Three common naming styles for variables include

\begin{Shaded}
\begin{Highlighting}[]
\NormalTok{eagles\_green\_swag    }\CommentTok{\# snake case}

\NormalTok{eaglesGreenSwag      }\CommentTok{\# camel case}

\NormalTok{eagles.green.swag    }\CommentTok{\# period case}
\end{Highlighting}
\end{Shaded}

💻 \textbf{Hands-On}

Create three variables that represent the phrase ``Crimson Roll Tide''
using the three naming styles.

\begin{tcolorbox}[enhanced jigsaw, colframe=quarto-callout-tip-color-frame, coltitle=black, left=2mm, rightrule=.15mm, colback=white, opacityback=0, toprule=.15mm, bottomtitle=1mm, colbacktitle=quarto-callout-tip-color!10!white, breakable, titlerule=0mm, title=\textcolor{quarto-callout-tip-color}{\faLightbulb}\hspace{0.5em}{Answer}, toptitle=1mm, arc=.35mm, bottomrule=.15mm, leftrule=.75mm, opacitybacktitle=0.6]

{[}Tidyverse style guide{]}\{https://style.tidyverse.org/syntax.html)
recommends the snake case, which is what I tend to use as well!

\begin{Shaded}
\begin{Highlighting}[]
\NormalTok{crimson\_roll\_tide    }\CommentTok{\# snake case}

\NormalTok{crimsonRollTide      }\CommentTok{\# camel case}

\NormalTok{crimson.roll.tide    }\CommentTok{\# period case}
\end{Highlighting}
\end{Shaded}

\end{tcolorbox}

\section{\texorpdfstring{\textbf{Functions}}{Functions}}\label{functions}

\begin{itemize}
\item
  A function is a collection of R code to perform a complex task or
  obtain certain results.
\item
  R has many useful built-in functions.
\end{itemize}

\begin{Shaded}
\begin{Highlighting}[]
\CommentTok{\# Absolute value, sign function, and squared root}
\NormalTok{abs, sign, sqrt}

\CommentTok{\# Rounding of numbers}
\NormalTok{round, floor, ceiling, trunc, signif}

\CommentTok{\# Logarithms and Exponentials}
\NormalTok{exp, log, expm1, log1p}

\CommentTok{\# Combination and factorial}
\NormalTok{choose, factorial, lchoose, lfactorial}

\CommentTok{\# Trigonometric and hyperbolic functions}
\NormalTok{cos, sin, tan, cospi, sinpi, tanpi, acos, asin, atan}
\NormalTok{cosh, sinh, tanh, acosh, asinh, atanh}

\CommentTok{\# Beta and gamma functions}
\NormalTok{beta, lbeta, gamma, lgamma, digamma, trigamma, psigamma}
\end{Highlighting}
\end{Shaded}

💻 \textbf{Hands-On}

Try the following code in the console.

\begin{Shaded}
\begin{Highlighting}[]
\FunctionTok{abs}\NormalTok{(}\SpecialCharTok{{-}}\DecValTok{7}\NormalTok{)         }\CommentTok{\# absolute value}

\FunctionTok{sqrt}\NormalTok{(}\DecValTok{25}\NormalTok{)        }\CommentTok{\# squared root}

\FunctionTok{exp}\NormalTok{(}\DecValTok{3}\NormalTok{)          }\CommentTok{\# exponential}

\FunctionTok{log}\NormalTok{(}\DecValTok{8}\NormalTok{)          }\CommentTok{\# natural log}

\FunctionTok{factorial}\NormalTok{(}\DecValTok{6}\NormalTok{)    }\CommentTok{\# 1 x 2 x 3 x 4 x 5 x 6}
\end{Highlighting}
\end{Shaded}

\begin{tcolorbox}[enhanced jigsaw, colframe=quarto-callout-tip-color-frame, coltitle=black, left=2mm, rightrule=.15mm, colback=white, opacityback=0, toprule=.15mm, bottomtitle=1mm, colbacktitle=quarto-callout-tip-color!10!white, breakable, titlerule=0mm, title=\textcolor{quarto-callout-tip-color}{\faLightbulb}\hspace{0.5em}{Answer}, toptitle=1mm, arc=.35mm, bottomrule=.15mm, leftrule=.75mm, opacitybacktitle=0.6]

\begin{Shaded}
\begin{Highlighting}[]
\FunctionTok{abs}\NormalTok{(}\SpecialCharTok{{-}}\DecValTok{7}\NormalTok{)         }\CommentTok{\# absolute value}
\end{Highlighting}
\end{Shaded}

\begin{verbatim}
[1] 7
\end{verbatim}

\begin{Shaded}
\begin{Highlighting}[]
\FunctionTok{sqrt}\NormalTok{(}\DecValTok{25}\NormalTok{)        }\CommentTok{\# squared root}
\end{Highlighting}
\end{Shaded}

\begin{verbatim}
[1] 5
\end{verbatim}

\begin{Shaded}
\begin{Highlighting}[]
\FunctionTok{exp}\NormalTok{(}\DecValTok{3}\NormalTok{)          }\CommentTok{\# exponential}
\end{Highlighting}
\end{Shaded}

\begin{verbatim}
[1] 20.08554
\end{verbatim}

\begin{Shaded}
\begin{Highlighting}[]
\FunctionTok{log}\NormalTok{(}\DecValTok{8}\NormalTok{)          }\CommentTok{\# natural log}
\end{Highlighting}
\end{Shaded}

\begin{verbatim}
[1] 2.079442
\end{verbatim}

\begin{Shaded}
\begin{Highlighting}[]
\FunctionTok{factorial}\NormalTok{(}\DecValTok{6}\NormalTok{)    }\CommentTok{\# 1 x 2 x 3 x 4 x 5 x 6}
\end{Highlighting}
\end{Shaded}

\begin{verbatim}
[1] 720
\end{verbatim}

\end{tcolorbox}

\section{Getting Help}\label{getting-help}

\begin{itemize}
\tightlist
\item
  If we want to learn more about a function, we can refer to its
  \textbf{help documentation}.
\end{itemize}

\begin{Shaded}
\begin{Highlighting}[]
\CommentTok{\# Using the operator \textquotesingle{}?\textquotesingle{}}
\NormalTok{?abs}
\NormalTok{?sqrt}

\CommentTok{\# Using the help() function}
\FunctionTok{help}\NormalTok{(abs)}
\FunctionTok{help}\NormalTok{(sqrt)}
\end{Highlighting}
\end{Shaded}

💻 \textbf{Hands-On}

Use the help documentation to explain the purpose of the following
functions.

\begin{Shaded}
\begin{Highlighting}[]
\FunctionTok{sign}\NormalTok{(}\SpecialCharTok{{-}}\DecValTok{17}\NormalTok{)}

\FunctionTok{log1p}\NormalTok{(}\DecValTok{3}\NormalTok{)}

\FunctionTok{choose}\NormalTok{(}\DecValTok{5}\NormalTok{, }\DecValTok{2}\NormalTok{)}
\end{Highlighting}
\end{Shaded}

\begin{tcolorbox}[enhanced jigsaw, colframe=quarto-callout-tip-color-frame, coltitle=black, left=2mm, rightrule=.15mm, colback=white, opacityback=0, toprule=.15mm, bottomtitle=1mm, colbacktitle=quarto-callout-tip-color!10!white, breakable, titlerule=0mm, title=\textcolor{quarto-callout-tip-color}{\faLightbulb}\hspace{0.5em}{Answer}, toptitle=1mm, arc=.35mm, bottomrule=.15mm, leftrule=.75mm, opacitybacktitle=0.6]

To get the help documentation of the functions, we can use \texttt{?} or
\texttt{help()}

\begin{Shaded}
\begin{Highlighting}[]
\NormalTok{?sign}
\FunctionTok{help}\NormalTok{(sign)}

\NormalTok{?log1p}
\FunctionTok{help}\NormalTok{(log1p)}

\NormalTok{?choose}
\FunctionTok{help}\NormalTok{(choose)}
\end{Highlighting}
\end{Shaded}

The functions do the following

\begin{Shaded}
\begin{Highlighting}[]
\FunctionTok{sign}\NormalTok{(}\SpecialCharTok{{-}}\DecValTok{17}\NormalTok{)       }\CommentTok{\# give {-}1 for negative number and 1 for positive number}
\end{Highlighting}
\end{Shaded}

\begin{verbatim}
[1] -1
\end{verbatim}

\begin{Shaded}
\begin{Highlighting}[]
\FunctionTok{log1p}\NormalTok{(}\DecValTok{3}\NormalTok{)        }\CommentTok{\# compute log(1 + 3)}
\end{Highlighting}
\end{Shaded}

\begin{verbatim}
[1] 1.386294
\end{verbatim}

\begin{Shaded}
\begin{Highlighting}[]
\FunctionTok{choose}\NormalTok{(}\DecValTok{5}\NormalTok{, }\DecValTok{2}\NormalTok{)    }\CommentTok{\# compute combination coefficient 5C2 }
\end{Highlighting}
\end{Shaded}

\begin{verbatim}
[1] 10
\end{verbatim}

\end{tcolorbox}

💻 \textbf{Hands-On}

What does the following code do? Use the help documentation of
\texttt{round()} to confirm your guess/observation.

\begin{Shaded}
\begin{Highlighting}[]
\FunctionTok{round}\NormalTok{(}\FloatTok{6.267}\NormalTok{)}

\FunctionTok{round}\NormalTok{(}\FloatTok{6.267}\NormalTok{, }\AttributeTok{digits =} \DecValTok{1}\NormalTok{)}
\FunctionTok{round}\NormalTok{(}\FloatTok{6.267}\NormalTok{, }\DecValTok{1}\NormalTok{)}

\FunctionTok{round}\NormalTok{(}\FloatTok{6.267}\NormalTok{, }\AttributeTok{digits =} \DecValTok{2}\NormalTok{)}
\FunctionTok{round}\NormalTok{(}\FloatTok{6.267}\NormalTok{, }\DecValTok{2}\NormalTok{)}
\end{Highlighting}
\end{Shaded}

\begin{tcolorbox}[enhanced jigsaw, colframe=quarto-callout-tip-color-frame, coltitle=black, left=2mm, rightrule=.15mm, colback=white, opacityback=0, toprule=.15mm, bottomtitle=1mm, colbacktitle=quarto-callout-tip-color!10!white, breakable, titlerule=0mm, title=\textcolor{quarto-callout-tip-color}{\faLightbulb}\hspace{0.5em}{Answer}, toptitle=1mm, arc=.35mm, bottomrule=.15mm, leftrule=.75mm, opacitybacktitle=0.6]

To get the help documentation of \texttt{round()}, we can use \texttt{?}
or \texttt{help()}. It is actually pretty self-explanatory!

\begin{Shaded}
\begin{Highlighting}[]
\NormalTok{?round}
\FunctionTok{help}\NormalTok{(round)}
\end{Highlighting}
\end{Shaded}

The given code does the following

\begin{Shaded}
\begin{Highlighting}[]
\CommentTok{\# Round to the whole number (digits = 0)}
\FunctionTok{round}\NormalTok{(}\FloatTok{6.267}\NormalTok{)}
\end{Highlighting}
\end{Shaded}

\begin{verbatim}
[1] 6
\end{verbatim}

\begin{Shaded}
\begin{Highlighting}[]
\CommentTok{\# Round to 1 decimal place}
\FunctionTok{round}\NormalTok{(}\FloatTok{6.267}\NormalTok{, }\AttributeTok{digits =} \DecValTok{1}\NormalTok{)}
\end{Highlighting}
\end{Shaded}

\begin{verbatim}
[1] 6.3
\end{verbatim}

\begin{Shaded}
\begin{Highlighting}[]
\CommentTok{\# Round to 2 decimal places}
\FunctionTok{round}\NormalTok{(}\FloatTok{6.267}\NormalTok{, }\AttributeTok{digits =} \DecValTok{2}\NormalTok{)}
\end{Highlighting}
\end{Shaded}

\begin{verbatim}
[1] 6.27
\end{verbatim}

\end{tcolorbox}

\section{Packages}\label{packages}

\begin{itemize}
\item
  A \textbf{package} is a collection of functions and data for specific
  purpose.
\item
  Most developers make their packages freely available on the
  Comprehensive R Archive Network (CRAN).
\end{itemize}

\begin{Shaded}
\begin{Highlighting}[]
\CommentTok{\# Install a package}
\FunctionTok{install.packages}\NormalTok{(}\StringTok{"bannerCommenter"}\NormalTok{)}

\CommentTok{\# Load an installed package}
\FunctionTok{library}\NormalTok{(bannerCommenter)}
\FunctionTok{library}\NormalTok{(}\StringTok{"bannerCommenter"}\NormalTok{)}

\CommentTok{\# Uninstall a package}
\FunctionTok{remove.packages}\NormalTok{(}\StringTok{"bannerCommenter"}\NormalTok{)}
\end{Highlighting}
\end{Shaded}

\chapter{2A: Vector}\label{a-vector}

\section{Readings}\label{readings}

From \textbf{R Coding Basics: An Introduction to the Basics of Coding in
R} by Dr.~Gaston Sanchez:

\begin{itemize}
\item
  \href{https://www.gastonsanchez.com/R-coding-basics/1-01-vectors-intro.html}{First
  Contact with Vectors}
\item
  \href{https://www.gastonsanchez.com/R-coding-basics/1-02-vectors-properties.html}{Properties
  of Vectors}
\item
  \href{https://www.gastonsanchez.com/R-coding-basics/1-03-vectors-creation.html}{Creating
  Vectors}
\item
  \href{https://www.gastonsanchez.com/R-coding-basics/1-04-vectors-concepts.html}{More
  about Vectors}
\end{itemize}

\section{Topics}\label{topics-1}

\begin{itemize}
\item
  Vectors
\item
  Atomic types
\item
  Special values
\item
  Creating vectors with \texttt{c()}, \texttt{:}, \texttt{seq()} , and
  \texttt{rep()}
\item
  Useful functions for numeric vectors
\item
  Built-in vectors
\end{itemize}

\section{Basic data structures}\label{basic-data-structures}

\begin{itemize}
\tightlist
\item
  Basic data structures in R include vector, factor, matrix, array, data
  frame, and list.
\end{itemize}

\begin{center}
\includegraphics[width=6.79167in,height=\textheight]{index_files/mediabag/r_data_structures.png}
\end{center}

\begin{itemize}
\tightlist
\item
  These structures are characterized by their dimension and whether they
  require all elements to be of the same atomic type.
\end{itemize}

\begin{longtable}[]{@{}lcc@{}}
\toprule\noalign{}
Structure & Dimension & Same Atomic Type \\
\midrule\noalign{}
\endhead
\bottomrule\noalign{}
\endlastfoot
Vector & 1 & Yes \\
Factor & 1 & Yes \\
Matrix & 2 & Yes \\
Data Frame & 2 & No \\
Array & \(\ge\) 2 & Yes \\
List & 1 & No \\
\end{longtable}

\section{Vector}\label{vector}

\begin{itemize}
\item
  A \textbf{vector} is a sequence of elements that are of the \ul{same}
  \textbf{atomic} \textbf{type}.
\item
  In R, the index of the first element is always \textbf{1}.
\end{itemize}

\begin{Shaded}
\begin{Highlighting}[]
\CommentTok{\#  1   2   3   4   5   6   7}
\FunctionTok{c}\NormalTok{(}\DecValTok{70}\NormalTok{, }\DecValTok{66}\NormalTok{, }\DecValTok{82}\NormalTok{, }\DecValTok{85}\NormalTok{, }\DecValTok{78}\NormalTok{, }\DecValTok{90}\NormalTok{, }\DecValTok{73}\NormalTok{)}
\end{Highlighting}
\end{Shaded}

\begin{itemize}
\tightlist
\item
  A single value is treated as a vector of one element
\end{itemize}

\begin{Shaded}
\begin{Highlighting}[]
\DecValTok{70}    \CommentTok{\# same as c(70)}
\end{Highlighting}
\end{Shaded}

\section{Atomic types}\label{atomic-types}

\begin{itemize}
\item
  An \textbf{atomic type} refers to the six fundamental types in R:
  \texttt{logical}, \texttt{integer}, \texttt{double},
  \texttt{character}, \texttt{raw}, and \texttt{complex}.
\item
  Note that \texttt{integer} and \texttt{double} are also known as
  \texttt{numeric}.
\end{itemize}

\begin{Shaded}
\begin{Highlighting}[]
\CommentTok{\# logical vector}
\FunctionTok{c}\NormalTok{(}\ConstantTok{TRUE}\NormalTok{, }\ConstantTok{FALSE}\NormalTok{, }\ConstantTok{FALSE}\NormalTok{, }\ConstantTok{TRUE}\NormalTok{, }\ConstantTok{TRUE}\NormalTok{)}
\FunctionTok{c}\NormalTok{(T, F, F, T, T)}

\CommentTok{\# integer vector (numeric)}
\FunctionTok{c}\NormalTok{(}\DecValTok{1}\NormalTok{L, }\DecValTok{3}\NormalTok{L, }\DecValTok{2}\NormalTok{L, }\DecValTok{4}\NormalTok{L, }\DecValTok{2}\NormalTok{L)}

\CommentTok{\# double vector (numeric)}
\FunctionTok{c}\NormalTok{(}\FloatTok{6.3}\NormalTok{, }\FloatTok{8.2}\NormalTok{, }\FloatTok{3.1}\NormalTok{, }\FloatTok{4.4}\NormalTok{, }\FloatTok{7.6}\NormalTok{)}

\CommentTok{\# character vector}
\FunctionTok{c}\NormalTok{(}\StringTok{"apple"}\NormalTok{, }\StringTok{"orange"}\NormalTok{, }\StringTok{"apple"}\NormalTok{, }\StringTok{"apple"}\NormalTok{, }\StringTok{"orange"}\NormalTok{)  }
\FunctionTok{c}\NormalTok{(}\StringTok{\textquotesingle{}dog\textquotesingle{}}\NormalTok{, }\StringTok{\textquotesingle{}cat\textquotesingle{}}\NormalTok{, }\StringTok{\textquotesingle{}dog\textquotesingle{}}\NormalTok{, }\StringTok{\textquotesingle{}dog\textquotesingle{}}\NormalTok{, }\StringTok{\textquotesingle{}cat\textquotesingle{}}\NormalTok{)  }

\CommentTok{\# raw and complex exist but not very popular}
\end{Highlighting}
\end{Shaded}

\begin{itemize}
\item
  The functions \texttt{typeof()} and \texttt{storage.mode()} tells us
  the atomic type of a vector.
\item
  The function \texttt{mode()} works similarly, except that it returns
  \texttt{numeric} for both \texttt{integer} and \texttt{double}.
\end{itemize}

\begin{Shaded}
\begin{Highlighting}[]
\FunctionTok{typeof}\NormalTok{(}\FloatTok{2.3}\NormalTok{)}
\end{Highlighting}
\end{Shaded}

\begin{verbatim}
[1] "double"
\end{verbatim}

\begin{Shaded}
\begin{Highlighting}[]
\FunctionTok{storage.mode}\NormalTok{(}\FloatTok{2.3}\NormalTok{)}
\end{Highlighting}
\end{Shaded}

\begin{verbatim}
[1] "double"
\end{verbatim}

\begin{Shaded}
\begin{Highlighting}[]
\FunctionTok{mode}\NormalTok{(}\FloatTok{2.3}\NormalTok{)}
\end{Highlighting}
\end{Shaded}

\begin{verbatim}
[1] "numeric"
\end{verbatim}

\begin{itemize}
\tightlist
\item
  There are also dedicated checking function for each atomic type.
\end{itemize}

\begin{Shaded}
\begin{Highlighting}[]
\FunctionTok{is.logical}\NormalTok{()}

\FunctionTok{is.integer}\NormalTok{()    }\CommentTok{\# integer but not double}
\FunctionTok{is.double}\NormalTok{()     }\CommentTok{\# double but not integer}
\FunctionTok{is.numeric}\NormalTok{()    }\CommentTok{\# either integer or double}

\FunctionTok{is.character}\NormalTok{()}
\end{Highlighting}
\end{Shaded}

💻 \textbf{Hands-On}

Try the following R code to see what it returns.

\begin{Shaded}
\begin{Highlighting}[]
\NormalTok{enrollment }\OtherTok{\textless{}{-}} \FunctionTok{c}\NormalTok{(}\DecValTok{10}\NormalTok{, }\DecValTok{30}\NormalTok{, }\DecValTok{15}\NormalTok{, }\DecValTok{20}\NormalTok{)}

\FunctionTok{typeof}\NormalTok{(enrollment)}

\FunctionTok{storage.mode}\NormalTok{(enrollment)}

\FunctionTok{mode}\NormalTok{(enrollment)}

\FunctionTok{is.integer}\NormalTok{(enrollment)}

\FunctionTok{is.double}\NormalTok{(enrollment)}

\FunctionTok{is.numeric}\NormalTok{(enrollment)   }
\end{Highlighting}
\end{Shaded}

\begin{tcolorbox}[enhanced jigsaw, colframe=quarto-callout-tip-color-frame, coltitle=black, left=2mm, rightrule=.15mm, colback=white, opacityback=0, toprule=.15mm, bottomtitle=1mm, colbacktitle=quarto-callout-tip-color!10!white, breakable, titlerule=0mm, title=\textcolor{quarto-callout-tip-color}{\faLightbulb}\hspace{0.5em}{Answer}, toptitle=1mm, arc=.35mm, bottomrule=.15mm, leftrule=.75mm, opacitybacktitle=0.6]

Note that \texttt{enrollment} contains whole numbers but without writing
\texttt{c(10L,\ 30L,\ 15L,\ 20L)}, R still thinks of them as
\texttt{double}.

\begin{Shaded}
\begin{Highlighting}[]
\NormalTok{enrollment }\OtherTok{\textless{}{-}} \FunctionTok{c}\NormalTok{(}\DecValTok{10}\NormalTok{, }\DecValTok{30}\NormalTok{, }\DecValTok{15}\NormalTok{, }\DecValTok{20}\NormalTok{)}

\FunctionTok{typeof}\NormalTok{(enrollment)}
\end{Highlighting}
\end{Shaded}

\begin{verbatim}
[1] "double"
\end{verbatim}

\begin{Shaded}
\begin{Highlighting}[]
\FunctionTok{storage.mode}\NormalTok{(enrollment)}
\end{Highlighting}
\end{Shaded}

\begin{verbatim}
[1] "double"
\end{verbatim}

\begin{Shaded}
\begin{Highlighting}[]
\FunctionTok{mode}\NormalTok{(enrollment)}
\end{Highlighting}
\end{Shaded}

\begin{verbatim}
[1] "numeric"
\end{verbatim}

\begin{Shaded}
\begin{Highlighting}[]
\FunctionTok{is.integer}\NormalTok{(enrollment)}
\end{Highlighting}
\end{Shaded}

\begin{verbatim}
[1] FALSE
\end{verbatim}

\begin{Shaded}
\begin{Highlighting}[]
\FunctionTok{is.double}\NormalTok{(enrollment)}
\end{Highlighting}
\end{Shaded}

\begin{verbatim}
[1] TRUE
\end{verbatim}

\begin{Shaded}
\begin{Highlighting}[]
\FunctionTok{is.numeric}\NormalTok{(enrollment)  }
\end{Highlighting}
\end{Shaded}

\begin{verbatim}
[1] TRUE
\end{verbatim}

\end{tcolorbox}

\section{Special values}\label{special-values}

\begin{itemize}
\item
  \texttt{NULL} indicates an undefined object
\item
  \texttt{NA} indicates missing or ``not available'' value
\item
  \texttt{NaN} indicates an object that is ``not a number''
\item
  \texttt{Inf} indicates positive infinite
\item
  \texttt{-Inf} indicates negative infinite
\end{itemize}

💻 \textbf{Hands-On}

Try the following R code to see what it returns.

\begin{Shaded}
\begin{Highlighting}[]
\FunctionTok{sqrt}\NormalTok{(}\SpecialCharTok{{-}}\DecValTok{7}\NormalTok{)}

\FunctionTok{log}\NormalTok{(}\SpecialCharTok{{-}}\DecValTok{5}\NormalTok{)}

\DecValTok{0} \SpecialCharTok{/} \DecValTok{0}

\DecValTok{100} \SpecialCharTok{/} \DecValTok{0}

\SpecialCharTok{{-}}\DecValTok{100} \SpecialCharTok{/} \DecValTok{0}

\FunctionTok{log}\NormalTok{(}\DecValTok{0}\NormalTok{)}
\end{Highlighting}
\end{Shaded}

\begin{tcolorbox}[enhanced jigsaw, colframe=quarto-callout-tip-color-frame, coltitle=black, left=2mm, rightrule=.15mm, colback=white, opacityback=0, toprule=.15mm, bottomtitle=1mm, colbacktitle=quarto-callout-tip-color!10!white, breakable, titlerule=0mm, title=\textcolor{quarto-callout-tip-color}{\faLightbulb}\hspace{0.5em}{Answer}, toptitle=1mm, arc=.35mm, bottomrule=.15mm, leftrule=.75mm, opacitybacktitle=0.6]

\begin{Shaded}
\begin{Highlighting}[]
\FunctionTok{sqrt}\NormalTok{(}\SpecialCharTok{{-}}\DecValTok{7}\NormalTok{)}
\end{Highlighting}
\end{Shaded}

\begin{verbatim}
[1] NaN
\end{verbatim}

\begin{Shaded}
\begin{Highlighting}[]
\FunctionTok{log}\NormalTok{(}\SpecialCharTok{{-}}\DecValTok{5}\NormalTok{)}
\end{Highlighting}
\end{Shaded}

\begin{verbatim}
[1] NaN
\end{verbatim}

\begin{Shaded}
\begin{Highlighting}[]
\DecValTok{0} \SpecialCharTok{/} \DecValTok{0}
\end{Highlighting}
\end{Shaded}

\begin{verbatim}
[1] NaN
\end{verbatim}

\begin{Shaded}
\begin{Highlighting}[]
\DecValTok{100} \SpecialCharTok{/} \DecValTok{0}
\end{Highlighting}
\end{Shaded}

\begin{verbatim}
[1] Inf
\end{verbatim}

\begin{Shaded}
\begin{Highlighting}[]
\SpecialCharTok{{-}}\DecValTok{100} \SpecialCharTok{/} \DecValTok{0}
\end{Highlighting}
\end{Shaded}

\begin{verbatim}
[1] -Inf
\end{verbatim}

\begin{Shaded}
\begin{Highlighting}[]
\FunctionTok{log}\NormalTok{(}\DecValTok{0}\NormalTok{)}
\end{Highlighting}
\end{Shaded}

\begin{verbatim}
[1] -Inf
\end{verbatim}

\end{tcolorbox}

\section{Creating vectors}\label{creating-vectors}

\begin{itemize}
\tightlist
\item
  As shown previously, a vector can be manually created using the
  combine \texttt{c()} function.
\end{itemize}

\begin{Shaded}
\begin{Highlighting}[]
\CommentTok{\# logical vector}
\FunctionTok{c}\NormalTok{(}\ConstantTok{TRUE}\NormalTok{, }\ConstantTok{FALSE}\NormalTok{, }\ConstantTok{FALSE}\NormalTok{, }\ConstantTok{TRUE}\NormalTok{, }\ConstantTok{TRUE}\NormalTok{)}
\FunctionTok{c}\NormalTok{(T, F, F, T, T)}

\CommentTok{\# integer vector (numeric)}
\FunctionTok{c}\NormalTok{(}\DecValTok{1}\NormalTok{L, }\DecValTok{3}\NormalTok{L, }\DecValTok{2}\NormalTok{L, }\DecValTok{4}\NormalTok{L, }\DecValTok{2}\NormalTok{L)}

\CommentTok{\# double vector (numeric)}
\FunctionTok{c}\NormalTok{(}\FloatTok{6.3}\NormalTok{, }\FloatTok{8.2}\NormalTok{, }\FloatTok{3.1}\NormalTok{, }\FloatTok{4.4}\NormalTok{, }\FloatTok{7.6}\NormalTok{)}

\CommentTok{\# character vector}
\FunctionTok{c}\NormalTok{(}\StringTok{"apple"}\NormalTok{, }\StringTok{"orange"}\NormalTok{, }\StringTok{"apple"}\NormalTok{, }\StringTok{"apple"}\NormalTok{, }\StringTok{"orange"}\NormalTok{)  }
\FunctionTok{c}\NormalTok{(}\StringTok{\textquotesingle{}dog\textquotesingle{}}\NormalTok{, }\StringTok{\textquotesingle{}cat\textquotesingle{}}\NormalTok{, }\StringTok{\textquotesingle{}dog\textquotesingle{}}\NormalTok{, }\StringTok{\textquotesingle{}dog\textquotesingle{}}\NormalTok{, }\StringTok{\textquotesingle{}cat\textquotesingle{}}\NormalTok{) }
\end{Highlighting}
\end{Shaded}

\begin{itemize}
\item
  Elements in a vector can have names!
\item
  We can give names directly in \texttt{c()}
\end{itemize}

\begin{Shaded}
\begin{Highlighting}[]
\FunctionTok{c}\NormalTok{(}\AttributeTok{exam1 =} \DecValTok{90}\NormalTok{, }\AttributeTok{exam2 =} \DecValTok{85}\NormalTok{, }\AttributeTok{final =} \DecValTok{92}\NormalTok{)}
\end{Highlighting}
\end{Shaded}

\begin{verbatim}
exam1 exam2 final 
   90    85    92 
\end{verbatim}

\begin{itemize}
\tightlist
\item
  We can also create the vector and assign names later.
\end{itemize}

\begin{Shaded}
\begin{Highlighting}[]
\NormalTok{scores }\OtherTok{\textless{}{-}} \FunctionTok{c}\NormalTok{(}\DecValTok{90}\NormalTok{, }\DecValTok{85}\NormalTok{, }\DecValTok{92}\NormalTok{)}
\FunctionTok{names}\NormalTok{(scores) }\OtherTok{\textless{}{-}} \FunctionTok{c}\NormalTok{(}\StringTok{\textquotesingle{}exam1\textquotesingle{}}\NormalTok{, }\StringTok{\textquotesingle{}exam2\textquotesingle{}}\NormalTok{, }\StringTok{\textquotesingle{}exam3\textquotesingle{}}\NormalTok{)}
\NormalTok{scores}
\end{Highlighting}
\end{Shaded}

\begin{verbatim}
exam1 exam2 exam3 
   90    85    92 
\end{verbatim}

💻 \textbf{Hands-On}

\begin{itemize}
\item
  Use \texttt{c()} to create a short vector for each of the four atomic
  types: \texttt{integer}, \texttt{double}, \texttt{logical}, and
  \texttt{character}.
\item
  Assign each vector to a variable with a descriptive name.
\item
  Choose one of your vectors and assign names to its elements.
\end{itemize}

\begin{tcolorbox}[enhanced jigsaw, colframe=quarto-callout-tip-color-frame, coltitle=black, left=2mm, rightrule=.15mm, colback=white, opacityback=0, toprule=.15mm, bottomtitle=1mm, colbacktitle=quarto-callout-tip-color!10!white, breakable, titlerule=0mm, title=\textcolor{quarto-callout-tip-color}{\faLightbulb}\hspace{0.5em}{Answer}, toptitle=1mm, arc=.35mm, bottomrule=.15mm, leftrule=.75mm, opacitybacktitle=0.6]

\begin{Shaded}
\begin{Highlighting}[]
\CommentTok{\# integer}
\NormalTok{experience }\OtherTok{\textless{}{-}} \FunctionTok{c}\NormalTok{(}\DecValTok{1}\NormalTok{L, }\DecValTok{3}\NormalTok{L, }\DecValTok{5}\NormalTok{L, }\DecValTok{2}\NormalTok{L)}

\CommentTok{\# double}
\NormalTok{weight }\OtherTok{\textless{}{-}} \FunctionTok{c}\NormalTok{(}\FloatTok{143.5}\NormalTok{, }\DecValTok{150}\NormalTok{, }\FloatTok{127.3}\NormalTok{, }\FloatTok{133.5}\NormalTok{)}

\CommentTok{\# logical}
\NormalTok{in\_stock }\OtherTok{\textless{}{-}} \FunctionTok{c}\NormalTok{(}\ConstantTok{TRUE}\NormalTok{, }\ConstantTok{FALSE}\NormalTok{, }\ConstantTok{FALSE}\NormalTok{, }\ConstantTok{TRUE}\NormalTok{)}

\CommentTok{\# character}
\NormalTok{student\_levels }\OtherTok{\textless{}{-}} \FunctionTok{c}\NormalTok{(}\StringTok{\textquotesingle{}Junior\textquotesingle{}}\NormalTok{, }\StringTok{\textquotesingle{}Freshman\textquotesingle{}}\NormalTok{, }\StringTok{\textquotesingle{}Junior\textquotesingle{}}\NormalTok{, }\StringTok{\textquotesingle{}Senior\textquotesingle{}}\NormalTok{)}
\end{Highlighting}
\end{Shaded}

\end{tcolorbox}

\section{Creating numeric vectors}\label{creating-numeric-vectors}

\subsection{The colon operator}\label{the-colon-operator}

\begin{itemize}
\tightlist
\item
  The colon operator \texttt{:} generate a numeric sequence of one-unit
  steps by
\end{itemize}

\begin{Shaded}
\begin{Highlighting}[]
\DocumentationTok{\#\#\# start:end (end is like an upper/lower bound)}

\SpecialCharTok{{-}}\DecValTok{2}\SpecialCharTok{:}\DecValTok{5}       \CommentTok{\# start with {-}2, increase by 1}
 
\DecValTok{5}\SpecialCharTok{:{-}}\DecValTok{2}       \CommentTok{\# start with 5, decrease by 1}

\FloatTok{3.7}\SpecialCharTok{:}\FloatTok{9.2}    \CommentTok{\# start with 3.7, increase by 1}
\end{Highlighting}
\end{Shaded}

💻 \textbf{Hands-On}

Use the colon operator \texttt{:} to quickly create the following
vectors

\begin{Shaded}
\begin{Highlighting}[]
\FunctionTok{c}\NormalTok{(}\DecValTok{3}\NormalTok{, }\DecValTok{4}\NormalTok{, }\DecValTok{5}\NormalTok{, }\DecValTok{6}\NormalTok{, }\DecValTok{7}\NormalTok{, }\DecValTok{8}\NormalTok{, }\DecValTok{9}\NormalTok{, }\DecValTok{10}\NormalTok{, }\DecValTok{11}\NormalTok{, }\DecValTok{12}\NormalTok{)}

\FunctionTok{c}\NormalTok{(}\DecValTok{17}\NormalTok{, }\DecValTok{16}\NormalTok{, }\DecValTok{15}\NormalTok{, }\DecValTok{14}\NormalTok{, }\DecValTok{13}\NormalTok{, }\DecValTok{12}\NormalTok{, }\DecValTok{11}\NormalTok{, }\DecValTok{10}\NormalTok{, }\DecValTok{9}\NormalTok{)}
\end{Highlighting}
\end{Shaded}

\begin{tcolorbox}[enhanced jigsaw, colframe=quarto-callout-tip-color-frame, coltitle=black, left=2mm, rightrule=.15mm, colback=white, opacityback=0, toprule=.15mm, bottomtitle=1mm, colbacktitle=quarto-callout-tip-color!10!white, breakable, titlerule=0mm, title=\textcolor{quarto-callout-tip-color}{\faLightbulb}\hspace{0.5em}{Answer}, toptitle=1mm, arc=.35mm, bottomrule=.15mm, leftrule=.75mm, opacitybacktitle=0.6]

Since the vectors contain consecutive elements, the colon operator
\texttt{:} is useful.

\begin{Shaded}
\begin{Highlighting}[]
\DecValTok{3}\SpecialCharTok{:}\DecValTok{12}   
\end{Highlighting}
\end{Shaded}

\begin{verbatim}
 [1]  3  4  5  6  7  8  9 10 11 12
\end{verbatim}

\begin{Shaded}
\begin{Highlighting}[]
\DecValTok{17}\SpecialCharTok{:}\DecValTok{9}   
\end{Highlighting}
\end{Shaded}

\begin{verbatim}
[1] 17 16 15 14 13 12 11 10  9
\end{verbatim}

\end{tcolorbox}

\subsection{\texorpdfstring{The \texttt{seq()}
function}{The seq() function}}\label{the-seq-function}

\begin{itemize}
\tightlist
\item
  The \texttt{seq()} function generates a numeric sequence of more
  general steps.
\end{itemize}

\begin{Shaded}
\begin{Highlighting}[]
\CommentTok{\# step size of 2}
\FunctionTok{seq}\NormalTok{(}\AttributeTok{from =} \SpecialCharTok{{-}}\DecValTok{2}\NormalTok{, }\AttributeTok{to =} \DecValTok{5}\NormalTok{, }\AttributeTok{by =} \DecValTok{2}\NormalTok{)            }
\end{Highlighting}
\end{Shaded}

\begin{verbatim}
[1] -2  0  2  4
\end{verbatim}

\begin{Shaded}
\begin{Highlighting}[]
\CommentTok{\# step size of 0.75}
\FunctionTok{seq}\NormalTok{(}\AttributeTok{from =} \SpecialCharTok{{-}}\DecValTok{2}\NormalTok{, }\AttributeTok{to =} \DecValTok{5}\NormalTok{, }\AttributeTok{by =} \FloatTok{0.75}\NormalTok{)         }
\end{Highlighting}
\end{Shaded}

\begin{verbatim}
 [1] -2.00 -1.25 -0.50  0.25  1.00  1.75  2.50  3.25  4.00  4.75
\end{verbatim}

\begin{Shaded}
\begin{Highlighting}[]
\CommentTok{\# steps are automatically adjusted}
\FunctionTok{seq}\NormalTok{(}\AttributeTok{from =} \SpecialCharTok{{-}}\DecValTok{2}\NormalTok{, }\AttributeTok{to =} \DecValTok{5}\NormalTok{, }\AttributeTok{length.out =} \DecValTok{6}\NormalTok{)    }
\end{Highlighting}
\end{Shaded}

\begin{verbatim}
[1] -2.0 -0.6  0.8  2.2  3.6  5.0
\end{verbatim}

💻 \textbf{Hands-On}

Use the \texttt{seq()} function to create the vector that

\begin{itemize}
\item
  Starts at \texttt{5} and ends at \texttt{-3}, increasing or decreasing
  by \texttt{1}
\item
  Starts at \texttt{-1} and ends at \texttt{7}, with a step size of
  \texttt{1.5}.
\item
  Starts at \texttt{0} and ends at \texttt{14}, containing \texttt{5}
  equally spaced values.
\item
  Starts at \texttt{-4} and ends at \texttt{4}, including only every
  other number.
\end{itemize}

\begin{tcolorbox}[enhanced jigsaw, colframe=quarto-callout-tip-color-frame, coltitle=black, left=2mm, rightrule=.15mm, colback=white, opacityback=0, toprule=.15mm, bottomtitle=1mm, colbacktitle=quarto-callout-tip-color!10!white, breakable, titlerule=0mm, title=\textcolor{quarto-callout-tip-color}{\faLightbulb}\hspace{0.5em}{Answer}, toptitle=1mm, arc=.35mm, bottomrule=.15mm, leftrule=.75mm, opacitybacktitle=0.6]

\begin{Shaded}
\begin{Highlighting}[]
\FunctionTok{seq}\NormalTok{(}\AttributeTok{from =} \DecValTok{5}\NormalTok{, }\AttributeTok{to =} \SpecialCharTok{{-}}\DecValTok{3}\NormalTok{, }\AttributeTok{by =} \SpecialCharTok{{-}}\DecValTok{1}\NormalTok{)}
\end{Highlighting}
\end{Shaded}

\begin{verbatim}
[1]  5  4  3  2  1  0 -1 -2 -3
\end{verbatim}

\begin{Shaded}
\begin{Highlighting}[]
\FunctionTok{seq}\NormalTok{(}\AttributeTok{from =} \SpecialCharTok{{-}}\DecValTok{1}\NormalTok{, }\AttributeTok{to =} \DecValTok{7}\NormalTok{, }\AttributeTok{by =} \FloatTok{1.5}\NormalTok{)}
\end{Highlighting}
\end{Shaded}

\begin{verbatim}
[1] -1.0  0.5  2.0  3.5  5.0  6.5
\end{verbatim}

\begin{Shaded}
\begin{Highlighting}[]
\FunctionTok{seq}\NormalTok{(}\AttributeTok{from =} \DecValTok{0}\NormalTok{, }\AttributeTok{to =} \DecValTok{14}\NormalTok{, }\AttributeTok{length.out =} \DecValTok{5}\NormalTok{)}
\end{Highlighting}
\end{Shaded}

\begin{verbatim}
[1]  0.0  3.5  7.0 10.5 14.0
\end{verbatim}

\begin{Shaded}
\begin{Highlighting}[]
\FunctionTok{seq}\NormalTok{(}\AttributeTok{from =} \SpecialCharTok{{-}}\DecValTok{4}\NormalTok{, }\AttributeTok{to =} \DecValTok{4}\NormalTok{, }\AttributeTok{by =} \DecValTok{2}\NormalTok{)}
\end{Highlighting}
\end{Shaded}

\begin{verbatim}
[1] -4 -2  0  2  4
\end{verbatim}

\end{tcolorbox}

\subsection{\texorpdfstring{The \texttt{rep()}
function}{The rep() function}}\label{the-rep-function}

\begin{itemize}
\tightlist
\item
  The \texttt{rep()} function creates vectors with repeated elements.
\end{itemize}

\begin{Shaded}
\begin{Highlighting}[]
\CommentTok{\# repeat {-}1 five times}
\FunctionTok{rep}\NormalTok{(}\SpecialCharTok{{-}}\DecValTok{1}\NormalTok{, }\AttributeTok{times =} \DecValTok{5}\NormalTok{)                      }
\end{Highlighting}
\end{Shaded}

\begin{verbatim}
[1] -1 -1 -1 -1 -1
\end{verbatim}

\begin{Shaded}
\begin{Highlighting}[]
\CommentTok{\# repeat c({-}1, 0, 3) four times}
\FunctionTok{rep}\NormalTok{(}\FunctionTok{c}\NormalTok{(}\SpecialCharTok{{-}}\DecValTok{1}\NormalTok{, }\DecValTok{0}\NormalTok{, }\DecValTok{3}\NormalTok{), }\AttributeTok{times =} \DecValTok{4}\NormalTok{)             }
\end{Highlighting}
\end{Shaded}

\begin{verbatim}
 [1] -1  0  3 -1  0  3 -1  0  3 -1  0  3
\end{verbatim}

\begin{Shaded}
\begin{Highlighting}[]
\CommentTok{\# repeat {-}1 two times, 0 three times, 3 four times}
\FunctionTok{rep}\NormalTok{(}\FunctionTok{c}\NormalTok{(}\SpecialCharTok{{-}}\DecValTok{1}\NormalTok{, }\DecValTok{0}\NormalTok{, }\DecValTok{3}\NormalTok{), }\AttributeTok{times =} \FunctionTok{c}\NormalTok{(}\DecValTok{2}\NormalTok{, }\DecValTok{3}\NormalTok{, }\DecValTok{4}\NormalTok{))    }
\end{Highlighting}
\end{Shaded}

\begin{verbatim}
[1] -1 -1  0  0  0  3  3  3  3
\end{verbatim}

\begin{Shaded}
\begin{Highlighting}[]
\CommentTok{\# repeat {-}1 five times, 0 five times, 3 five times}
\FunctionTok{rep}\NormalTok{(}\FunctionTok{c}\NormalTok{(}\SpecialCharTok{{-}}\DecValTok{1}\NormalTok{, }\DecValTok{0}\NormalTok{, }\DecValTok{3}\NormalTok{), }\AttributeTok{each =} \DecValTok{5}\NormalTok{)              }
\end{Highlighting}
\end{Shaded}

\begin{verbatim}
 [1] -1 -1 -1 -1 -1  0  0  0  0  0  3  3  3  3  3
\end{verbatim}

💻 \textbf{Hands-On}

Use the \texttt{rep()} function to create the vector in which

\begin{itemize}
\item
  Each value in the vector \texttt{c(1,\ 3,\ 6)} is repeated exactly 4
  times
\item
  The value \texttt{4} is repeated \texttt{6} times
\item
  The vector \texttt{c(2,\ −1,\ 1)} is repeated \texttt{3} times
\item
  The values in \texttt{c(5,\ 0,\ −2)} are repeated so that \texttt{5}
  appears once, \texttt{0} appears three times, and \texttt{-2} appears
  four times.
\end{itemize}

\textbf{Answer:}

\begin{tcolorbox}[enhanced jigsaw, colframe=quarto-callout-tip-color-frame, coltitle=black, left=2mm, rightrule=.15mm, colback=white, opacityback=0, toprule=.15mm, bottomtitle=1mm, colbacktitle=quarto-callout-tip-color!10!white, breakable, titlerule=0mm, title=\textcolor{quarto-callout-tip-color}{\faLightbulb}\hspace{0.5em}{Answer}, toptitle=1mm, arc=.35mm, bottomrule=.15mm, leftrule=.75mm, opacitybacktitle=0.6]

\begin{Shaded}
\begin{Highlighting}[]
\FunctionTok{rep}\NormalTok{(}\FunctionTok{c}\NormalTok{(}\DecValTok{1}\NormalTok{, }\DecValTok{3}\NormalTok{, }\DecValTok{6}\NormalTok{), }\AttributeTok{each =} \DecValTok{4}\NormalTok{)}
\end{Highlighting}
\end{Shaded}

\begin{verbatim}
 [1] 1 1 1 1 3 3 3 3 6 6 6 6
\end{verbatim}

\begin{Shaded}
\begin{Highlighting}[]
\FunctionTok{rep}\NormalTok{(}\DecValTok{4}\NormalTok{, }\AttributeTok{times =} \DecValTok{6}\NormalTok{)}
\end{Highlighting}
\end{Shaded}

\begin{verbatim}
[1] 4 4 4 4 4 4
\end{verbatim}

\begin{Shaded}
\begin{Highlighting}[]
\FunctionTok{rep}\NormalTok{(}\FunctionTok{c}\NormalTok{(}\DecValTok{2}\NormalTok{, }\SpecialCharTok{{-}}\DecValTok{1}\NormalTok{, }\DecValTok{1}\NormalTok{), }\AttributeTok{times =} \DecValTok{3}\NormalTok{)}
\end{Highlighting}
\end{Shaded}

\begin{verbatim}
[1]  2 -1  1  2 -1  1  2 -1  1
\end{verbatim}

\begin{Shaded}
\begin{Highlighting}[]
\FunctionTok{rep}\NormalTok{(}\FunctionTok{c}\NormalTok{(}\DecValTok{5}\NormalTok{, }\DecValTok{0}\NormalTok{, }\SpecialCharTok{{-}}\DecValTok{2}\NormalTok{), }\AttributeTok{times =} \FunctionTok{c}\NormalTok{(}\DecValTok{1}\NormalTok{, }\DecValTok{3}\NormalTok{, }\DecValTok{4}\NormalTok{))}
\end{Highlighting}
\end{Shaded}

\begin{verbatim}
[1]  5  0  0  0 -2 -2 -2 -2
\end{verbatim}

\end{tcolorbox}

\section{Summary functions for numeric
vectors}\label{summary-functions-for-numeric-vectors}

Consider a vector

\begin{Shaded}
\begin{Highlighting}[]
\CommentTok{\#       1   2   3   4   5   6   7}
\NormalTok{v }\OtherTok{\textless{}{-}} \FunctionTok{c}\NormalTok{(}\DecValTok{70}\NormalTok{, }\DecValTok{66}\NormalTok{, }\DecValTok{82}\NormalTok{, }\DecValTok{85}\NormalTok{, }\DecValTok{78}\NormalTok{, }\DecValTok{90}\NormalTok{, }\DecValTok{73}\NormalTok{)}
\end{Highlighting}
\end{Shaded}

\begin{itemize}
\item
  \texttt{length()} returns its length
\item
  \texttt{min()} returns its minimum value
\item
  \texttt{max()} returns its maximum value
\item
  \texttt{which.min()} returns the index of its minimum value
\item
  \texttt{which.max()} returns the index of its maximum value
\item
  \texttt{sum()} returns the sum of its elements
\item
  \texttt{prod()} returns the product of its elements
\end{itemize}

\begin{Shaded}
\begin{Highlighting}[]
\FunctionTok{length}\NormalTok{(v)}
\end{Highlighting}
\end{Shaded}

\begin{verbatim}
[1] 7
\end{verbatim}

\begin{Shaded}
\begin{Highlighting}[]
\FunctionTok{min}\NormalTok{(v)}
\end{Highlighting}
\end{Shaded}

\begin{verbatim}
[1] 66
\end{verbatim}

\begin{Shaded}
\begin{Highlighting}[]
\FunctionTok{which.min}\NormalTok{(v)}
\end{Highlighting}
\end{Shaded}

\begin{verbatim}
[1] 2
\end{verbatim}

\begin{Shaded}
\begin{Highlighting}[]
\FunctionTok{max}\NormalTok{(v)}
\end{Highlighting}
\end{Shaded}

\begin{verbatim}
[1] 90
\end{verbatim}

\begin{Shaded}
\begin{Highlighting}[]
\FunctionTok{which.max}\NormalTok{(v)}
\end{Highlighting}
\end{Shaded}

\begin{verbatim}
[1] 6
\end{verbatim}

\begin{Shaded}
\begin{Highlighting}[]
\FunctionTok{sum}\NormalTok{(v)}
\end{Highlighting}
\end{Shaded}

\begin{verbatim}
[1] 544
\end{verbatim}

\begin{Shaded}
\begin{Highlighting}[]
\FunctionTok{prod}\NormalTok{(v)}
\end{Highlighting}
\end{Shaded}

\begin{verbatim}
[1] 1.650193e+13
\end{verbatim}

💻 \textbf{Hands-On}

Consider a vector containing calorie content of light beer brands. Write
R code to answer the following questions:

\begin{itemize}
\item
  How many light beer brands are included?
\item
  What is the lowest calorie content among the light beers?
\item
  What is the highest calorie content among the light beers?
\item
  At which position does the highest calorie value occur?
\item
  What is the total calorie content across all light beer brands?
\end{itemize}

\begin{Shaded}
\begin{Highlighting}[]
\CommentTok{\# Calories per 100ml of light beer}
\NormalTok{beer\_cals }\OtherTok{\textless{}{-}} \FunctionTok{c}\NormalTok{(}\DecValTok{29}\NormalTok{, }\DecValTok{28}\NormalTok{, }\DecValTok{33}\NormalTok{, }\DecValTok{31}\NormalTok{, }\DecValTok{30}\NormalTok{, }\DecValTok{33}\NormalTok{, }\DecValTok{30}\NormalTok{, }\DecValTok{28}\NormalTok{, }\DecValTok{27}\NormalTok{, }\DecValTok{41}\NormalTok{, }\DecValTok{39}\NormalTok{, }\DecValTok{31}\NormalTok{, }\DecValTok{29}\NormalTok{, }
               \DecValTok{23}\NormalTok{, }\DecValTok{32}\NormalTok{, }\DecValTok{31}\NormalTok{, }\DecValTok{32}\NormalTok{, }\DecValTok{19}\NormalTok{, }\DecValTok{40}\NormalTok{, }\DecValTok{22}\NormalTok{, }\DecValTok{34}\NormalTok{, }\DecValTok{31}\NormalTok{, }\DecValTok{42}\NormalTok{, }\DecValTok{35}\NormalTok{, }\DecValTok{29}\NormalTok{, }\DecValTok{43}\NormalTok{)}
\end{Highlighting}
\end{Shaded}

\begin{tcolorbox}[enhanced jigsaw, colframe=quarto-callout-tip-color-frame, coltitle=black, left=2mm, rightrule=.15mm, colback=white, opacityback=0, toprule=.15mm, bottomtitle=1mm, colbacktitle=quarto-callout-tip-color!10!white, breakable, titlerule=0mm, title=\textcolor{quarto-callout-tip-color}{\faLightbulb}\hspace{0.5em}{Answer}, toptitle=1mm, arc=.35mm, bottomrule=.15mm, leftrule=.75mm, opacitybacktitle=0.6]

\begin{Shaded}
\begin{Highlighting}[]
\CommentTok{\# Number of light beer brands}
\FunctionTok{length}\NormalTok{(beer\_cals)}
\end{Highlighting}
\end{Shaded}

\begin{verbatim}
[1] 26
\end{verbatim}

\begin{Shaded}
\begin{Highlighting}[]
\CommentTok{\# Lowest calorie content}
\FunctionTok{min}\NormalTok{(beer\_cals)}
\end{Highlighting}
\end{Shaded}

\begin{verbatim}
[1] 19
\end{verbatim}

\begin{Shaded}
\begin{Highlighting}[]
\CommentTok{\# Highest calorie content}
\FunctionTok{max}\NormalTok{(beer\_cals)}
\end{Highlighting}
\end{Shaded}

\begin{verbatim}
[1] 43
\end{verbatim}

\begin{Shaded}
\begin{Highlighting}[]
\CommentTok{\# Position of the highest calorie value}
\FunctionTok{which.max}\NormalTok{(beer\_cals)}
\end{Highlighting}
\end{Shaded}

\begin{verbatim}
[1] 26
\end{verbatim}

\begin{Shaded}
\begin{Highlighting}[]
\CommentTok{\# Total calorie content}
\FunctionTok{sum}\NormalTok{(beer\_cals)}
\end{Highlighting}
\end{Shaded}

\begin{verbatim}
[1] 822
\end{verbatim}

\end{tcolorbox}

💻 \textbf{Hands-On}

Try the following R code to see what it returns.

\begin{Shaded}
\begin{Highlighting}[]
\CommentTok{\#       1   2   3   4   5   6   7}
\NormalTok{v }\OtherTok{\textless{}{-}} \FunctionTok{c}\NormalTok{(}\DecValTok{70}\NormalTok{, }\DecValTok{66}\NormalTok{, }\DecValTok{82}\NormalTok{, }\DecValTok{85}\NormalTok{, }\ConstantTok{NA}\NormalTok{, }\DecValTok{90}\NormalTok{, }\DecValTok{73}\NormalTok{)}

\FunctionTok{length}\NormalTok{(v)}

\FunctionTok{min}\NormalTok{(v)}

\FunctionTok{which.min}\NormalTok{(v)}

\FunctionTok{max}\NormalTok{(v)}

\FunctionTok{which.max}\NormalTok{(v)}

\FunctionTok{sum}\NormalTok{(v)}

\FunctionTok{prod}\NormalTok{(v)}
\end{Highlighting}
\end{Shaded}

\begin{tcolorbox}[enhanced jigsaw, colframe=quarto-callout-tip-color-frame, coltitle=black, left=2mm, rightrule=.15mm, colback=white, opacityback=0, toprule=.15mm, bottomtitle=1mm, colbacktitle=quarto-callout-tip-color!10!white, breakable, titlerule=0mm, title=\textcolor{quarto-callout-tip-color}{\faLightbulb}\hspace{0.5em}{Answer}, toptitle=1mm, arc=.35mm, bottomrule=.15mm, leftrule=.75mm, opacitybacktitle=0.6]

Note that the vector \texttt{v} contains a missing value \texttt{NA}

\begin{Shaded}
\begin{Highlighting}[]
\CommentTok{\#       1   2   3   4   5   6   7}
\NormalTok{v }\OtherTok{\textless{}{-}} \FunctionTok{c}\NormalTok{(}\DecValTok{70}\NormalTok{, }\DecValTok{66}\NormalTok{, }\DecValTok{82}\NormalTok{, }\DecValTok{85}\NormalTok{, }\ConstantTok{NA}\NormalTok{, }\DecValTok{90}\NormalTok{, }\DecValTok{73}\NormalTok{)}
\end{Highlighting}
\end{Shaded}

Therefore, some functions return \texttt{NA}. The functions
\texttt{which.min()} and \texttt{which.max()} exclude missing values
first so they return some values.

\begin{Shaded}
\begin{Highlighting}[]
\FunctionTok{length}\NormalTok{(v)}
\end{Highlighting}
\end{Shaded}

\begin{verbatim}
[1] 7
\end{verbatim}

\begin{Shaded}
\begin{Highlighting}[]
\FunctionTok{min}\NormalTok{(v)}
\end{Highlighting}
\end{Shaded}

\begin{verbatim}
[1] NA
\end{verbatim}

\begin{Shaded}
\begin{Highlighting}[]
\FunctionTok{which.min}\NormalTok{(v)}
\end{Highlighting}
\end{Shaded}

\begin{verbatim}
[1] 2
\end{verbatim}

\begin{Shaded}
\begin{Highlighting}[]
\FunctionTok{max}\NormalTok{(v)}
\end{Highlighting}
\end{Shaded}

\begin{verbatim}
[1] NA
\end{verbatim}

\begin{Shaded}
\begin{Highlighting}[]
\FunctionTok{which.max}\NormalTok{(v)}
\end{Highlighting}
\end{Shaded}

\begin{verbatim}
[1] 6
\end{verbatim}

\begin{Shaded}
\begin{Highlighting}[]
\FunctionTok{sum}\NormalTok{(v)}
\end{Highlighting}
\end{Shaded}

\begin{verbatim}
[1] NA
\end{verbatim}

\begin{Shaded}
\begin{Highlighting}[]
\FunctionTok{prod}\NormalTok{(v)}
\end{Highlighting}
\end{Shaded}

\begin{verbatim}
[1] NA
\end{verbatim}

\end{tcolorbox}

\section{Built-in vectors}\label{built-in-vectors}

\begin{itemize}
\tightlist
\item
  R includes several built-in vectors for alphabets, \(\pi\), months,
  and US states.
\end{itemize}

💻 \textbf{Hands-On}

Try the following R code and see what it returns. Feel free to get the
help documentation.

\begin{Shaded}
\begin{Highlighting}[]
\NormalTok{LETTERS}
\NormalTok{letters}

\NormalTok{month.abb}
\NormalTok{month.name}

\NormalTok{pi}

\NormalTok{state.abb}
\NormalTok{state.name}
\NormalTok{state.area}
\end{Highlighting}
\end{Shaded}

\begin{tcolorbox}[enhanced jigsaw, colframe=quarto-callout-tip-color-frame, coltitle=black, left=2mm, rightrule=.15mm, colback=white, opacityback=0, toprule=.15mm, bottomtitle=1mm, colbacktitle=quarto-callout-tip-color!10!white, breakable, titlerule=0mm, title=\textcolor{quarto-callout-tip-color}{\faLightbulb}\hspace{0.5em}{Answer}, toptitle=1mm, arc=.35mm, bottomrule=.15mm, leftrule=.75mm, opacitybacktitle=0.6]

\begin{Shaded}
\begin{Highlighting}[]
\NormalTok{LETTERS}
\end{Highlighting}
\end{Shaded}

\begin{verbatim}
 [1] "A" "B" "C" "D" "E" "F" "G" "H" "I" "J" "K" "L" "M" "N" "O" "P" "Q" "R" "S"
[20] "T" "U" "V" "W" "X" "Y" "Z"
\end{verbatim}

\begin{Shaded}
\begin{Highlighting}[]
\NormalTok{letters}
\end{Highlighting}
\end{Shaded}

\begin{verbatim}
 [1] "a" "b" "c" "d" "e" "f" "g" "h" "i" "j" "k" "l" "m" "n" "o" "p" "q" "r" "s"
[20] "t" "u" "v" "w" "x" "y" "z"
\end{verbatim}

\begin{Shaded}
\begin{Highlighting}[]
\NormalTok{month.abb}
\end{Highlighting}
\end{Shaded}

\begin{verbatim}
 [1] "Jan" "Feb" "Mar" "Apr" "May" "Jun" "Jul" "Aug" "Sep" "Oct" "Nov" "Dec"
\end{verbatim}

\begin{Shaded}
\begin{Highlighting}[]
\NormalTok{month.name}
\end{Highlighting}
\end{Shaded}

\begin{verbatim}
 [1] "January"   "February"  "March"     "April"     "May"       "June"     
 [7] "July"      "August"    "September" "October"   "November"  "December" 
\end{verbatim}

\begin{Shaded}
\begin{Highlighting}[]
\NormalTok{pi}
\end{Highlighting}
\end{Shaded}

\begin{verbatim}
[1] 3.141593
\end{verbatim}

\begin{Shaded}
\begin{Highlighting}[]
\NormalTok{state.abb}
\end{Highlighting}
\end{Shaded}

\begin{verbatim}
 [1] "AL" "AK" "AZ" "AR" "CA" "CO" "CT" "DE" "FL" "GA" "HI" "ID" "IL" "IN" "IA"
[16] "KS" "KY" "LA" "ME" "MD" "MA" "MI" "MN" "MS" "MO" "MT" "NE" "NV" "NH" "NJ"
[31] "NM" "NY" "NC" "ND" "OH" "OK" "OR" "PA" "RI" "SC" "SD" "TN" "TX" "UT" "VT"
[46] "VA" "WA" "WV" "WI" "WY"
\end{verbatim}

\begin{Shaded}
\begin{Highlighting}[]
\NormalTok{state.name}
\end{Highlighting}
\end{Shaded}

\begin{verbatim}
 [1] "Alabama"        "Alaska"         "Arizona"        "Arkansas"      
 [5] "California"     "Colorado"       "Connecticut"    "Delaware"      
 [9] "Florida"        "Georgia"        "Hawaii"         "Idaho"         
[13] "Illinois"       "Indiana"        "Iowa"           "Kansas"        
[17] "Kentucky"       "Louisiana"      "Maine"          "Maryland"      
[21] "Massachusetts"  "Michigan"       "Minnesota"      "Mississippi"   
[25] "Missouri"       "Montana"        "Nebraska"       "Nevada"        
[29] "New Hampshire"  "New Jersey"     "New Mexico"     "New York"      
[33] "North Carolina" "North Dakota"   "Ohio"           "Oklahoma"      
[37] "Oregon"         "Pennsylvania"   "Rhode Island"   "South Carolina"
[41] "South Dakota"   "Tennessee"      "Texas"          "Utah"          
[45] "Vermont"        "Virginia"       "Washington"     "West Virginia" 
[49] "Wisconsin"      "Wyoming"       
\end{verbatim}

\begin{Shaded}
\begin{Highlighting}[]
\NormalTok{state.area}
\end{Highlighting}
\end{Shaded}

\begin{verbatim}
 [1]  51609 589757 113909  53104 158693 104247   5009   2057  58560  58876
[11]   6450  83557  56400  36291  56290  82264  40395  48523  33215  10577
[21]   8257  58216  84068  47716  69686 147138  77227 110540   9304   7836
[31] 121666  49576  52586  70665  41222  69919  96981  45333   1214  31055
[41]  77047  42244 267339  84916   9609  40815  68192  24181  56154  97914
\end{verbatim}

\end{tcolorbox}

\chapter{2B: More on Vector}\label{b-more-on-vector}

\section{Readings}\label{readings-1}

From \textbf{R Coding Basics: An Introduction to the Basics of Coding in
R} by Dr.~Gaston Sanchez:

\begin{itemize}
\item
  \href{https://www.gastonsanchez.com/R-coding-basics/1-01-vectors-intro.html}{First
  Contact with Vectors}
\item
  \href{https://www.gastonsanchez.com/R-coding-basics/1-02-vectors-properties.html}{Properties
  of Vectors}
\item
  \href{https://www.gastonsanchez.com/R-coding-basics/1-03-vectors-creation.html}{Creating
  Vectors}
\item
  \href{https://www.gastonsanchez.com/R-coding-basics/1-04-vectors-concepts.html}{More
  about Vectors}
\end{itemize}

\section{Topics}\label{topics-2}

\begin{itemize}
\item
  Implicit and explicit coercion
\item
  Vectorization and recycling
\item
  Working with logical vectors
\item
  Subsetting: numeric and logical indexing
\item
  Useful functions for vectors
\end{itemize}

\section{Implicit coercion}\label{implicit-coercion}

\begin{itemize}
\item
  Broadly speaking, \textbf{implicit coercion} is how R decides what
  type the output should be when combining or performing operations on
  different atomic types.
\item
  Implicit coercion is most commonly found in calculations with logical
  values and in vector creation using the combine \texttt{c()} function.
\end{itemize}

\subsection{Calculations with logical
values}\label{calculations-with-logical-values}

\begin{itemize}
\tightlist
\item
  Logical values are automatically coerced to numeric when used in
  calculations.
\end{itemize}

\begin{Shaded}
\begin{Highlighting}[]
\CommentTok{\# TRUE becomes 1 so 1 + 5 = 6}
\ConstantTok{TRUE} \SpecialCharTok{+} \DecValTok{5}     
\end{Highlighting}
\end{Shaded}

\begin{verbatim}
[1] 6
\end{verbatim}

\begin{Shaded}
\begin{Highlighting}[]
\CommentTok{\# FALSE becomes 0 so 0 + 5 = 5}
\ConstantTok{FALSE} \SpecialCharTok{+} \DecValTok{5}    
\end{Highlighting}
\end{Shaded}

\begin{verbatim}
[1] 5
\end{verbatim}

\begin{Shaded}
\begin{Highlighting}[]
\CommentTok{\# 1 + 0 + 1 = 2}
\NormalTok{v }\OtherTok{\textless{}{-}} \FunctionTok{c}\NormalTok{(}\ConstantTok{TRUE}\NormalTok{, }\ConstantTok{FALSE}\NormalTok{, }\ConstantTok{TRUE}\NormalTok{)}
\FunctionTok{sum}\NormalTok{(v)    }
\end{Highlighting}
\end{Shaded}

\begin{verbatim}
[1] 2
\end{verbatim}

💻 \textbf{Hands-On}

Guess the output in the following R code. Try the code to confirm your
guess.

\begin{Shaded}
\begin{Highlighting}[]
\ConstantTok{TRUE} \SpecialCharTok{*} \DecValTok{2}

\ConstantTok{FALSE} \SpecialCharTok{*} \DecValTok{2}

\NormalTok{v1 }\OtherTok{\textless{}{-}} \FunctionTok{c}\NormalTok{(}\ConstantTok{TRUE}\NormalTok{, }\ConstantTok{TRUE}\NormalTok{, }\ConstantTok{FALSE}\NormalTok{, }\ConstantTok{FALSE}\NormalTok{, }\ConstantTok{TRUE}\NormalTok{, }\ConstantTok{FALSE}\NormalTok{, }\ConstantTok{FALSE}\NormalTok{)}
\FunctionTok{sum}\NormalTok{(v1)}

\NormalTok{v2 }\OtherTok{\textless{}{-}} \FunctionTok{c}\NormalTok{(F, T, F, F, T, F, T)}
\FunctionTok{sum}\NormalTok{(v2)}
\end{Highlighting}
\end{Shaded}

\begin{tcolorbox}[enhanced jigsaw, colframe=quarto-callout-tip-color-frame, coltitle=black, left=2mm, rightrule=.15mm, colback=white, opacityback=0, toprule=.15mm, bottomtitle=1mm, colbacktitle=quarto-callout-tip-color!10!white, breakable, titlerule=0mm, title=\textcolor{quarto-callout-tip-color}{\faLightbulb}\hspace{0.5em}{Answer}, toptitle=1mm, arc=.35mm, bottomrule=.15mm, leftrule=.75mm, opacitybacktitle=0.6]

\begin{Shaded}
\begin{Highlighting}[]
\CommentTok{\# TRUE becomes 1 so 1 * 2 = 2}
\ConstantTok{TRUE} \SpecialCharTok{*} \DecValTok{2}
\end{Highlighting}
\end{Shaded}

\begin{verbatim}
[1] 2
\end{verbatim}

\begin{Shaded}
\begin{Highlighting}[]
\CommentTok{\# FALSE becomes 0 so 0 * 2 = 0}
\ConstantTok{FALSE} \SpecialCharTok{*} \DecValTok{2}
\end{Highlighting}
\end{Shaded}

\begin{verbatim}
[1] 0
\end{verbatim}

\begin{Shaded}
\begin{Highlighting}[]
\CommentTok{\# 1 + 1 + 0 + 0 + 1 + 0 + 0 = 3}
\NormalTok{v1 }\OtherTok{\textless{}{-}} \FunctionTok{c}\NormalTok{(}\ConstantTok{TRUE}\NormalTok{, }\ConstantTok{TRUE}\NormalTok{, }\ConstantTok{FALSE}\NormalTok{, }\ConstantTok{FALSE}\NormalTok{, }\ConstantTok{TRUE}\NormalTok{, }\ConstantTok{FALSE}\NormalTok{, }\ConstantTok{FALSE}\NormalTok{)}
\FunctionTok{sum}\NormalTok{(v1)}
\end{Highlighting}
\end{Shaded}

\begin{verbatim}
[1] 3
\end{verbatim}

\begin{Shaded}
\begin{Highlighting}[]
\CommentTok{\# 0 + 1 + 0 + 0 + 1 + 0 + 1 = 3}
\NormalTok{v2 }\OtherTok{\textless{}{-}} \FunctionTok{c}\NormalTok{(F, T, F, F, T, F, T)}
\FunctionTok{sum}\NormalTok{(v2)}
\end{Highlighting}
\end{Shaded}

\begin{verbatim}
[1] 3
\end{verbatim}

\end{tcolorbox}

\subsection{\texorpdfstring{Combining values with
\texttt{c()}}{Combining values with c()}}\label{combining-values-with-c}

\begin{itemize}
\item
  Implicit coercion also occurs when we combine different atomic types
  into a single vector.
\item
  In this case, all elements are converted to the most complex type
  (character).
\end{itemize}

\[
\text{Character} \, > \, \text{Double} \, > \, \text{Integer} \, > \, \text{Logical}
\]

\begin{Shaded}
\begin{Highlighting}[]
\CommentTok{\# integer \textgreater{} logical}
\FunctionTok{c}\NormalTok{(}\DecValTok{4}\NormalTok{L, }\ConstantTok{TRUE}\NormalTok{, }\ConstantTok{TRUE}\NormalTok{, }\ConstantTok{FALSE}\NormalTok{)                   }
\end{Highlighting}
\end{Shaded}

\begin{verbatim}
[1] 4 1 1 0
\end{verbatim}

\begin{Shaded}
\begin{Highlighting}[]
\CommentTok{\# double \textgreater{} integer \textgreater{} logical}
\FunctionTok{c}\NormalTok{(}\FloatTok{7.3}\NormalTok{, }\DecValTok{4}\NormalTok{L, }\ConstantTok{TRUE}\NormalTok{, }\ConstantTok{TRUE}\NormalTok{, }\ConstantTok{FALSE}\NormalTok{)              }
\end{Highlighting}
\end{Shaded}

\begin{verbatim}
[1] 7.3 4.0 1.0 1.0 0.0
\end{verbatim}

\begin{Shaded}
\begin{Highlighting}[]
\CommentTok{\# character \textgreater{} double \textgreater{} integer \textgreater{} logical}
\FunctionTok{c}\NormalTok{(}\StringTok{"eagles"}\NormalTok{, }\FloatTok{7.3}\NormalTok{, }\DecValTok{4}\NormalTok{L, }\ConstantTok{TRUE}\NormalTok{, }\ConstantTok{TRUE}\NormalTok{, }\ConstantTok{FALSE}\NormalTok{)    }
\end{Highlighting}
\end{Shaded}

\begin{verbatim}
[1] "eagles" "7.3"    "4"      "TRUE"   "TRUE"   "FALSE" 
\end{verbatim}

💻 \textbf{Hands-On}

Guess the output in the following R code. Try the code to confirm your
guess.

\begin{Shaded}
\begin{Highlighting}[]
\NormalTok{v1 }\OtherTok{\textless{}{-}} \FunctionTok{c}\NormalTok{(}\ConstantTok{TRUE}\NormalTok{, }\FloatTok{3.5}\NormalTok{, }\ConstantTok{FALSE}\NormalTok{, }\DecValTok{1}\NormalTok{L)}
\FunctionTok{sum}\NormalTok{(v1)}

\NormalTok{v2 }\OtherTok{\textless{}{-}} \FunctionTok{c}\NormalTok{(}\ConstantTok{FALSE}\NormalTok{, }\ConstantTok{TRUE}\NormalTok{, }\DecValTok{2}\NormalTok{, }\DecValTok{3}\NormalTok{, }\StringTok{"4"}\NormalTok{, }\DecValTok{5}\NormalTok{L)}
\FunctionTok{sum}\NormalTok{(v2)}
\end{Highlighting}
\end{Shaded}

\begin{tcolorbox}[enhanced jigsaw, colframe=quarto-callout-tip-color-frame, coltitle=black, left=2mm, rightrule=.15mm, colback=white, opacityback=0, toprule=.15mm, bottomtitle=1mm, colbacktitle=quarto-callout-tip-color!10!white, breakable, titlerule=0mm, title=\textcolor{quarto-callout-tip-color}{\faLightbulb}\hspace{0.5em}{Answer}, toptitle=1mm, arc=.35mm, bottomrule=.15mm, leftrule=.75mm, opacitybacktitle=0.6]

\begin{Shaded}
\begin{Highlighting}[]
\CommentTok{\# Coercion gives c(1.0, 3.5, 0.0, 1.0)}
\NormalTok{v1 }\OtherTok{\textless{}{-}} \FunctionTok{c}\NormalTok{(}\ConstantTok{TRUE}\NormalTok{, }\FloatTok{3.5}\NormalTok{, }\ConstantTok{FALSE}\NormalTok{, }\DecValTok{1}\NormalTok{L)}
\FunctionTok{sum}\NormalTok{(v1)}
\end{Highlighting}
\end{Shaded}

\begin{verbatim}
[1] 5.5
\end{verbatim}

\begin{Shaded}
\begin{Highlighting}[]
\CommentTok{\# Coercion gives c("FALSE", "TRUE", "2", "3", "4", "5") so sum() gives an error}
\NormalTok{v2 }\OtherTok{\textless{}{-}} \FunctionTok{c}\NormalTok{(}\ConstantTok{FALSE}\NormalTok{, }\ConstantTok{TRUE}\NormalTok{, }\DecValTok{2}\NormalTok{, }\DecValTok{3}\NormalTok{, }\StringTok{"4"}\NormalTok{, }\DecValTok{5}\NormalTok{L)}
\FunctionTok{sum}\NormalTok{(v2)}
\end{Highlighting}
\end{Shaded}

\begin{verbatim}
Error in sum(v2): invalid 'type' (character) of argument
\end{verbatim}

\end{tcolorbox}

\section{Explicit coercion}\label{explicit-coercion}

\begin{itemize}
\tightlist
\item
  \textbf{Explicit coercion} directly forces one atomic type to become
  another.
\end{itemize}

\begin{Shaded}
\begin{Highlighting}[]
\FunctionTok{as.logical}\NormalTok{()      }\CommentTok{\# to logical type}

\FunctionTok{as.integer}\NormalTok{()      }\CommentTok{\# to integer type}
\FunctionTok{as.double}\NormalTok{()       }\CommentTok{\# to double type}
\FunctionTok{as.numeric}\NormalTok{()      }\CommentTok{\# to numeric type}

\FunctionTok{as.character}\NormalTok{()    }\CommentTok{\# to character type}
\end{Highlighting}
\end{Shaded}

\section{Vectorization}\label{vectorization}

\begin{itemize}
\item
  In R, \textbf{vectorization} means when we apply a function or
  operation on a vector, it will be done element by element
  automatically.
\item
  This means we do not need to loop through the vector.
\end{itemize}

\begin{Shaded}
\begin{Highlighting}[]
\CommentTok{\#       x:      4        8         25         49}
\CommentTok{\# sqrt(x): sqrt(4)  sqrt(8)   sqrt(25)   sqrt(49)}
\CommentTok{\#  output:      2        3          5          7}

\NormalTok{x }\OtherTok{\textless{}{-}} \FunctionTok{c}\NormalTok{(}\DecValTok{4}\NormalTok{, }\DecValTok{9}\NormalTok{, }\DecValTok{25}\NormalTok{, }\DecValTok{49}\NormalTok{)}
\FunctionTok{sqrt}\NormalTok{(x)}
\end{Highlighting}
\end{Shaded}

\begin{verbatim}
[1] 2 3 5 7
\end{verbatim}

\begin{itemize}
\tightlist
\item
  Vectorization also implies that an operation on two vectors will be
  done in an element-wise manner.
\end{itemize}

\begin{Shaded}
\begin{Highlighting}[]
\CommentTok{\#      x:  1  2  3  4}
\CommentTok{\#          +  +  +  +}
\CommentTok{\#      y:  3  5  6  2}
\CommentTok{\# output:  4  7  9  6}

\NormalTok{x }\OtherTok{\textless{}{-}} \FunctionTok{c}\NormalTok{(}\DecValTok{1}\NormalTok{, }\DecValTok{2}\NormalTok{, }\DecValTok{3}\NormalTok{, }\DecValTok{4}\NormalTok{)}
\NormalTok{y }\OtherTok{\textless{}{-}} \FunctionTok{c}\NormalTok{(}\DecValTok{3}\NormalTok{, }\DecValTok{5}\NormalTok{, }\DecValTok{6}\NormalTok{, }\DecValTok{2}\NormalTok{)}
\NormalTok{x }\SpecialCharTok{+}\NormalTok{ y}
\end{Highlighting}
\end{Shaded}

\begin{verbatim}
[1] 4 7 9 6
\end{verbatim}

💻 \textbf{Hands-On}

Verify the properties of vectorization with the following code

\begin{Shaded}
\begin{Highlighting}[]
\NormalTok{x }\OtherTok{\textless{}{-}} \FunctionTok{c}\NormalTok{(}\DecValTok{1}\NormalTok{, }\DecValTok{2}\NormalTok{, }\DecValTok{3}\NormalTok{, }\DecValTok{4}\NormalTok{)}
\NormalTok{y }\OtherTok{\textless{}{-}} \FunctionTok{c}\NormalTok{(}\DecValTok{3}\NormalTok{, }\DecValTok{5}\NormalTok{, }\DecValTok{6}\NormalTok{, }\DecValTok{2}\NormalTok{)}

\NormalTok{x }\SpecialCharTok{{-}}\NormalTok{ y}

\NormalTok{x }\SpecialCharTok{*}\NormalTok{ y}

\NormalTok{y }\SpecialCharTok{/}\NormalTok{ x}

\FunctionTok{log}\NormalTok{(y)}

\FunctionTok{sin}\NormalTok{(x)}
\end{Highlighting}
\end{Shaded}

\begin{tcolorbox}[enhanced jigsaw, colframe=quarto-callout-tip-color-frame, coltitle=black, left=2mm, rightrule=.15mm, colback=white, opacityback=0, toprule=.15mm, bottomtitle=1mm, colbacktitle=quarto-callout-tip-color!10!white, breakable, titlerule=0mm, title=\textcolor{quarto-callout-tip-color}{\faLightbulb}\hspace{0.5em}{Answer}, toptitle=1mm, arc=.35mm, bottomrule=.15mm, leftrule=.75mm, opacitybacktitle=0.6]

Due to vectorization, R will perform operations on corresponding
elements.

\begin{Shaded}
\begin{Highlighting}[]
\NormalTok{x }\OtherTok{\textless{}{-}} \FunctionTok{c}\NormalTok{(}\DecValTok{1}\NormalTok{, }\DecValTok{2}\NormalTok{, }\DecValTok{3}\NormalTok{, }\DecValTok{4}\NormalTok{)}
\NormalTok{y }\OtherTok{\textless{}{-}} \FunctionTok{c}\NormalTok{(}\DecValTok{3}\NormalTok{, }\DecValTok{5}\NormalTok{, }\DecValTok{6}\NormalTok{, }\DecValTok{2}\NormalTok{)}
\end{Highlighting}
\end{Shaded}

See the following

\begin{Shaded}
\begin{Highlighting}[]
\NormalTok{x }\SpecialCharTok{{-}}\NormalTok{ y}
\end{Highlighting}
\end{Shaded}

\begin{verbatim}
[1] -2 -3 -3  2
\end{verbatim}

\begin{Shaded}
\begin{Highlighting}[]
\NormalTok{x }\SpecialCharTok{*}\NormalTok{ y}
\end{Highlighting}
\end{Shaded}

\begin{verbatim}
[1]  3 10 18  8
\end{verbatim}

\begin{Shaded}
\begin{Highlighting}[]
\NormalTok{y }\SpecialCharTok{/}\NormalTok{ x}
\end{Highlighting}
\end{Shaded}

\begin{verbatim}
[1] 3.0 2.5 2.0 0.5
\end{verbatim}

\begin{Shaded}
\begin{Highlighting}[]
\FunctionTok{log}\NormalTok{(y)}
\end{Highlighting}
\end{Shaded}

\begin{verbatim}
[1] 1.0986123 1.6094379 1.7917595 0.6931472
\end{verbatim}

\begin{Shaded}
\begin{Highlighting}[]
\FunctionTok{sin}\NormalTok{(x)}
\end{Highlighting}
\end{Shaded}

\begin{verbatim}
[1]  0.8414710  0.9092974  0.1411200 -0.7568025
\end{verbatim}

\end{tcolorbox}

\section{Recycling}\label{recycling}

\begin{itemize}
\tightlist
\item
  When performing operations on two vectors are different lengths, R
  will \textbf{recycle} the shorter vector to match the length of the
  longer one.
\end{itemize}

\begin{Shaded}
\begin{Highlighting}[]
\CommentTok{\#      x: 1  2  3  4}
\CommentTok{\#         *  *  *  *}
\CommentTok{\#         2  2  2  2}
\CommentTok{\# output: 2  4  6  8}

\NormalTok{x }\OtherTok{\textless{}{-}} \FunctionTok{c}\NormalTok{(}\DecValTok{1}\NormalTok{, }\DecValTok{2}\NormalTok{, }\DecValTok{3}\NormalTok{, }\DecValTok{4}\NormalTok{)}
\NormalTok{x }\SpecialCharTok{*} \DecValTok{2}
\end{Highlighting}
\end{Shaded}

\begin{verbatim}
[1] 2 4 6 8
\end{verbatim}

\begin{itemize}
\tightlist
\item
  Recycling can produce confusing results, so it is recommended that two
  vectors either have the same length or that one of them has length 1.
\end{itemize}

💻 \textbf{Hands-On}

Try the following R code and see what it returns. What do you think this
sneaky code does?

\begin{Shaded}
\begin{Highlighting}[]
\CommentTok{\# Please avoid!}

\NormalTok{x }\OtherTok{\textless{}{-}} \FunctionTok{c}\NormalTok{(}\DecValTok{5}\NormalTok{, }\DecValTok{3}\NormalTok{, }\DecValTok{7}\NormalTok{, }\DecValTok{4}\NormalTok{, }\DecValTok{6}\NormalTok{, }\DecValTok{8}\NormalTok{, }\DecValTok{3}\NormalTok{)}
\NormalTok{y }\OtherTok{\textless{}{-}} \FunctionTok{c}\NormalTok{(}\DecValTok{1}\NormalTok{, }\DecValTok{2}\NormalTok{)}
\NormalTok{x }\SpecialCharTok{+}\NormalTok{ y}
\end{Highlighting}
\end{Shaded}

\begin{tcolorbox}[enhanced jigsaw, colframe=quarto-callout-tip-color-frame, coltitle=black, left=2mm, rightrule=.15mm, colback=white, opacityback=0, toprule=.15mm, bottomtitle=1mm, colbacktitle=quarto-callout-tip-color!10!white, breakable, titlerule=0mm, title=\textcolor{quarto-callout-tip-color}{\faLightbulb}\hspace{0.5em}{Answer}, toptitle=1mm, arc=.35mm, bottomrule=.15mm, leftrule=.75mm, opacitybacktitle=0.6]

We will see the followling

\begin{Shaded}
\begin{Highlighting}[]
\NormalTok{x }\OtherTok{\textless{}{-}} \FunctionTok{c}\NormalTok{(}\DecValTok{5}\NormalTok{, }\DecValTok{3}\NormalTok{, }\DecValTok{7}\NormalTok{, }\DecValTok{4}\NormalTok{, }\DecValTok{6}\NormalTok{, }\DecValTok{8}\NormalTok{, }\DecValTok{3}\NormalTok{)}
\NormalTok{y }\OtherTok{\textless{}{-}} \FunctionTok{c}\NormalTok{(}\DecValTok{1}\NormalTok{, }\DecValTok{2}\NormalTok{)}
\NormalTok{x }\SpecialCharTok{+}\NormalTok{ y}
\end{Highlighting}
\end{Shaded}

\begin{verbatim}
[1]  6  5  8  6  7 10  4
\end{verbatim}

The shorter vector \texttt{y} recycles its elements to match the length
of the longer vector \texttt{x}. The above is equivalent to

\begin{Shaded}
\begin{Highlighting}[]
\NormalTok{x }\OtherTok{\textless{}{-}} \FunctionTok{c}\NormalTok{(}\DecValTok{5}\NormalTok{, }\DecValTok{3}\NormalTok{, }\DecValTok{7}\NormalTok{, }\DecValTok{4}\NormalTok{, }\DecValTok{6}\NormalTok{, }\DecValTok{8}\NormalTok{, }\DecValTok{3}\NormalTok{)}
\NormalTok{y }\OtherTok{\textless{}{-}} \FunctionTok{c}\NormalTok{(}\DecValTok{1}\NormalTok{, }\DecValTok{2}\NormalTok{, }\DecValTok{1}\NormalTok{, }\DecValTok{2}\NormalTok{, }\DecValTok{1}\NormalTok{, }\DecValTok{2}\NormalTok{, }\DecValTok{1}\NormalTok{)}
\NormalTok{x }\SpecialCharTok{+}\NormalTok{ y}
\end{Highlighting}
\end{Shaded}

\begin{verbatim}
[1]  6  5  8  6  7 10  4
\end{verbatim}

\end{tcolorbox}

\section{Working with logical
vectors}\label{working-with-logical-vectors}

\begin{itemize}
\tightlist
\item
  \textbf{Logical vectors} are frequently used in data analysis to
  account for different conditions.
\end{itemize}

\begin{Shaded}
\begin{Highlighting}[]
\CommentTok{\# logical vector}
\FunctionTok{c}\NormalTok{(}\ConstantTok{TRUE}\NormalTok{, }\ConstantTok{FALSE}\NormalTok{, }\ConstantTok{FALSE}\NormalTok{, }\ConstantTok{TRUE}\NormalTok{, }\ConstantTok{TRUE}\NormalTok{)}
\FunctionTok{c}\NormalTok{(T, F, F, T, T)}
\end{Highlighting}
\end{Shaded}

\subsection{Comparison operators}\label{comparison-operators}

\begin{itemize}
\tightlist
\item
  A logical vector can be created using \textbf{comparison operators}.
\end{itemize}

\begin{longtable}[]{@{}ll@{}}
\toprule\noalign{}
\endhead
\bottomrule\noalign{}
\endlastfoot
\textbf{Logical Operator} & \textbf{Description} \\
\texttt{\textless{}} & Less than \\
\texttt{\textless{}=} & Less than or equal to \\
\texttt{\textgreater{}} & Greater than \\
\texttt{\textgreater{}=} & Greater than or equal to \\
\texttt{==} & Exactly equal to \\
\texttt{!=} & Not equal to \\
\end{longtable}

\begin{itemize}
\tightlist
\item
  Consider the following vector
\end{itemize}

\begin{Shaded}
\begin{Highlighting}[]
\NormalTok{x }\OtherTok{\textless{}{-}} \FunctionTok{c}\NormalTok{(}\SpecialCharTok{{-}}\DecValTok{2}\NormalTok{, }\SpecialCharTok{{-}}\DecValTok{1}\NormalTok{, }\DecValTok{0}\NormalTok{, }\DecValTok{1}\NormalTok{, }\DecValTok{2}\NormalTok{, }\DecValTok{3}\NormalTok{)}
\end{Highlighting}
\end{Shaded}

\begin{itemize}
\tightlist
\item
  We have
\end{itemize}

\begin{Shaded}
\begin{Highlighting}[]
\CommentTok{\# less than}
\NormalTok{x }\SpecialCharTok{\textless{}} \DecValTok{0}     
\end{Highlighting}
\end{Shaded}

\begin{verbatim}
[1]  TRUE  TRUE FALSE FALSE FALSE FALSE
\end{verbatim}

\begin{Shaded}
\begin{Highlighting}[]
\CommentTok{\# less than or equal to}
\NormalTok{x }\SpecialCharTok{\textless{}=} \DecValTok{0}    
\end{Highlighting}
\end{Shaded}

\begin{verbatim}
[1]  TRUE  TRUE  TRUE FALSE FALSE FALSE
\end{verbatim}

\begin{Shaded}
\begin{Highlighting}[]
\CommentTok{\# greater than}
\NormalTok{x }\SpecialCharTok{\textgreater{}} \DecValTok{0}     
\end{Highlighting}
\end{Shaded}

\begin{verbatim}
[1] FALSE FALSE FALSE  TRUE  TRUE  TRUE
\end{verbatim}

\begin{Shaded}
\begin{Highlighting}[]
\CommentTok{\# greater than or equal to}
\NormalTok{x }\SpecialCharTok{\textgreater{}=} \DecValTok{0}    
\end{Highlighting}
\end{Shaded}

\begin{verbatim}
[1] FALSE FALSE  TRUE  TRUE  TRUE  TRUE
\end{verbatim}

\begin{Shaded}
\begin{Highlighting}[]
\CommentTok{\# exactly equal to}
\NormalTok{x }\SpecialCharTok{==} \DecValTok{0}    
\end{Highlighting}
\end{Shaded}

\begin{verbatim}
[1] FALSE FALSE  TRUE FALSE FALSE FALSE
\end{verbatim}

\begin{Shaded}
\begin{Highlighting}[]
\CommentTok{\# not equal to}
\NormalTok{x }\SpecialCharTok{!=} \DecValTok{0}    
\end{Highlighting}
\end{Shaded}

\begin{verbatim}
[1]  TRUE  TRUE FALSE  TRUE  TRUE  TRUE
\end{verbatim}

💻 \textbf{Hands-On}

Consider daily high temperatures in celsius over two weeks. Write R code
to answer the following questions:

\begin{itemize}
\item
  Which days had temperatures \textbf{at least 30°C}?
\item
  Which days had temperatures \textbf{exactly equal to 19°C}?
\item
  Which days had temperatures \textbf{below 20°C}?
\item
  Which days had temperatures \textbf{greater than 25°C}?
\item
  Which days had temperatures \textbf{at most 22°C}?
\end{itemize}

\begin{Shaded}
\begin{Highlighting}[]
\CommentTok{\# Daily high temperatures over two weeks}
\NormalTok{temps }\OtherTok{\textless{}{-}} \FunctionTok{c}\NormalTok{(}
  \AttributeTok{Mon02 =} \DecValTok{18}\NormalTok{, }\AttributeTok{Tue03 =} \DecValTok{21}\NormalTok{, }\AttributeTok{Wed04 =} \DecValTok{19}\NormalTok{, }\AttributeTok{Thu05 =} \DecValTok{25}\NormalTok{, }\AttributeTok{Fri06 =} \DecValTok{27}\NormalTok{, }\AttributeTok{Sat07 =} \DecValTok{30}\NormalTok{, }\AttributeTok{Sun08 =} \DecValTok{22}\NormalTok{, }
  \AttributeTok{Mon09 =} \DecValTok{20}\NormalTok{, }\AttributeTok{Tue10 =} \DecValTok{16}\NormalTok{, }\AttributeTok{Wed11 =} \DecValTok{24}\NormalTok{, }\AttributeTok{Thu12 =} \DecValTok{28}\NormalTok{, }\AttributeTok{Fri13 =} \DecValTok{31}\NormalTok{, }\AttributeTok{Sat14 =} \DecValTok{29}\NormalTok{, }\AttributeTok{Sun15 =} \DecValTok{17}
\NormalTok{)}
\end{Highlighting}
\end{Shaded}

\begin{tcolorbox}[enhanced jigsaw, colframe=quarto-callout-tip-color-frame, coltitle=black, left=2mm, rightrule=.15mm, colback=white, opacityback=0, toprule=.15mm, bottomtitle=1mm, colbacktitle=quarto-callout-tip-color!10!white, breakable, titlerule=0mm, title=\textcolor{quarto-callout-tip-color}{\faLightbulb}\hspace{0.5em}{Answer}, toptitle=1mm, arc=.35mm, bottomrule=.15mm, leftrule=.75mm, opacitybacktitle=0.6]

Here, we will create logical vectors and use \texttt{TRUE} values to
tell which days satisfy the conditions. Note that the vector
\texttt{temps} has names to make it easier to tell.

\begin{Shaded}
\begin{Highlighting}[]
\NormalTok{temps }\SpecialCharTok{\textgreater{}=} \DecValTok{30}
\end{Highlighting}
\end{Shaded}

\begin{verbatim}
Mon02 Tue03 Wed04 Thu05 Fri06 Sat07 Sun08 Mon09 Tue10 Wed11 Thu12 Fri13 Sat14 
FALSE FALSE FALSE FALSE FALSE  TRUE FALSE FALSE FALSE FALSE FALSE  TRUE FALSE 
Sun15 
FALSE 
\end{verbatim}

\begin{Shaded}
\begin{Highlighting}[]
\NormalTok{temps }\SpecialCharTok{==} \DecValTok{19}
\end{Highlighting}
\end{Shaded}

\begin{verbatim}
Mon02 Tue03 Wed04 Thu05 Fri06 Sat07 Sun08 Mon09 Tue10 Wed11 Thu12 Fri13 Sat14 
FALSE FALSE  TRUE FALSE FALSE FALSE FALSE FALSE FALSE FALSE FALSE FALSE FALSE 
Sun15 
FALSE 
\end{verbatim}

\begin{Shaded}
\begin{Highlighting}[]
\NormalTok{temps }\SpecialCharTok{\textless{}} \DecValTok{20}
\end{Highlighting}
\end{Shaded}

\begin{verbatim}
Mon02 Tue03 Wed04 Thu05 Fri06 Sat07 Sun08 Mon09 Tue10 Wed11 Thu12 Fri13 Sat14 
 TRUE FALSE  TRUE FALSE FALSE FALSE FALSE FALSE  TRUE FALSE FALSE FALSE FALSE 
Sun15 
 TRUE 
\end{verbatim}

\begin{Shaded}
\begin{Highlighting}[]
\NormalTok{temps }\SpecialCharTok{\textgreater{}} \DecValTok{25}
\end{Highlighting}
\end{Shaded}

\begin{verbatim}
Mon02 Tue03 Wed04 Thu05 Fri06 Sat07 Sun08 Mon09 Tue10 Wed11 Thu12 Fri13 Sat14 
FALSE FALSE FALSE FALSE  TRUE  TRUE FALSE FALSE FALSE FALSE  TRUE  TRUE  TRUE 
Sun15 
FALSE 
\end{verbatim}

\begin{Shaded}
\begin{Highlighting}[]
\NormalTok{temps }\SpecialCharTok{\textless{}=} \DecValTok{22}
\end{Highlighting}
\end{Shaded}

\begin{verbatim}
Mon02 Tue03 Wed04 Thu05 Fri06 Sat07 Sun08 Mon09 Tue10 Wed11 Thu12 Fri13 Sat14 
 TRUE  TRUE  TRUE FALSE FALSE FALSE  TRUE  TRUE  TRUE FALSE FALSE FALSE FALSE 
Sun15 
 TRUE 
\end{verbatim}

\end{tcolorbox}

\subsection{\texorpdfstring{The \texttt{sum()}
function}{The sum() function}}\label{the-sum-function}

\begin{itemize}
\item
  Applying the \texttt{sum()} function to a logical vector gives the
  number of \texttt{TRUE} values.
\item
  Recall that implicit coercion turns logical values to \(1\)'s and
  \(0\)'s.
\end{itemize}

\begin{Shaded}
\begin{Highlighting}[]
\CommentTok{\# 1 + 0 + 0 + 1 + 1 }
\NormalTok{v }\OtherTok{\textless{}{-}} \FunctionTok{c}\NormalTok{(}\ConstantTok{TRUE}\NormalTok{, }\ConstantTok{FALSE}\NormalTok{, }\ConstantTok{FALSE}\NormalTok{, }\ConstantTok{TRUE}\NormalTok{, }\ConstantTok{TRUE}\NormalTok{)}
\FunctionTok{sum}\NormalTok{(v)}
\end{Highlighting}
\end{Shaded}

\begin{verbatim}
[1] 3
\end{verbatim}

💻 \textbf{Hands-On}

Consider the following vector

\begin{Shaded}
\begin{Highlighting}[]
\NormalTok{x }\OtherTok{\textless{}{-}} \FunctionTok{c}\NormalTok{(}\SpecialCharTok{{-}}\DecValTok{2}\NormalTok{, }\SpecialCharTok{{-}}\DecValTok{1}\NormalTok{, }\DecValTok{0}\NormalTok{, }\DecValTok{1}\NormalTok{, }\DecValTok{2}\NormalTok{, }\DecValTok{3}\NormalTok{)}
\end{Highlighting}
\end{Shaded}

Try the following R code and see what it returns.

\begin{Shaded}
\begin{Highlighting}[]
\FunctionTok{sum}\NormalTok{(x }\SpecialCharTok{\textless{}} \DecValTok{0}\NormalTok{)}

\FunctionTok{sum}\NormalTok{(x }\SpecialCharTok{\textless{}=} \DecValTok{0}\NormalTok{)}

\FunctionTok{sum}\NormalTok{(x }\SpecialCharTok{\textgreater{}} \DecValTok{0}\NormalTok{)}

\FunctionTok{sum}\NormalTok{(x }\SpecialCharTok{\textgreater{}=} \DecValTok{0}\NormalTok{)}

\FunctionTok{sum}\NormalTok{(x }\SpecialCharTok{==} \DecValTok{0}\NormalTok{)}

\FunctionTok{sum}\NormalTok{(x }\SpecialCharTok{!=} \DecValTok{0}\NormalTok{)}
\end{Highlighting}
\end{Shaded}

\begin{tcolorbox}[enhanced jigsaw, colframe=quarto-callout-tip-color-frame, coltitle=black, left=2mm, rightrule=.15mm, colback=white, opacityback=0, toprule=.15mm, bottomtitle=1mm, colbacktitle=quarto-callout-tip-color!10!white, breakable, titlerule=0mm, title=\textcolor{quarto-callout-tip-color}{\faLightbulb}\hspace{0.5em}{Answer}, toptitle=1mm, arc=.35mm, bottomrule=.15mm, leftrule=.75mm, opacitybacktitle=0.6]

\begin{Shaded}
\begin{Highlighting}[]
\FunctionTok{sum}\NormalTok{(x }\SpecialCharTok{\textless{}} \DecValTok{0}\NormalTok{)}
\end{Highlighting}
\end{Shaded}

\begin{verbatim}
[1] 2
\end{verbatim}

\begin{Shaded}
\begin{Highlighting}[]
\FunctionTok{sum}\NormalTok{(x }\SpecialCharTok{\textless{}=} \DecValTok{0}\NormalTok{)}
\end{Highlighting}
\end{Shaded}

\begin{verbatim}
[1] 3
\end{verbatim}

\begin{Shaded}
\begin{Highlighting}[]
\FunctionTok{sum}\NormalTok{(x }\SpecialCharTok{\textgreater{}} \DecValTok{0}\NormalTok{)}
\end{Highlighting}
\end{Shaded}

\begin{verbatim}
[1] 3
\end{verbatim}

\begin{Shaded}
\begin{Highlighting}[]
\FunctionTok{sum}\NormalTok{(x }\SpecialCharTok{\textgreater{}=} \DecValTok{0}\NormalTok{)}
\end{Highlighting}
\end{Shaded}

\begin{verbatim}
[1] 4
\end{verbatim}

\begin{Shaded}
\begin{Highlighting}[]
\FunctionTok{sum}\NormalTok{(x }\SpecialCharTok{==} \DecValTok{0}\NormalTok{)}
\end{Highlighting}
\end{Shaded}

\begin{verbatim}
[1] 1
\end{verbatim}

\begin{Shaded}
\begin{Highlighting}[]
\FunctionTok{sum}\NormalTok{(x }\SpecialCharTok{!=} \DecValTok{0}\NormalTok{)}
\end{Highlighting}
\end{Shaded}

\begin{verbatim}
[1] 5
\end{verbatim}

\end{tcolorbox}

💻 \textbf{Hands-On}

Consider daily high temperatures in celsius over two weeks. Write R code
to answer the following questions:

\begin{itemize}
\item
  How many days had temperatures \textbf{below 18°C}?
\item
  How many days had temperatures \textbf{greater than 26°C}?
\item
  How many days had temperatures \textbf{at most 21°C}?
\item
  How many days had temperatures \textbf{at least 28°C}?
\end{itemize}

\begin{Shaded}
\begin{Highlighting}[]
\CommentTok{\# Daily high temperatures over two weeks}
\NormalTok{temps }\OtherTok{\textless{}{-}} \FunctionTok{c}\NormalTok{(}
  \AttributeTok{Mon02 =} \DecValTok{18}\NormalTok{, }\AttributeTok{Tue03 =} \DecValTok{21}\NormalTok{, }\AttributeTok{Wed04 =} \DecValTok{19}\NormalTok{, }\AttributeTok{Thu05 =} \DecValTok{25}\NormalTok{, }\AttributeTok{Fri06 =} \DecValTok{27}\NormalTok{, }\AttributeTok{Sat07 =} \DecValTok{30}\NormalTok{, }\AttributeTok{Sun08 =} \DecValTok{22}\NormalTok{, }
  \AttributeTok{Mon09 =} \DecValTok{20}\NormalTok{, }\AttributeTok{Tue10 =} \DecValTok{16}\NormalTok{, }\AttributeTok{Wed11 =} \DecValTok{24}\NormalTok{, }\AttributeTok{Thu12 =} \DecValTok{28}\NormalTok{, }\AttributeTok{Fri13 =} \DecValTok{31}\NormalTok{, }\AttributeTok{Sat14 =} \DecValTok{29}\NormalTok{, }\AttributeTok{Sun15 =} \DecValTok{17}
\NormalTok{)}
\end{Highlighting}
\end{Shaded}

\begin{tcolorbox}[enhanced jigsaw, colframe=quarto-callout-tip-color-frame, coltitle=black, left=2mm, rightrule=.15mm, colback=white, opacityback=0, toprule=.15mm, bottomtitle=1mm, colbacktitle=quarto-callout-tip-color!10!white, breakable, titlerule=0mm, title=\textcolor{quarto-callout-tip-color}{\faLightbulb}\hspace{0.5em}{Answer}, toptitle=1mm, arc=.35mm, bottomrule=.15mm, leftrule=.75mm, opacitybacktitle=0.6]

\begin{Shaded}
\begin{Highlighting}[]
\FunctionTok{sum}\NormalTok{(temps }\SpecialCharTok{\textless{}} \DecValTok{18}\NormalTok{)}
\end{Highlighting}
\end{Shaded}

\begin{verbatim}
[1] 2
\end{verbatim}

\begin{Shaded}
\begin{Highlighting}[]
\FunctionTok{sum}\NormalTok{(temps }\SpecialCharTok{\textgreater{}} \DecValTok{26}\NormalTok{)}
\end{Highlighting}
\end{Shaded}

\begin{verbatim}
[1] 5
\end{verbatim}

\begin{Shaded}
\begin{Highlighting}[]
\FunctionTok{sum}\NormalTok{(temps }\SpecialCharTok{\textless{}=} \DecValTok{21}\NormalTok{)}
\end{Highlighting}
\end{Shaded}

\begin{verbatim}
[1] 6
\end{verbatim}

\begin{Shaded}
\begin{Highlighting}[]
\FunctionTok{sum}\NormalTok{(temps }\SpecialCharTok{\textgreater{}=} \DecValTok{28}\NormalTok{)}
\end{Highlighting}
\end{Shaded}

\begin{verbatim}
[1] 4
\end{verbatim}

\end{tcolorbox}

\subsection{\texorpdfstring{The \texttt{which()}
function}{The which() function}}\label{the-which-function}

\begin{itemize}
\tightlist
\item
  The \texttt{which()} function returns the indices of \texttt{TRUE}
  values.
\end{itemize}

💻 \textbf{Hands-On}

Consider the following vector

\begin{Shaded}
\begin{Highlighting}[]
\NormalTok{x }\OtherTok{\textless{}{-}} \FunctionTok{c}\NormalTok{(}\SpecialCharTok{{-}}\DecValTok{2}\NormalTok{, }\SpecialCharTok{{-}}\DecValTok{1}\NormalTok{, }\DecValTok{0}\NormalTok{, }\DecValTok{1}\NormalTok{, }\DecValTok{2}\NormalTok{, }\DecValTok{3}\NormalTok{)}
\end{Highlighting}
\end{Shaded}

Try the following R code and see what it returns.

\begin{Shaded}
\begin{Highlighting}[]
\FunctionTok{which}\NormalTok{(x }\SpecialCharTok{\textless{}} \DecValTok{0}\NormalTok{)}

\FunctionTok{which}\NormalTok{(x }\SpecialCharTok{\textgreater{}} \DecValTok{0}\NormalTok{)}
 
\FunctionTok{which}\NormalTok{(x }\SpecialCharTok{\textless{}=} \DecValTok{0}\NormalTok{)}
 
\FunctionTok{which}\NormalTok{(x }\SpecialCharTok{\textgreater{}=} \DecValTok{0}\NormalTok{)}

\FunctionTok{which}\NormalTok{(x }\SpecialCharTok{==} \DecValTok{0}\NormalTok{)}

\FunctionTok{which}\NormalTok{(x }\SpecialCharTok{!=} \DecValTok{0}\NormalTok{)}
\end{Highlighting}
\end{Shaded}

\begin{tcolorbox}[enhanced jigsaw, colframe=quarto-callout-tip-color-frame, coltitle=black, left=2mm, rightrule=.15mm, colback=white, opacityback=0, toprule=.15mm, bottomtitle=1mm, colbacktitle=quarto-callout-tip-color!10!white, breakable, titlerule=0mm, title=\textcolor{quarto-callout-tip-color}{\faLightbulb}\hspace{0.5em}{Answer}, toptitle=1mm, arc=.35mm, bottomrule=.15mm, leftrule=.75mm, opacitybacktitle=0.6]

\begin{Shaded}
\begin{Highlighting}[]
\FunctionTok{which}\NormalTok{(x }\SpecialCharTok{\textless{}} \DecValTok{0}\NormalTok{)}
\end{Highlighting}
\end{Shaded}

\begin{verbatim}
[1] 1 2
\end{verbatim}

\begin{Shaded}
\begin{Highlighting}[]
\FunctionTok{which}\NormalTok{(x }\SpecialCharTok{\textgreater{}} \DecValTok{0}\NormalTok{)}
\end{Highlighting}
\end{Shaded}

\begin{verbatim}
[1] 4 5 6
\end{verbatim}

\begin{Shaded}
\begin{Highlighting}[]
\FunctionTok{which}\NormalTok{(x }\SpecialCharTok{\textless{}=} \DecValTok{0}\NormalTok{)}
\end{Highlighting}
\end{Shaded}

\begin{verbatim}
[1] 1 2 3
\end{verbatim}

\begin{Shaded}
\begin{Highlighting}[]
\FunctionTok{which}\NormalTok{(x }\SpecialCharTok{\textgreater{}=} \DecValTok{0}\NormalTok{)}
\end{Highlighting}
\end{Shaded}

\begin{verbatim}
[1] 3 4 5 6
\end{verbatim}

\begin{Shaded}
\begin{Highlighting}[]
\FunctionTok{which}\NormalTok{(x }\SpecialCharTok{==} \DecValTok{0}\NormalTok{)}
\end{Highlighting}
\end{Shaded}

\begin{verbatim}
[1] 3
\end{verbatim}

\begin{Shaded}
\begin{Highlighting}[]
\FunctionTok{which}\NormalTok{(x }\SpecialCharTok{!=} \DecValTok{0}\NormalTok{)}
\end{Highlighting}
\end{Shaded}

\begin{verbatim}
[1] 1 2 4 5 6
\end{verbatim}

\end{tcolorbox}

\subsection{Relational operators}\label{relational-operators}

\begin{itemize}
\tightlist
\item
  \textbf{Relational operators} are powerful tools to combine
  conditional statements.
\end{itemize}

\begin{longtable}[]{@{}ll@{}}
\toprule\noalign{}
\endhead
\bottomrule\noalign{}
\endlastfoot
\textbf{Relationship Operator} & \textbf{Description} \\
\& & and \\
\textbar{} & or \\
! & not \\
\end{longtable}

💻 \textbf{Hands-On}

Consider the following vector

\begin{Shaded}
\begin{Highlighting}[]
\NormalTok{x }\OtherTok{\textless{}{-}} \FunctionTok{c}\NormalTok{(}\DecValTok{1}\NormalTok{, }\DecValTok{2}\NormalTok{, }\DecValTok{3}\NormalTok{, }\DecValTok{4}\NormalTok{, }\DecValTok{5}\NormalTok{, }\DecValTok{6}\NormalTok{, }\DecValTok{7}\NormalTok{, }\DecValTok{8}\NormalTok{, }\DecValTok{9}\NormalTok{)}
\end{Highlighting}
\end{Shaded}

Try the following R code and see what it returns.

\begin{Shaded}
\begin{Highlighting}[]
\NormalTok{x }\SpecialCharTok{\textgreater{}} \DecValTok{3} \SpecialCharTok{\&}\NormalTok{ x }\SpecialCharTok{\textless{}} \DecValTok{8}

\SpecialCharTok{!}\NormalTok{(x }\SpecialCharTok{\textgreater{}} \DecValTok{3} \SpecialCharTok{\&}\NormalTok{ x }\SpecialCharTok{\textless{}} \DecValTok{8}\NormalTok{)}
\end{Highlighting}
\end{Shaded}

\begin{tcolorbox}[enhanced jigsaw, colframe=quarto-callout-tip-color-frame, coltitle=black, left=2mm, rightrule=.15mm, colback=white, opacityback=0, toprule=.15mm, bottomtitle=1mm, colbacktitle=quarto-callout-tip-color!10!white, breakable, titlerule=0mm, title=\textcolor{quarto-callout-tip-color}{\faLightbulb}\hspace{0.5em}{Answer}, toptitle=1mm, arc=.35mm, bottomrule=.15mm, leftrule=.75mm, opacitybacktitle=0.6]

\begin{Shaded}
\begin{Highlighting}[]
\CommentTok{\# 3 \textless{} x \textless{} 8}
\NormalTok{x }\SpecialCharTok{\textgreater{}} \DecValTok{3} \SpecialCharTok{\&}\NormalTok{ x }\SpecialCharTok{\textless{}} \DecValTok{8}
\end{Highlighting}
\end{Shaded}

\begin{verbatim}
[1] FALSE FALSE FALSE  TRUE  TRUE  TRUE  TRUE FALSE FALSE
\end{verbatim}

\begin{Shaded}
\begin{Highlighting}[]
\CommentTok{\# Opposite of 3 \textless{} x \textless{} 8}
\SpecialCharTok{!}\NormalTok{(x }\SpecialCharTok{\textgreater{}} \DecValTok{3} \SpecialCharTok{\&}\NormalTok{ x }\SpecialCharTok{\textless{}} \DecValTok{8}\NormalTok{)}
\end{Highlighting}
\end{Shaded}

\begin{verbatim}
[1]  TRUE  TRUE  TRUE FALSE FALSE FALSE FALSE  TRUE  TRUE
\end{verbatim}

\end{tcolorbox}

💻 \textbf{Hands-On}

Consider the following vector

\begin{Shaded}
\begin{Highlighting}[]
\NormalTok{x }\OtherTok{\textless{}{-}} \FunctionTok{c}\NormalTok{(}\DecValTok{1}\NormalTok{, }\DecValTok{2}\NormalTok{, }\DecValTok{3}\NormalTok{, }\DecValTok{4}\NormalTok{, }\DecValTok{5}\NormalTok{, }\DecValTok{6}\NormalTok{, }\DecValTok{7}\NormalTok{, }\DecValTok{8}\NormalTok{, }\DecValTok{9}\NormalTok{)}
\end{Highlighting}
\end{Shaded}

Try the following R code and see what it returns.

\begin{Shaded}
\begin{Highlighting}[]
\NormalTok{x }\SpecialCharTok{\textless{}} \DecValTok{4} \SpecialCharTok{|}\NormalTok{ x }\SpecialCharTok{\textgreater{}} \DecValTok{6}

\SpecialCharTok{!}\NormalTok{(x }\SpecialCharTok{\textless{}} \DecValTok{4} \SpecialCharTok{|}\NormalTok{ x }\SpecialCharTok{\textgreater{}} \DecValTok{6}\NormalTok{)}
\end{Highlighting}
\end{Shaded}

\begin{tcolorbox}[enhanced jigsaw, colframe=quarto-callout-tip-color-frame, coltitle=black, left=2mm, rightrule=.15mm, colback=white, opacityback=0, toprule=.15mm, bottomtitle=1mm, colbacktitle=quarto-callout-tip-color!10!white, breakable, titlerule=0mm, title=\textcolor{quarto-callout-tip-color}{\faLightbulb}\hspace{0.5em}{Answer}, toptitle=1mm, arc=.35mm, bottomrule=.15mm, leftrule=.75mm, opacitybacktitle=0.6]

\begin{Shaded}
\begin{Highlighting}[]
\CommentTok{\# x \textless{} 4 or x \textgreater{} 6}
\NormalTok{x }\SpecialCharTok{\textless{}} \DecValTok{4} \SpecialCharTok{|}\NormalTok{ x }\SpecialCharTok{\textgreater{}} \DecValTok{6}
\end{Highlighting}
\end{Shaded}

\begin{verbatim}
[1]  TRUE  TRUE  TRUE FALSE FALSE FALSE  TRUE  TRUE  TRUE
\end{verbatim}

\begin{Shaded}
\begin{Highlighting}[]
\CommentTok{\# Opposite of x \textless{} 4 or x \textgreater{} 6}
\SpecialCharTok{!}\NormalTok{(x }\SpecialCharTok{\textless{}} \DecValTok{4} \SpecialCharTok{|}\NormalTok{ x }\SpecialCharTok{\textgreater{}} \DecValTok{6}\NormalTok{)}
\end{Highlighting}
\end{Shaded}

\begin{verbatim}
[1] FALSE FALSE FALSE  TRUE  TRUE  TRUE FALSE FALSE FALSE
\end{verbatim}

\end{tcolorbox}

💻 \textbf{Hands-On}

Consider daily high temperatures in celsius over two weeks. Write R code
to answer the following questions:

\begin{itemize}
\item
  Which days had temperatures \textbf{at least 20°C and at most 25°C}?
\item
  Which days had temperatures \textbf{below 18°C or above 28°C}?
\end{itemize}

\begin{Shaded}
\begin{Highlighting}[]
\CommentTok{\# Daily high temperatures over two weeks}
\NormalTok{temps }\OtherTok{\textless{}{-}} \FunctionTok{c}\NormalTok{(}
  \AttributeTok{Mon02 =} \DecValTok{18}\NormalTok{, }\AttributeTok{Tue03 =} \DecValTok{21}\NormalTok{, }\AttributeTok{Wed04 =} \DecValTok{19}\NormalTok{, }\AttributeTok{Thu05 =} \DecValTok{25}\NormalTok{, }\AttributeTok{Fri06 =} \DecValTok{27}\NormalTok{, }\AttributeTok{Sat07 =} \DecValTok{30}\NormalTok{, }\AttributeTok{Sun08 =} \DecValTok{22}\NormalTok{, }
  \AttributeTok{Mon09 =} \DecValTok{20}\NormalTok{, }\AttributeTok{Tue10 =} \DecValTok{16}\NormalTok{, }\AttributeTok{Wed11 =} \DecValTok{24}\NormalTok{, }\AttributeTok{Thu12 =} \DecValTok{28}\NormalTok{, }\AttributeTok{Fri13 =} \DecValTok{31}\NormalTok{, }\AttributeTok{Sat14 =} \DecValTok{29}\NormalTok{, }\AttributeTok{Sun15 =} \DecValTok{17}
\NormalTok{)}
\end{Highlighting}
\end{Shaded}

\begin{tcolorbox}[enhanced jigsaw, colframe=quarto-callout-tip-color-frame, coltitle=black, left=2mm, rightrule=.15mm, colback=white, opacityback=0, toprule=.15mm, bottomtitle=1mm, colbacktitle=quarto-callout-tip-color!10!white, breakable, titlerule=0mm, title=\textcolor{quarto-callout-tip-color}{\faLightbulb}\hspace{0.5em}{Answer}, toptitle=1mm, arc=.35mm, bottomrule=.15mm, leftrule=.75mm, opacitybacktitle=0.6]

\begin{Shaded}
\begin{Highlighting}[]
\FunctionTok{which}\NormalTok{(temps }\SpecialCharTok{\textgreater{}=} \DecValTok{20} \SpecialCharTok{\&}\NormalTok{ temps }\SpecialCharTok{\textless{}=} \DecValTok{25}\NormalTok{)}
\end{Highlighting}
\end{Shaded}

\begin{verbatim}
Tue03 Thu05 Sun08 Mon09 Wed11 
    2     4     7     8    10 
\end{verbatim}

\begin{Shaded}
\begin{Highlighting}[]
\FunctionTok{which}\NormalTok{(temps }\SpecialCharTok{\textless{}} \DecValTok{18} \SpecialCharTok{|}\NormalTok{ temps }\SpecialCharTok{\textgreater{}} \DecValTok{28}\NormalTok{)}
\end{Highlighting}
\end{Shaded}

\begin{verbatim}
Sat07 Tue10 Fri13 Sat14 Sun15 
    6     9    12    13    14 
\end{verbatim}

\end{tcolorbox}

\section{Subsetting}\label{subsetting}

\begin{itemize}
\tightlist
\item
  Certain elements of a vector can be subsetted using the brackets
  \texttt{{[}{]}}
\end{itemize}

\includegraphics{index_files/mediabag/2_vector_index.png}

\begin{itemize}
\item
  Subsetting can be done using one of the following indexing methods:

  \begin{itemize}
  \item
    Numeric indexing
  \item
    Logical indexing
  \item
    Character indexing (if the vector has names; not covered here)
  \end{itemize}
\end{itemize}

\section{Numeric indexing}\label{numeric-indexing}

\begin{itemize}
\item
  Numeric indexing means telling R the specific indices to return
  values.
\item
  Consider the following vector
\end{itemize}

\begin{Shaded}
\begin{Highlighting}[]
\CommentTok{\#       1   2   3   4   5   6   7}
\NormalTok{v }\OtherTok{\textless{}{-}} \FunctionTok{c}\NormalTok{(}\DecValTok{70}\NormalTok{, }\DecValTok{66}\NormalTok{, }\DecValTok{82}\NormalTok{, }\DecValTok{85}\NormalTok{, }\DecValTok{78}\NormalTok{, }\DecValTok{90}\NormalTok{, }\DecValTok{73}\NormalTok{)}
\end{Highlighting}
\end{Shaded}

\begin{itemize}
\tightlist
\item
  We have
\end{itemize}

\begin{Shaded}
\begin{Highlighting}[]
\CommentTok{\# 1st value}
\NormalTok{v[}\DecValTok{1}\NormalTok{]            }
\end{Highlighting}
\end{Shaded}

\begin{verbatim}
[1] 70
\end{verbatim}

\begin{Shaded}
\begin{Highlighting}[]
\CommentTok{\# 4th value}
\NormalTok{v[}\DecValTok{4}\NormalTok{]            }
\end{Highlighting}
\end{Shaded}

\begin{verbatim}
[1] 85
\end{verbatim}

\begin{Shaded}
\begin{Highlighting}[]
\CommentTok{\# last value, 7th value}
\NormalTok{v[}\FunctionTok{length}\NormalTok{(v)]    }
\end{Highlighting}
\end{Shaded}

\begin{verbatim}
[1] 73
\end{verbatim}

\begin{Shaded}
\begin{Highlighting}[]
\CommentTok{\# Not available}
\NormalTok{v[}\DecValTok{320}\NormalTok{]          }
\end{Highlighting}
\end{Shaded}

\begin{verbatim}
[1] NA
\end{verbatim}

\begin{itemize}
\tightlist
\item
  Numeric indexing can also use a vector of multiple indices.
\end{itemize}

\begin{Shaded}
\begin{Highlighting}[]
\CommentTok{\# at 2nd, 3rd, 4th, 5th}
\NormalTok{v[}\DecValTok{2}\SpecialCharTok{:}\DecValTok{5}\NormalTok{]                 }
\end{Highlighting}
\end{Shaded}

\begin{verbatim}
[1] 66 82 85 78
\end{verbatim}

\begin{Shaded}
\begin{Highlighting}[]
\CommentTok{\# at 5th, 4th, 3rd, 2nd}
\NormalTok{v[}\DecValTok{5}\SpecialCharTok{:}\DecValTok{2}\NormalTok{]                 }
\end{Highlighting}
\end{Shaded}

\begin{verbatim}
[1] 78 85 82 66
\end{verbatim}

\begin{Shaded}
\begin{Highlighting}[]
\CommentTok{\# at 3rd, 1st, 3rd, 5th, 2nd}
\NormalTok{v[}\FunctionTok{c}\NormalTok{(}\DecValTok{3}\NormalTok{, }\DecValTok{1}\NormalTok{, }\DecValTok{3}\NormalTok{, }\DecValTok{5}\NormalTok{, }\DecValTok{2}\NormalTok{)]    }
\end{Highlighting}
\end{Shaded}

\begin{verbatim}
[1] 82 70 82 78 66
\end{verbatim}

\begin{itemize}
\tightlist
\item
  Note that negative indices mean all but the specified indices.
\end{itemize}

\begin{Shaded}
\begin{Highlighting}[]
\CommentTok{\# all but 1st value}
\NormalTok{v[}\SpecialCharTok{{-}}\DecValTok{1}\NormalTok{]                    }
\end{Highlighting}
\end{Shaded}

\begin{verbatim}
[1] 66 82 85 78 90 73
\end{verbatim}

\begin{Shaded}
\begin{Highlighting}[]
\CommentTok{\# all but 2nd and 5th values}
\NormalTok{v[}\SpecialCharTok{{-}}\FunctionTok{c}\NormalTok{(}\DecValTok{2}\NormalTok{, }\DecValTok{5}\NormalTok{)]              }
\end{Highlighting}
\end{Shaded}

\begin{verbatim}
[1] 70 82 85 90 73
\end{verbatim}

\begin{Shaded}
\begin{Highlighting}[]
\CommentTok{\# all but 1st to 4th values}
\NormalTok{v[}\SpecialCharTok{{-}}\NormalTok{(}\DecValTok{1}\SpecialCharTok{:}\DecValTok{4}\NormalTok{)]                }
\end{Highlighting}
\end{Shaded}

\begin{verbatim}
[1] 78 90 73
\end{verbatim}

\begin{Shaded}
\begin{Highlighting}[]
\CommentTok{\# error!}
\NormalTok{v[}\SpecialCharTok{{-}}\DecValTok{1}\SpecialCharTok{:}\DecValTok{4}\NormalTok{]                  }
\end{Highlighting}
\end{Shaded}

\begin{verbatim}
Error in v[-1:4]: only 0's may be mixed with negative subscripts
\end{verbatim}

💻 \textbf{Hands-On}

Consider the following vector

\begin{Shaded}
\begin{Highlighting}[]
\CommentTok{\#       1   2   3   4   5   6   7}
\NormalTok{v }\OtherTok{\textless{}{-}} \FunctionTok{c}\NormalTok{(}\DecValTok{70}\NormalTok{, }\DecValTok{66}\NormalTok{, }\DecValTok{82}\NormalTok{, }\DecValTok{85}\NormalTok{, }\DecValTok{78}\NormalTok{, }\DecValTok{90}\NormalTok{, }\DecValTok{73}\NormalTok{)}
\end{Highlighting}
\end{Shaded}

Write R code to return the following in \texttt{v}

\begin{itemize}
\item
  The second value
\item
  The sixth value
\item
  The first and last values.
\item
  The values at positions 3 through 6
\item
  The values at positions 6 through 3
\item
  The values at positions 1, 4, and 7
\item
  All values except the first.
\item
  All values except the last two.
\item
  All values except positions 2 and 6.
\end{itemize}

\begin{tcolorbox}[enhanced jigsaw, colframe=quarto-callout-tip-color-frame, coltitle=black, left=2mm, rightrule=.15mm, colback=white, opacityback=0, toprule=.15mm, bottomtitle=1mm, colbacktitle=quarto-callout-tip-color!10!white, breakable, titlerule=0mm, title=\textcolor{quarto-callout-tip-color}{\faLightbulb}\hspace{0.5em}{Answer}, toptitle=1mm, arc=.35mm, bottomrule=.15mm, leftrule=.75mm, opacitybacktitle=0.6]

\begin{Shaded}
\begin{Highlighting}[]
\NormalTok{v[}\DecValTok{2}\NormalTok{]}
\end{Highlighting}
\end{Shaded}

\begin{verbatim}
[1] 66
\end{verbatim}

\begin{Shaded}
\begin{Highlighting}[]
\NormalTok{v[}\DecValTok{6}\NormalTok{]}
\end{Highlighting}
\end{Shaded}

\begin{verbatim}
[1] 90
\end{verbatim}

\begin{Shaded}
\begin{Highlighting}[]
\NormalTok{v[}\FunctionTok{c}\NormalTok{(}\DecValTok{1}\NormalTok{, }\FunctionTok{length}\NormalTok{(v))]}
\end{Highlighting}
\end{Shaded}

\begin{verbatim}
[1] 70 73
\end{verbatim}

\begin{Shaded}
\begin{Highlighting}[]
\NormalTok{v[}\DecValTok{3}\SpecialCharTok{:}\DecValTok{6}\NormalTok{]}
\end{Highlighting}
\end{Shaded}

\begin{verbatim}
[1] 82 85 78 90
\end{verbatim}

\begin{Shaded}
\begin{Highlighting}[]
\NormalTok{v[}\DecValTok{6}\SpecialCharTok{:}\DecValTok{3}\NormalTok{]}
\end{Highlighting}
\end{Shaded}

\begin{verbatim}
[1] 90 78 85 82
\end{verbatim}

\begin{Shaded}
\begin{Highlighting}[]
\NormalTok{v[}\FunctionTok{c}\NormalTok{(}\DecValTok{1}\NormalTok{, }\DecValTok{4}\NormalTok{, }\DecValTok{7}\NormalTok{)]}
\end{Highlighting}
\end{Shaded}

\begin{verbatim}
[1] 70 85 73
\end{verbatim}

\begin{Shaded}
\begin{Highlighting}[]
\NormalTok{v[}\SpecialCharTok{{-}}\DecValTok{1}\NormalTok{]}
\end{Highlighting}
\end{Shaded}

\begin{verbatim}
[1] 66 82 85 78 90 73
\end{verbatim}

\begin{Shaded}
\begin{Highlighting}[]
\NormalTok{v[}\SpecialCharTok{{-}}\NormalTok{(}\DecValTok{6}\SpecialCharTok{:}\DecValTok{7}\NormalTok{)]    }\CommentTok{\# or v[{-}c(6, 7)]}
\end{Highlighting}
\end{Shaded}

\begin{verbatim}
[1] 70 66 82 85 78
\end{verbatim}

\begin{Shaded}
\begin{Highlighting}[]
\NormalTok{v[}\SpecialCharTok{{-}}\FunctionTok{c}\NormalTok{(}\DecValTok{2}\NormalTok{, }\DecValTok{6}\NormalTok{)]}
\end{Highlighting}
\end{Shaded}

\begin{verbatim}
[1] 70 82 85 78 73
\end{verbatim}

\end{tcolorbox}

\section{Logical Indexing}\label{logical-indexing}

\begin{itemize}
\item
  \textbf{Logical indexing} means telling R which values to return and
  which values not to return.
\item
  With logical indexing, \texttt{TRUE} will retain the elements, while
  \texttt{FALSE} will discard the elements.
\item
  Logical indexing is powerful because it allows for subsetting with
  conditions.
\item
  Consider the following vector
\end{itemize}

\begin{Shaded}
\begin{Highlighting}[]
\CommentTok{\#       1   2   3   4   5   6   7}
\NormalTok{v }\OtherTok{\textless{}{-}} \FunctionTok{c}\NormalTok{(}\DecValTok{70}\NormalTok{, }\DecValTok{66}\NormalTok{, }\DecValTok{82}\NormalTok{, }\DecValTok{85}\NormalTok{, }\DecValTok{78}\NormalTok{, }\DecValTok{90}\NormalTok{, }\DecValTok{73}\NormalTok{)}
\end{Highlighting}
\end{Shaded}

\begin{Shaded}
\begin{Highlighting}[]
\CommentTok{\# 1st, 6th}
\NormalTok{v[}\FunctionTok{c}\NormalTok{(}\ConstantTok{TRUE}\NormalTok{, }\ConstantTok{FALSE}\NormalTok{, }\ConstantTok{FALSE}\NormalTok{, }\ConstantTok{FALSE}\NormalTok{, }\ConstantTok{FALSE}\NormalTok{, }\ConstantTok{TRUE}\NormalTok{, }\ConstantTok{FALSE}\NormalTok{)]    }
\end{Highlighting}
\end{Shaded}

\begin{verbatim}
[1] 70 90
\end{verbatim}

\begin{Shaded}
\begin{Highlighting}[]
\CommentTok{\# 2nd, 3rd, 4th}
\NormalTok{v[}\FunctionTok{c}\NormalTok{(}\ConstantTok{FALSE}\NormalTok{, }\ConstantTok{TRUE}\NormalTok{, }\ConstantTok{TRUE}\NormalTok{, }\ConstantTok{TRUE}\NormalTok{, }\ConstantTok{FALSE}\NormalTok{, }\ConstantTok{FALSE}\NormalTok{, }\ConstantTok{FALSE}\NormalTok{)]     }
\end{Highlighting}
\end{Shaded}

\begin{verbatim}
[1] 66 82 85
\end{verbatim}

\begin{Shaded}
\begin{Highlighting}[]
\CommentTok{\# those values greater than 80}
\NormalTok{v[v }\SpecialCharTok{\textgreater{}} \DecValTok{80}\NormalTok{]    }
\end{Highlighting}
\end{Shaded}

\begin{verbatim}
[1] 82 85 90
\end{verbatim}

💻 \textbf{Hands-On}

Consider the following vector

\begin{Shaded}
\begin{Highlighting}[]
\CommentTok{\#       1   2   3   4   5   6   7}
\NormalTok{v }\OtherTok{\textless{}{-}} \FunctionTok{c}\NormalTok{(}\DecValTok{70}\NormalTok{, }\DecValTok{66}\NormalTok{, }\DecValTok{82}\NormalTok{, }\DecValTok{85}\NormalTok{, }\DecValTok{78}\NormalTok{, }\DecValTok{90}\NormalTok{, }\DecValTok{73}\NormalTok{)}
\end{Highlighting}
\end{Shaded}

Use numeric and logical indexing to subset

\begin{itemize}
\item
  \texttt{66\ 82\ 90\ 73}
\item
  \texttt{70\ 66\ 78\ 73}
\end{itemize}

\begin{tcolorbox}[enhanced jigsaw, colframe=quarto-callout-tip-color-frame, coltitle=black, left=2mm, rightrule=.15mm, colback=white, opacityback=0, toprule=.15mm, bottomtitle=1mm, colbacktitle=quarto-callout-tip-color!10!white, breakable, titlerule=0mm, title=\textcolor{quarto-callout-tip-color}{\faLightbulb}\hspace{0.5em}{Answer}, toptitle=1mm, arc=.35mm, bottomrule=.15mm, leftrule=.75mm, opacitybacktitle=0.6]

\begin{Shaded}
\begin{Highlighting}[]
\CommentTok{\# 66 82 90 73}

\NormalTok{v[}\FunctionTok{c}\NormalTok{(}\DecValTok{2}\NormalTok{, }\DecValTok{3}\NormalTok{, }\DecValTok{6}\NormalTok{, }\DecValTok{7}\NormalTok{)]}
\NormalTok{v[}\FunctionTok{c}\NormalTok{(}\ConstantTok{FALSE}\NormalTok{, }\ConstantTok{TRUE}\NormalTok{, }\ConstantTok{TRUE}\NormalTok{, }\ConstantTok{FALSE}\NormalTok{, }\ConstantTok{FALSE}\NormalTok{, }\ConstantTok{TRUE}\NormalTok{, }\ConstantTok{TRUE}\NormalTok{)]}

\CommentTok{\# 70 66 78 73}

\NormalTok{v[}\FunctionTok{c}\NormalTok{(}\DecValTok{1}\NormalTok{, }\DecValTok{2}\NormalTok{, }\DecValTok{5}\NormalTok{, }\DecValTok{7}\NormalTok{)]}
\NormalTok{v[}\FunctionTok{c}\NormalTok{(}\ConstantTok{TRUE}\NormalTok{, }\ConstantTok{TRUE}\NormalTok{, }\ConstantTok{FALSE}\NormalTok{, }\ConstantTok{FALSE}\NormalTok{, }\ConstantTok{TRUE}\NormalTok{, }\ConstantTok{FALSE}\NormalTok{, }\ConstantTok{TRUE}\NormalTok{)]}
\NormalTok{v[v }\SpecialCharTok{\textless{}} \DecValTok{80}\NormalTok{]}
\end{Highlighting}
\end{Shaded}

\end{tcolorbox}

💻 \textbf{Hands-On}

The monthly cost for a stand-alone drug plan varies from plan to plan
and from state to state. The vector \texttt{cost} gives the premium for
the plan with the lowest cost for each state in 2005 (excludes District
of Columbia). Write R code using \texttt{cost} and \texttt{state.name}
to find state(s) with a monthly premium

\begin{itemize}
\item
  Less than or equal to \(\$5\)
\item
  Greater than \(\$10\)
\item
  Between \(\$5\) and \(\$10\) (exclusive)
\item
  Exactly equal to \(\$1.87\)
\item
  At most \(\$7\)
\item
  At least \(\$15\)
\item
  To be the smallest
\item
  To be the largest
\end{itemize}

\begin{Shaded}
\begin{Highlighting}[]
\CommentTok{\# Recall the vector \textasciigrave{}state.name\textasciigrave{}}

\NormalTok{cost }\OtherTok{\textless{}{-}} \FunctionTok{c}\NormalTok{(}
  \FloatTok{14.08}\NormalTok{, }\FloatTok{20.05}\NormalTok{, }\FloatTok{6.14}\NormalTok{, }\FloatTok{10.31}\NormalTok{, }\FloatTok{5.41}\NormalTok{, }\FloatTok{8.62}\NormalTok{, }\FloatTok{7.32}\NormalTok{,}
  \FloatTok{6.44}\NormalTok{, }\FloatTok{10.35}\NormalTok{, }\FloatTok{17.91}\NormalTok{, }\FloatTok{17.18}\NormalTok{, }\FloatTok{6.33}\NormalTok{, }\FloatTok{13.32}\NormalTok{, }\FloatTok{12.30}\NormalTok{, }\FloatTok{1.87}\NormalTok{,}
  \FloatTok{9.48}\NormalTok{, }\FloatTok{12.30}\NormalTok{, }\FloatTok{17.06}\NormalTok{, }\FloatTok{19.60}\NormalTok{, }\FloatTok{6.44}\NormalTok{, }\FloatTok{7.32}\NormalTok{, }\FloatTok{13.75}\NormalTok{, }\FloatTok{1.87}\NormalTok{,}
  \FloatTok{11.60}\NormalTok{, }\FloatTok{10.29}\NormalTok{, }\FloatTok{1.87}\NormalTok{, }\FloatTok{1.87}\NormalTok{, }\FloatTok{6.42}\NormalTok{, }\FloatTok{19.60}\NormalTok{, }\FloatTok{4.43}\NormalTok{, }\FloatTok{10.65}\NormalTok{,}
  \FloatTok{4.10}\NormalTok{, }\FloatTok{13.27}\NormalTok{, }\FloatTok{1.87}\NormalTok{, }\FloatTok{14.43}\NormalTok{, }\FloatTok{10.07}\NormalTok{, }\FloatTok{6.93}\NormalTok{, }\FloatTok{10.14}\NormalTok{, }\FloatTok{7.32}\NormalTok{,}
  \FloatTok{16.57}\NormalTok{, }\FloatTok{1.87}\NormalTok{, }\FloatTok{14.08}\NormalTok{, }\FloatTok{10.31}\NormalTok{, }\FloatTok{6.33}\NormalTok{, }\FloatTok{7.32}\NormalTok{, }\FloatTok{8.81}\NormalTok{, }\FloatTok{6.93}\NormalTok{,}
  \FloatTok{10.14}\NormalTok{, }\FloatTok{11.42}\NormalTok{, }\FloatTok{1.87}
\NormalTok{)}
\end{Highlighting}
\end{Shaded}

\begin{tcolorbox}[enhanced jigsaw, colframe=quarto-callout-tip-color-frame, coltitle=black, left=2mm, rightrule=.15mm, colback=white, opacityback=0, toprule=.15mm, bottomtitle=1mm, colbacktitle=quarto-callout-tip-color!10!white, breakable, titlerule=0mm, title=\textcolor{quarto-callout-tip-color}{\faLightbulb}\hspace{0.5em}{Answer}, toptitle=1mm, arc=.35mm, bottomrule=.15mm, leftrule=.75mm, opacitybacktitle=0.6]

\begin{Shaded}
\begin{Highlighting}[]
\NormalTok{state.name[cost }\SpecialCharTok{\textless{}=} \DecValTok{5}\NormalTok{]}
\end{Highlighting}
\end{Shaded}

\begin{verbatim}
[1] "Iowa"         "Minnesota"    "Montana"      "Nebraska"     "New Jersey"  
[6] "New York"     "North Dakota" "South Dakota" "Wyoming"     
\end{verbatim}

\begin{Shaded}
\begin{Highlighting}[]
\NormalTok{state.name[cost }\SpecialCharTok{\textgreater{}} \DecValTok{10}\NormalTok{]}
\end{Highlighting}
\end{Shaded}

\begin{verbatim}
 [1] "Alabama"        "Alaska"         "Arkansas"       "Florida"       
 [5] "Georgia"        "Hawaii"         "Illinois"       "Indiana"       
 [9] "Kentucky"       "Louisiana"      "Maine"          "Michigan"      
[13] "Mississippi"    "Missouri"       "New Hampshire"  "New Mexico"    
[17] "North Carolina" "Ohio"           "Oklahoma"       "Pennsylvania"  
[21] "South Carolina" "Tennessee"      "Texas"          "West Virginia" 
[25] "Wisconsin"     
\end{verbatim}

\begin{Shaded}
\begin{Highlighting}[]
\NormalTok{state.name[cost }\SpecialCharTok{\textgreater{}} \DecValTok{5} \SpecialCharTok{\&}\NormalTok{ cost }\SpecialCharTok{\textless{}} \DecValTok{10}\NormalTok{]}
\end{Highlighting}
\end{Shaded}

\begin{verbatim}
 [1] "Arizona"       "California"    "Colorado"      "Connecticut"  
 [5] "Delaware"      "Idaho"         "Kansas"        "Maryland"     
 [9] "Massachusetts" "Nevada"        "Oregon"        "Rhode Island" 
[13] "Utah"          "Vermont"       "Virginia"      "Washington"   
\end{verbatim}

\begin{Shaded}
\begin{Highlighting}[]
\NormalTok{state.name[cost }\SpecialCharTok{==} \FloatTok{1.87}\NormalTok{]}
\end{Highlighting}
\end{Shaded}

\begin{verbatim}
[1] "Iowa"         "Minnesota"    "Montana"      "Nebraska"     "North Dakota"
[6] "South Dakota" "Wyoming"     
\end{verbatim}

\begin{Shaded}
\begin{Highlighting}[]
\NormalTok{state.name[cost }\SpecialCharTok{\textless{}=} \DecValTok{7}\NormalTok{]}
\end{Highlighting}
\end{Shaded}

\begin{verbatim}
 [1] "Arizona"      "California"   "Delaware"     "Idaho"        "Iowa"        
 [6] "Maryland"     "Minnesota"    "Montana"      "Nebraska"     "Nevada"      
[11] "New Jersey"   "New York"     "North Dakota" "Oregon"       "South Dakota"
[16] "Utah"         "Washington"   "Wyoming"     
\end{verbatim}

\begin{Shaded}
\begin{Highlighting}[]
\NormalTok{state.name[cost }\SpecialCharTok{\textgreater{}=} \DecValTok{15}\NormalTok{]}
\end{Highlighting}
\end{Shaded}

\begin{verbatim}
[1] "Alaska"         "Georgia"        "Hawaii"         "Louisiana"     
[5] "Maine"          "New Hampshire"  "South Carolina"
\end{verbatim}

\begin{Shaded}
\begin{Highlighting}[]
\NormalTok{state.name[cost }\SpecialCharTok{==} \FunctionTok{min}\NormalTok{(cost)]}
\end{Highlighting}
\end{Shaded}

\begin{verbatim}
[1] "Iowa"         "Minnesota"    "Montana"      "Nebraska"     "North Dakota"
[6] "South Dakota" "Wyoming"     
\end{verbatim}

\begin{Shaded}
\begin{Highlighting}[]
\NormalTok{state.name[cost }\SpecialCharTok{==} \FunctionTok{max}\NormalTok{(cost)]}
\end{Highlighting}
\end{Shaded}

\begin{verbatim}
[1] "Alaska"
\end{verbatim}

\end{tcolorbox}

\section{Useful functions for
vectors}\label{useful-functions-for-vectors}

Given a vector,

\begin{itemize}
\item
  \texttt{head()} returns its first few elements
\item
  \texttt{tail()} returns its last few elements
\item
  \texttt{sort()} returns its sorted version in increasing or decreasing
  order
\item
  \texttt{order()} returns the indices of its sorted version
\item
  \texttt{rev()} returns the reversed vector
\item
  \texttt{unique()} returns a vector of its unique elements
\item
  \texttt{table()} shows unique elements and their frequencies
\end{itemize}

💻 \textbf{Hands-On}

The monthly cost for a stand-alone drug plan varies from plan to plan
and from state to state. The vector \texttt{cost} gives the premium for
the plan with the lowest cost for each state in 2005 (excludes District
of Columbia).

\begin{Shaded}
\begin{Highlighting}[]
\NormalTok{cost }\OtherTok{\textless{}{-}} \FunctionTok{c}\NormalTok{(}
  \FloatTok{14.08}\NormalTok{, }\FloatTok{20.05}\NormalTok{, }\FloatTok{6.14}\NormalTok{, }\FloatTok{10.31}\NormalTok{, }\FloatTok{5.41}\NormalTok{, }\FloatTok{8.62}\NormalTok{, }\FloatTok{7.32}\NormalTok{,}
  \FloatTok{6.44}\NormalTok{, }\FloatTok{10.35}\NormalTok{, }\FloatTok{17.91}\NormalTok{, }\FloatTok{17.18}\NormalTok{, }\FloatTok{6.33}\NormalTok{, }\FloatTok{13.32}\NormalTok{, }\FloatTok{12.30}\NormalTok{, }\FloatTok{1.87}\NormalTok{,}
  \FloatTok{9.48}\NormalTok{, }\FloatTok{12.30}\NormalTok{, }\FloatTok{17.06}\NormalTok{, }\FloatTok{19.60}\NormalTok{, }\FloatTok{6.44}\NormalTok{, }\FloatTok{7.32}\NormalTok{, }\FloatTok{13.75}\NormalTok{, }\FloatTok{1.87}\NormalTok{,}
  \FloatTok{11.60}\NormalTok{, }\FloatTok{10.29}\NormalTok{, }\FloatTok{1.87}\NormalTok{, }\FloatTok{1.87}\NormalTok{, }\FloatTok{6.42}\NormalTok{, }\FloatTok{19.60}\NormalTok{, }\FloatTok{4.43}\NormalTok{, }\FloatTok{10.65}\NormalTok{,}
  \FloatTok{4.10}\NormalTok{, }\FloatTok{13.27}\NormalTok{, }\FloatTok{1.87}\NormalTok{, }\FloatTok{14.43}\NormalTok{, }\FloatTok{10.07}\NormalTok{, }\FloatTok{6.93}\NormalTok{, }\FloatTok{10.14}\NormalTok{, }\FloatTok{7.32}\NormalTok{,}
  \FloatTok{16.57}\NormalTok{, }\FloatTok{1.87}\NormalTok{, }\FloatTok{14.08}\NormalTok{, }\FloatTok{10.31}\NormalTok{, }\FloatTok{6.33}\NormalTok{, }\FloatTok{7.32}\NormalTok{, }\FloatTok{8.81}\NormalTok{, }\FloatTok{6.93}\NormalTok{,}
  \FloatTok{10.14}\NormalTok{, }\FloatTok{11.42}\NormalTok{, }\FloatTok{1.87}
\NormalTok{)}
\end{Highlighting}
\end{Shaded}

Try the following R code and see what it returns.

\begin{Shaded}
\begin{Highlighting}[]
\FunctionTok{head}\NormalTok{(cost)    }\CommentTok{\# head(cost, n = 6)}

\FunctionTok{tail}\NormalTok{(cost)    }\CommentTok{\# tail(cost, n = 6)}

\FunctionTok{sort}\NormalTok{(cost)}

\FunctionTok{sort}\NormalTok{(cost, }\AttributeTok{decreasing =} \ConstantTok{TRUE}\NormalTok{)}

\FunctionTok{order}\NormalTok{(cost)}

\FunctionTok{order}\NormalTok{(cost, }\AttributeTok{decreasing =} \ConstantTok{TRUE}\NormalTok{)}

\FunctionTok{rev}\NormalTok{(cost)}
\end{Highlighting}
\end{Shaded}

\begin{tcolorbox}[enhanced jigsaw, colframe=quarto-callout-tip-color-frame, coltitle=black, left=2mm, rightrule=.15mm, colback=white, opacityback=0, toprule=.15mm, bottomtitle=1mm, colbacktitle=quarto-callout-tip-color!10!white, breakable, titlerule=0mm, title=\textcolor{quarto-callout-tip-color}{\faLightbulb}\hspace{0.5em}{Answer}, toptitle=1mm, arc=.35mm, bottomrule=.15mm, leftrule=.75mm, opacitybacktitle=0.6]

\begin{Shaded}
\begin{Highlighting}[]
\FunctionTok{head}\NormalTok{(cost)    }\CommentTok{\# head(cost, n = 6)}
\end{Highlighting}
\end{Shaded}

\begin{verbatim}
[1] 14.08 20.05  6.14 10.31  5.41  8.62
\end{verbatim}

\begin{Shaded}
\begin{Highlighting}[]
\FunctionTok{tail}\NormalTok{(cost)    }\CommentTok{\# tail(cost, n = 6)}
\end{Highlighting}
\end{Shaded}

\begin{verbatim}
[1]  7.32  8.81  6.93 10.14 11.42  1.87
\end{verbatim}

\begin{Shaded}
\begin{Highlighting}[]
\FunctionTok{sort}\NormalTok{(cost)}
\end{Highlighting}
\end{Shaded}

\begin{verbatim}
 [1]  1.87  1.87  1.87  1.87  1.87  1.87  1.87  4.10  4.43  5.41  6.14  6.33
[13]  6.33  6.42  6.44  6.44  6.93  6.93  7.32  7.32  7.32  7.32  8.62  8.81
[25]  9.48 10.07 10.14 10.14 10.29 10.31 10.31 10.35 10.65 11.42 11.60 12.30
[37] 12.30 13.27 13.32 13.75 14.08 14.08 14.43 16.57 17.06 17.18 17.91 19.60
[49] 19.60 20.05
\end{verbatim}

\begin{Shaded}
\begin{Highlighting}[]
\FunctionTok{sort}\NormalTok{(cost, }\AttributeTok{decreasing =} \ConstantTok{TRUE}\NormalTok{)}
\end{Highlighting}
\end{Shaded}

\begin{verbatim}
 [1] 20.05 19.60 19.60 17.91 17.18 17.06 16.57 14.43 14.08 14.08 13.75 13.32
[13] 13.27 12.30 12.30 11.60 11.42 10.65 10.35 10.31 10.31 10.29 10.14 10.14
[25] 10.07  9.48  8.81  8.62  7.32  7.32  7.32  7.32  6.93  6.93  6.44  6.44
[37]  6.42  6.33  6.33  6.14  5.41  4.43  4.10  1.87  1.87  1.87  1.87  1.87
[49]  1.87  1.87
\end{verbatim}

\begin{Shaded}
\begin{Highlighting}[]
\FunctionTok{order}\NormalTok{(cost)}
\end{Highlighting}
\end{Shaded}

\begin{verbatim}
 [1] 15 23 26 27 34 41 50 32 30  5  3 12 44 28  8 20 37 47  7 21 39 45  6 46 16
[26] 36 38 48 25  4 43  9 31 49 24 14 17 33 13 22  1 42 35 40 18 11 10 19 29  2
\end{verbatim}

\begin{Shaded}
\begin{Highlighting}[]
\FunctionTok{order}\NormalTok{(cost, }\AttributeTok{decreasing =} \ConstantTok{TRUE}\NormalTok{)}
\end{Highlighting}
\end{Shaded}

\begin{verbatim}
 [1]  2 19 29 10 11 18 40 35  1 42 22 13 33 14 17 24 49 31  9  4 43 25 38 48 36
[26] 16 46  6  7 21 39 45 37 47  8 20 28 12 44  3  5 30 32 15 23 26 27 34 41 50
\end{verbatim}

\begin{Shaded}
\begin{Highlighting}[]
\FunctionTok{rev}\NormalTok{(cost)}
\end{Highlighting}
\end{Shaded}

\begin{verbatim}
 [1]  1.87 11.42 10.14  6.93  8.81  7.32  6.33 10.31 14.08  1.87 16.57  7.32
[13] 10.14  6.93 10.07 14.43  1.87 13.27  4.10 10.65  4.43 19.60  6.42  1.87
[25]  1.87 10.29 11.60  1.87 13.75  7.32  6.44 19.60 17.06 12.30  9.48  1.87
[37] 12.30 13.32  6.33 17.18 17.91 10.35  6.44  7.32  8.62  5.41 10.31  6.14
[49] 20.05 14.08
\end{verbatim}

\end{tcolorbox}

💻 \textbf{Hands-On}

The vectors \texttt{scores} and \texttt{grades} contain exam scores and
the corresponding letter grades for students in a statistics class.

\begin{Shaded}
\begin{Highlighting}[]
\NormalTok{scores }\OtherTok{\textless{}{-}} \FunctionTok{c}\NormalTok{(}
  \DecValTok{88}\NormalTok{, }\DecValTok{92}\NormalTok{, }\DecValTok{75}\NormalTok{, }\DecValTok{84}\NormalTok{, }\DecValTok{91}\NormalTok{, }\DecValTok{67}\NormalTok{, }\DecValTok{73}\NormalTok{, }\DecValTok{88}\NormalTok{, }\DecValTok{95}\NormalTok{, }\DecValTok{78}\NormalTok{,}
  \DecValTok{82}\NormalTok{, }\DecValTok{90}\NormalTok{, }\DecValTok{69}\NormalTok{, }\DecValTok{74}\NormalTok{, }\DecValTok{85}\NormalTok{, }\DecValTok{88}\NormalTok{, }\DecValTok{93}\NormalTok{, }\DecValTok{77}\NormalTok{, }\DecValTok{81}\NormalTok{, }\DecValTok{89}\NormalTok{,}
  \DecValTok{94}\NormalTok{, }\DecValTok{72}\NormalTok{, }\DecValTok{68}\NormalTok{, }\DecValTok{86}\NormalTok{, }\DecValTok{80}\NormalTok{, }\DecValTok{76}\NormalTok{, }\DecValTok{83}\NormalTok{, }\DecValTok{88}\NormalTok{, }\DecValTok{91}\NormalTok{, }\DecValTok{79}
\NormalTok{)}

\NormalTok{grades }\OtherTok{\textless{}{-}} \FunctionTok{c}\NormalTok{(}
  \StringTok{\textquotesingle{}B\textquotesingle{}}\NormalTok{, }\StringTok{\textquotesingle{}A\textquotesingle{}}\NormalTok{, }\StringTok{\textquotesingle{}C\textquotesingle{}}\NormalTok{, }\StringTok{\textquotesingle{}B\textquotesingle{}}\NormalTok{, }\StringTok{\textquotesingle{}A\textquotesingle{}}\NormalTok{, }\StringTok{\textquotesingle{}D\textquotesingle{}}\NormalTok{, }\StringTok{\textquotesingle{}C\textquotesingle{}}\NormalTok{, }\StringTok{\textquotesingle{}B\textquotesingle{}}\NormalTok{, }\StringTok{\textquotesingle{}A\textquotesingle{}}\NormalTok{, }\StringTok{\textquotesingle{}C\textquotesingle{}}\NormalTok{,}
  \StringTok{\textquotesingle{}B\textquotesingle{}}\NormalTok{, }\StringTok{\textquotesingle{}A\textquotesingle{}}\NormalTok{, }\StringTok{\textquotesingle{}D\textquotesingle{}}\NormalTok{, }\StringTok{\textquotesingle{}C\textquotesingle{}}\NormalTok{, }\StringTok{\textquotesingle{}B\textquotesingle{}}\NormalTok{, }\StringTok{\textquotesingle{}B\textquotesingle{}}\NormalTok{, }\StringTok{\textquotesingle{}A\textquotesingle{}}\NormalTok{, }\StringTok{\textquotesingle{}C\textquotesingle{}}\NormalTok{, }\StringTok{\textquotesingle{}B\textquotesingle{}}\NormalTok{, }\StringTok{\textquotesingle{}B\textquotesingle{}}\NormalTok{,}
  \StringTok{\textquotesingle{}A\textquotesingle{}}\NormalTok{, }\StringTok{\textquotesingle{}C\textquotesingle{}}\NormalTok{, }\StringTok{\textquotesingle{}D\textquotesingle{}}\NormalTok{, }\StringTok{\textquotesingle{}B\textquotesingle{}}\NormalTok{, }\StringTok{\textquotesingle{}B\textquotesingle{}}\NormalTok{, }\StringTok{\textquotesingle{}C\textquotesingle{}}\NormalTok{, }\StringTok{\textquotesingle{}B\textquotesingle{}}\NormalTok{, }\StringTok{\textquotesingle{}B\textquotesingle{}}\NormalTok{, }\StringTok{\textquotesingle{}A\textquotesingle{}}\NormalTok{, }\StringTok{\textquotesingle{}C\textquotesingle{}}
\NormalTok{)}
\end{Highlighting}
\end{Shaded}

Write R code to:

\begin{itemize}
\item
  Show the first and last few exam scores
\item
  Sort the exam scores in decreasing order
\item
  Show unique letter grades of students in this class and their
  frequencies
\end{itemize}

\begin{tcolorbox}[enhanced jigsaw, colframe=quarto-callout-tip-color-frame, coltitle=black, left=2mm, rightrule=.15mm, colback=white, opacityback=0, toprule=.15mm, bottomtitle=1mm, colbacktitle=quarto-callout-tip-color!10!white, breakable, titlerule=0mm, title=\textcolor{quarto-callout-tip-color}{\faLightbulb}\hspace{0.5em}{Answer}, toptitle=1mm, arc=.35mm, bottomrule=.15mm, leftrule=.75mm, opacitybacktitle=0.6]

\begin{Shaded}
\begin{Highlighting}[]
\FunctionTok{head}\NormalTok{(scores)}
\end{Highlighting}
\end{Shaded}

\begin{verbatim}
[1] 88 92 75 84 91 67
\end{verbatim}

\begin{Shaded}
\begin{Highlighting}[]
\FunctionTok{tail}\NormalTok{(scores)}
\end{Highlighting}
\end{Shaded}

\begin{verbatim}
[1] 80 76 83 88 91 79
\end{verbatim}

\begin{Shaded}
\begin{Highlighting}[]
\FunctionTok{sort}\NormalTok{(scores, }\AttributeTok{decreasing =} \ConstantTok{TRUE}\NormalTok{)}
\end{Highlighting}
\end{Shaded}

\begin{verbatim}
 [1] 95 94 93 92 91 91 90 89 88 88 88 88 86 85 84 83 82 81 80 79 78 77 76 75 74
[26] 73 72 69 68 67
\end{verbatim}

\begin{Shaded}
\begin{Highlighting}[]
\FunctionTok{table}\NormalTok{(grades)}
\end{Highlighting}
\end{Shaded}

\begin{verbatim}
grades
 A  B  C  D 
 7 12  8  3 
\end{verbatim}

\end{tcolorbox}

\chapter{2C: Character Vector and
Factor}\label{c-character-vector-and-factor}

\section{Readings}\label{readings-2}

From \textbf{R Coding Basics: An Introduction to the Basics of Coding in
R} by Dr.~Gaston Sanchez:

\begin{itemize}
\tightlist
\item
  \href{https://www.gastonsanchez.com/R-coding-basics/2-01-factors.html}{Factors}
\end{itemize}

\section{Topics}\label{topics-3}

\begin{itemize}
\item
  Character vector
\item
  Useful functions for character vector
\item
  Factor
\end{itemize}

\section{Character vector}\label{character-vector}

\begin{itemize}
\item
  Data can be in the form of specific categories that describe a certain
  characteristic.
\item
  Categories such as gender, color, nationality, and college major are
  said to be \textbf{nominal} because they have \ul{no} natural
  ordering.
\item
  In contrast, categories such as education level, customer
  satisfaction, and Likert scale responses, are \textbf{ordinal} because
  they do have a natural ordering.
\item
  In R, a \textbf{character vector} is often used to represent nominal
  categories.
\end{itemize}

\begin{Shaded}
\begin{Highlighting}[]
\CommentTok{\# character vector}
\FunctionTok{c}\NormalTok{(}\StringTok{\textquotesingle{}male\textquotesingle{}}\NormalTok{, }\StringTok{\textquotesingle{}female\textquotesingle{}}\NormalTok{, }\StringTok{\textquotesingle{}female\textquotesingle{}}\NormalTok{, }\StringTok{\textquotesingle{}male\textquotesingle{}}\NormalTok{, }\StringTok{\textquotesingle{}female\textquotesingle{}}\NormalTok{)  }
\FunctionTok{c}\NormalTok{(}\StringTok{\textquotesingle{}Math\textquotesingle{}}\NormalTok{, }\StringTok{\textquotesingle{}Data Science\textquotesingle{}}\NormalTok{, }\StringTok{\textquotesingle{}Math\textquotesingle{}}\NormalTok{, }\StringTok{\textquotesingle{}Computer Science\textquotesingle{}}\NormalTok{, }\StringTok{\textquotesingle{}Data Science\textquotesingle{}}\NormalTok{)  }
\end{Highlighting}
\end{Shaded}

\section{Useful functions for character
vector}\label{useful-functions-for-character-vector}

\subsection{\texorpdfstring{The \texttt{nchar()}
function}{The nchar() function}}\label{the-nchar-function}

\begin{itemize}
\tightlist
\item
  \texttt{nchar()} counts the number of characters of each element.
\end{itemize}

\begin{Shaded}
\begin{Highlighting}[]
\NormalTok{v }\OtherTok{\textless{}{-}} \FunctionTok{c}\NormalTok{(}\StringTok{\textquotesingle{}Data\textquotesingle{}}\NormalTok{, }\StringTok{\textquotesingle{}science\textquotesingle{}}\NormalTok{, }\StringTok{\textquotesingle{}is\textquotesingle{}}\NormalTok{, }\StringTok{\textquotesingle{}fun\textquotesingle{}}\NormalTok{, }\StringTok{\textquotesingle{}and\textquotesingle{}}\NormalTok{, }\StringTok{\textquotesingle{}challenging\textquotesingle{}}\NormalTok{)}
\FunctionTok{nchar}\NormalTok{(v)}
\end{Highlighting}
\end{Shaded}

\begin{verbatim}
[1]  4  7  2  3  3 11
\end{verbatim}

💻 \textbf{Hands-On}

Write R code to count the number of characters of the following vector:

\begin{Shaded}
\begin{Highlighting}[]
\NormalTok{towns }\OtherTok{\textless{}{-}} \FunctionTok{c}\NormalTok{(}\StringTok{\textquotesingle{}Glassboro\textquotesingle{}}\NormalTok{, }\StringTok{\textquotesingle{}Clayton\textquotesingle{}}\NormalTok{, }\StringTok{\textquotesingle{}Pitman\textquotesingle{}}\NormalTok{, }\StringTok{\textquotesingle{}Deptford\textquotesingle{}}\NormalTok{)}
\end{Highlighting}
\end{Shaded}

\begin{tcolorbox}[enhanced jigsaw, colframe=quarto-callout-tip-color-frame, coltitle=black, left=2mm, rightrule=.15mm, colback=white, opacityback=0, toprule=.15mm, bottomtitle=1mm, colbacktitle=quarto-callout-tip-color!10!white, breakable, titlerule=0mm, title=\textcolor{quarto-callout-tip-color}{\faLightbulb}\hspace{0.5em}{Answer}, toptitle=1mm, arc=.35mm, bottomrule=.15mm, leftrule=.75mm, opacitybacktitle=0.6]

\begin{Shaded}
\begin{Highlighting}[]
\FunctionTok{nchar}\NormalTok{(towns)}
\end{Highlighting}
\end{Shaded}

\begin{verbatim}
[1] 9 7 6 8
\end{verbatim}

\end{tcolorbox}

\subsection{\texorpdfstring{The \texttt{strsplit()}
function}{The strsplit() function}}\label{the-strsplit-function}

\begin{itemize}
\tightlist
\item
  \texttt{strsplit()} splits a string into words.
\end{itemize}

\begin{Shaded}
\begin{Highlighting}[]
\FunctionTok{strsplit}\NormalTok{(}\StringTok{\textquotesingle{}Data science is fun and challenging\textquotesingle{}}\NormalTok{, }\AttributeTok{split =} \StringTok{\textquotesingle{} \textquotesingle{}}\NormalTok{)}
\end{Highlighting}
\end{Shaded}

\begin{verbatim}
[[1]]
[1] "Data"        "science"     "is"          "fun"         "and"        
[6] "challenging"
\end{verbatim}

💻 \textbf{Hands-On}

Write R code extract words in the following string:

\begin{Shaded}
\begin{Highlighting}[]
\CommentTok{\#\textgreater{} \textquotesingle{}Name,Age,Major,GPA,Hobbies\textquotesingle{}}
\end{Highlighting}
\end{Shaded}

\begin{tcolorbox}[enhanced jigsaw, colframe=quarto-callout-tip-color-frame, coltitle=black, left=2mm, rightrule=.15mm, colback=white, opacityback=0, toprule=.15mm, bottomtitle=1mm, colbacktitle=quarto-callout-tip-color!10!white, breakable, titlerule=0mm, title=\textcolor{quarto-callout-tip-color}{\faLightbulb}\hspace{0.5em}{Answer}, toptitle=1mm, arc=.35mm, bottomrule=.15mm, leftrule=.75mm, opacitybacktitle=0.6]

\begin{Shaded}
\begin{Highlighting}[]
\FunctionTok{strsplit}\NormalTok{(}\StringTok{\textquotesingle{}Name,Age,Major,GPA,Hobbies\textquotesingle{}}\NormalTok{, }\AttributeTok{split =} \StringTok{\textquotesingle{},\textquotesingle{}}\NormalTok{)}
\end{Highlighting}
\end{Shaded}

\begin{verbatim}
[[1]]
[1] "Name"    "Age"     "Major"   "GPA"     "Hobbies"
\end{verbatim}

\end{tcolorbox}

\subsection{\texorpdfstring{The \texttt{paste()}
function}{The paste() function}}\label{the-paste-function}

\begin{itemize}
\tightlist
\item
  \texttt{paste()} concatenates elements of character vectors
\end{itemize}

\begin{Shaded}
\begin{Highlighting}[]
\FunctionTok{paste}\NormalTok{(towns, }\StringTok{\textquotesingle{}NJ\textquotesingle{}}\NormalTok{)}
\end{Highlighting}
\end{Shaded}

\begin{verbatim}
[1] "Glassboro NJ" "Clayton NJ"   "Pitman NJ"    "Deptford NJ" 
\end{verbatim}

\begin{Shaded}
\begin{Highlighting}[]
\FunctionTok{paste}\NormalTok{(towns, }\StringTok{\textquotesingle{}NJ\textquotesingle{}}\NormalTok{, }\AttributeTok{sep =} \StringTok{\textquotesingle{} \textquotesingle{}}\NormalTok{)}
\end{Highlighting}
\end{Shaded}

\begin{verbatim}
[1] "Glassboro NJ" "Clayton NJ"   "Pitman NJ"    "Deptford NJ" 
\end{verbatim}

\begin{Shaded}
\begin{Highlighting}[]
\FunctionTok{paste}\NormalTok{(towns, }\StringTok{\textquotesingle{}NJ\textquotesingle{}}\NormalTok{, }\AttributeTok{sep =} \StringTok{\textquotesingle{}, \textquotesingle{}}\NormalTok{)}
\end{Highlighting}
\end{Shaded}

\begin{verbatim}
[1] "Glassboro, NJ" "Clayton, NJ"   "Pitman, NJ"    "Deptford, NJ" 
\end{verbatim}

\begin{Shaded}
\begin{Highlighting}[]
\FunctionTok{paste}\NormalTok{(towns, }\StringTok{\textquotesingle{}NJ\textquotesingle{}}\NormalTok{, }\AttributeTok{sep =} \StringTok{\textquotesingle{} in \textquotesingle{}}\NormalTok{)}
\end{Highlighting}
\end{Shaded}

\begin{verbatim}
[1] "Glassboro in NJ" "Clayton in NJ"   "Pitman in NJ"    "Deptford in NJ" 
\end{verbatim}

💻 \textbf{Hands-On}

Write R code to generate the following output:

\begin{Shaded}
\begin{Highlighting}[]
\CommentTok{\#\textgreater{} [1] "Glassboro Township" "Clayton Township"   "Pitman Township"   }
\CommentTok{\#\textgreater{} [4] "Deptford Township"}
\end{Highlighting}
\end{Shaded}

\begin{tcolorbox}[enhanced jigsaw, colframe=quarto-callout-tip-color-frame, coltitle=black, left=2mm, rightrule=.15mm, colback=white, opacityback=0, toprule=.15mm, bottomtitle=1mm, colbacktitle=quarto-callout-tip-color!10!white, breakable, titlerule=0mm, title=\textcolor{quarto-callout-tip-color}{\faLightbulb}\hspace{0.5em}{Answer}, toptitle=1mm, arc=.35mm, bottomrule=.15mm, leftrule=.75mm, opacitybacktitle=0.6]

\begin{Shaded}
\begin{Highlighting}[]
\FunctionTok{paste}\NormalTok{(towns, }\StringTok{\textquotesingle{}Township\textquotesingle{}}\NormalTok{, }\AttributeTok{sep =} \StringTok{\textquotesingle{} \textquotesingle{}}\NormalTok{)}
\end{Highlighting}
\end{Shaded}

\begin{verbatim}
[1] "Glassboro Township" "Clayton Township"   "Pitman Township"   
[4] "Deptford Township" 
\end{verbatim}

\end{tcolorbox}

💻 \textbf{Hands-On}

What does the \texttt{paste0()} function does?

\begin{Shaded}
\begin{Highlighting}[]
\NormalTok{towns }\OtherTok{\textless{}{-}} \FunctionTok{c}\NormalTok{(}\StringTok{\textquotesingle{}Glassboro\textquotesingle{}}\NormalTok{, }\StringTok{\textquotesingle{}Clayton\textquotesingle{}}\NormalTok{, }\StringTok{\textquotesingle{}Pitman\textquotesingle{}}\NormalTok{, }\StringTok{\textquotesingle{}Deptford\textquotesingle{}}\NormalTok{)}

\FunctionTok{paste0}\NormalTok{(towns, }\StringTok{\textquotesingle{}NJ\textquotesingle{}}\NormalTok{)}
\end{Highlighting}
\end{Shaded}

\begin{verbatim}
[1] "GlassboroNJ" "ClaytonNJ"   "PitmanNJ"    "DeptfordNJ" 
\end{verbatim}

\begin{tcolorbox}[enhanced jigsaw, colframe=quarto-callout-tip-color-frame, coltitle=black, left=2mm, rightrule=.15mm, colback=white, opacityback=0, toprule=.15mm, bottomtitle=1mm, colbacktitle=quarto-callout-tip-color!10!white, breakable, titlerule=0mm, title=\textcolor{quarto-callout-tip-color}{\faLightbulb}\hspace{0.5em}{Answer}, toptitle=1mm, arc=.35mm, bottomrule=.15mm, leftrule=.75mm, opacitybacktitle=0.6]

\texttt{paste0} is a shortcut for
\texttt{paste(...,\ sep\ =\ \textquotesingle{}\textquotesingle{})} (no
separator)

\begin{Shaded}
\begin{Highlighting}[]
\NormalTok{towns }\OtherTok{\textless{}{-}} \FunctionTok{c}\NormalTok{(}\StringTok{\textquotesingle{}Glassboro\textquotesingle{}}\NormalTok{, }\StringTok{\textquotesingle{}Clayton\textquotesingle{}}\NormalTok{, }\StringTok{\textquotesingle{}Pitman\textquotesingle{}}\NormalTok{, }\StringTok{\textquotesingle{}Deptford\textquotesingle{}}\NormalTok{)}

\FunctionTok{paste0}\NormalTok{(towns, }\StringTok{\textquotesingle{}NJ\textquotesingle{}}\NormalTok{)}
\end{Highlighting}
\end{Shaded}

\begin{verbatim}
[1] "GlassboroNJ" "ClaytonNJ"   "PitmanNJ"    "DeptfordNJ" 
\end{verbatim}

\end{tcolorbox}

\begin{itemize}
\tightlist
\item
  Thanks to vectorization, \texttt{paste()} allows corresponding
  elements to be concatenated.
\end{itemize}

\begin{Shaded}
\begin{Highlighting}[]
\NormalTok{first\_names }\OtherTok{\textless{}{-}} \FunctionTok{c}\NormalTok{(}\StringTok{\textquotesingle{}James\textquotesingle{}}\NormalTok{, }\StringTok{\textquotesingle{}Robert\textquotesingle{}}\NormalTok{, }\StringTok{\textquotesingle{}Mary\textquotesingle{}}\NormalTok{)}
\NormalTok{last\_names  }\OtherTok{\textless{}{-}} \FunctionTok{c}\NormalTok{(}\StringTok{\textquotesingle{}Smith\textquotesingle{}}\NormalTok{, }\StringTok{\textquotesingle{}Williams\textquotesingle{}}\NormalTok{, }\StringTok{\textquotesingle{}Brown\textquotesingle{}}\NormalTok{)}

\FunctionTok{paste}\NormalTok{(first\_names, last\_names)}
\end{Highlighting}
\end{Shaded}

\begin{verbatim}
[1] "James Smith"     "Robert Williams" "Mary Brown"     
\end{verbatim}

💻 \textbf{Hands-On}

Write R code to generate the following output:

\begin{Shaded}
\begin{Highlighting}[]
\CommentTok{\#\textgreater{} [1] "Smith, James"     "Williams, Robert" "Brown, Mary"}
\end{Highlighting}
\end{Shaded}

\begin{tcolorbox}[enhanced jigsaw, colframe=quarto-callout-tip-color-frame, coltitle=black, left=2mm, rightrule=.15mm, colback=white, opacityback=0, toprule=.15mm, bottomtitle=1mm, colbacktitle=quarto-callout-tip-color!10!white, breakable, titlerule=0mm, title=\textcolor{quarto-callout-tip-color}{\faLightbulb}\hspace{0.5em}{Answer}, toptitle=1mm, arc=.35mm, bottomrule=.15mm, leftrule=.75mm, opacitybacktitle=0.6]

\begin{Shaded}
\begin{Highlighting}[]
\FunctionTok{paste}\NormalTok{(last\_names, first\_names, }\AttributeTok{collapse =} \StringTok{\textquotesingle{}, \textquotesingle{}}\NormalTok{)}
\end{Highlighting}
\end{Shaded}

\begin{verbatim}
[1] "Smith James, Williams Robert, Brown Mary"
\end{verbatim}

\end{tcolorbox}

💻 \textbf{Hands-On}

Write R code to generate the following output:

\begin{Shaded}
\begin{Highlighting}[]
\CommentTok{\#\textgreater{} [1] "Glassboro, NJ 08028" "Clayton, NJ 08312"   "Pitman, NJ 08071"   }
\CommentTok{\#\textgreater{} [4] "Deptford, NJ 08096" }
\end{Highlighting}
\end{Shaded}

\begin{tcolorbox}[enhanced jigsaw, colframe=quarto-callout-tip-color-frame, coltitle=black, left=2mm, rightrule=.15mm, colback=white, opacityback=0, toprule=.15mm, bottomtitle=1mm, colbacktitle=quarto-callout-tip-color!10!white, breakable, titlerule=0mm, title=\textcolor{quarto-callout-tip-color}{\faLightbulb}\hspace{0.5em}{Answer}, toptitle=1mm, arc=.35mm, bottomrule=.15mm, leftrule=.75mm, opacitybacktitle=0.6]

\begin{Shaded}
\begin{Highlighting}[]
\NormalTok{zip\_code }\OtherTok{\textless{}{-}} \FunctionTok{c}\NormalTok{(}\StringTok{\textquotesingle{}08028\textquotesingle{}}\NormalTok{, }\StringTok{\textquotesingle{}08312\textquotesingle{}}\NormalTok{, }\StringTok{\textquotesingle{}08071\textquotesingle{}}\NormalTok{, }\StringTok{\textquotesingle{}08096\textquotesingle{}}\NormalTok{)}

\FunctionTok{paste0}\NormalTok{(towns, }\StringTok{\textquotesingle{}, NJ \textquotesingle{}}\NormalTok{, zip\_code)}
\end{Highlighting}
\end{Shaded}

\begin{verbatim}
[1] "Glassboro, NJ 08028" "Clayton, NJ 08312"   "Pitman, NJ 08071"   
[4] "Deptford, NJ 08096" 
\end{verbatim}

\end{tcolorbox}

\section{Factor}\label{factor}

\begin{itemize}
\item
  A \textbf{factor} is a specialized data structure for categories,
  especially ordinal categories.
\item
  It behaves like a vector but has additional attributes.
\item
  A factor displays categories (levels) but stores them internally as
  integers.
\item
  A factor is created from a character vector using the
  \texttt{factor()} function.
\end{itemize}

\begin{Shaded}
\begin{Highlighting}[]
\DocumentationTok{\#\#\# Nominal factor}

\CommentTok{\#\textgreater{} [1] O  A  AB O  O  B  A  B  AB}
\CommentTok{\#\textgreater{} Levels: A B AB O}

\DocumentationTok{\#\#\# Ordinal factor}

\CommentTok{\#\textgreater{}  [1] extra hot mild      hot       mild      medium    extra hot mild     }
\CommentTok{\#\textgreater{}  [8] medium    extra hot medium    extra hot mild     }
\CommentTok{\#\textgreater{} Levels: mild \textless{} medium \textless{} hot \textless{} extra hot}
\end{Highlighting}
\end{Shaded}

💻 \textbf{Hands-On}

Create a nominal factor from the following character vector:

\begin{Shaded}
\begin{Highlighting}[]
\NormalTok{blood }\OtherTok{\textless{}{-}} \FunctionTok{c}\NormalTok{(}\StringTok{\textquotesingle{}O\textquotesingle{}}\NormalTok{, }\StringTok{\textquotesingle{}A\textquotesingle{}}\NormalTok{, }\StringTok{\textquotesingle{}AB\textquotesingle{}}\NormalTok{, }\StringTok{\textquotesingle{}O\textquotesingle{}}\NormalTok{, }\StringTok{\textquotesingle{}O\textquotesingle{}}\NormalTok{, }\StringTok{\textquotesingle{}B\textquotesingle{}}\NormalTok{, }\StringTok{\textquotesingle{}A\textquotesingle{}}\NormalTok{, }\StringTok{\textquotesingle{}B\textquotesingle{}}\NormalTok{, }\StringTok{\textquotesingle{}AB\textquotesingle{}}\NormalTok{)}
\end{Highlighting}
\end{Shaded}

\begin{tcolorbox}[enhanced jigsaw, colframe=quarto-callout-tip-color-frame, coltitle=black, left=2mm, rightrule=.15mm, colback=white, opacityback=0, toprule=.15mm, bottomtitle=1mm, colbacktitle=quarto-callout-tip-color!10!white, breakable, titlerule=0mm, title=\textcolor{quarto-callout-tip-color}{\faLightbulb}\hspace{0.5em}{Answer}, toptitle=1mm, arc=.35mm, bottomrule=.15mm, leftrule=.75mm, opacitybacktitle=0.6]

\begin{Shaded}
\begin{Highlighting}[]
\FunctionTok{factor}\NormalTok{(blood, }\AttributeTok{levels =} \FunctionTok{c}\NormalTok{(}\StringTok{\textquotesingle{}O\textquotesingle{}}\NormalTok{, }\StringTok{\textquotesingle{}A\textquotesingle{}}\NormalTok{, }\StringTok{\textquotesingle{}B\textquotesingle{}}\NormalTok{, }\StringTok{\textquotesingle{}AB\textquotesingle{}}\NormalTok{))}
\end{Highlighting}
\end{Shaded}

\begin{verbatim}
[1] O  A  AB O  O  B  A  B  AB
Levels: O A B AB
\end{verbatim}

\end{tcolorbox}

💻 \textbf{Hands-On}

Create an ordinal factor from the following character vector:

\begin{Shaded}
\begin{Highlighting}[]
\NormalTok{pepper }\OtherTok{\textless{}{-}} \FunctionTok{c}\NormalTok{(}\StringTok{\textquotesingle{}extra hot\textquotesingle{}}\NormalTok{, }\StringTok{\textquotesingle{}mild\textquotesingle{}}\NormalTok{, }\StringTok{\textquotesingle{}hot\textquotesingle{}}\NormalTok{, }\StringTok{\textquotesingle{}mild\textquotesingle{}}\NormalTok{, }\StringTok{\textquotesingle{}medium\textquotesingle{}}\NormalTok{, }\StringTok{\textquotesingle{}extra hot\textquotesingle{}}\NormalTok{, }
            \StringTok{\textquotesingle{}mild\textquotesingle{}}\NormalTok{, }\StringTok{\textquotesingle{}medium\textquotesingle{}}\NormalTok{, }\StringTok{\textquotesingle{}extra hot\textquotesingle{}}\NormalTok{, }\StringTok{\textquotesingle{}medium\textquotesingle{}}\NormalTok{, }\StringTok{\textquotesingle{}extra hot\textquotesingle{}}\NormalTok{, }\StringTok{\textquotesingle{}mild\textquotesingle{}}\NormalTok{)}
\end{Highlighting}
\end{Shaded}

\begin{tcolorbox}[enhanced jigsaw, colframe=quarto-callout-tip-color-frame, coltitle=black, left=2mm, rightrule=.15mm, colback=white, opacityback=0, toprule=.15mm, bottomtitle=1mm, colbacktitle=quarto-callout-tip-color!10!white, breakable, titlerule=0mm, title=\textcolor{quarto-callout-tip-color}{\faLightbulb}\hspace{0.5em}{Answer}, toptitle=1mm, arc=.35mm, bottomrule=.15mm, leftrule=.75mm, opacitybacktitle=0.6]

\begin{Shaded}
\begin{Highlighting}[]
\FunctionTok{factor}\NormalTok{(pepper, }\AttributeTok{levels =} \FunctionTok{c}\NormalTok{(}\StringTok{\textquotesingle{}mild\textquotesingle{}}\NormalTok{, }\StringTok{\textquotesingle{}medium\textquotesingle{}}\NormalTok{, }\StringTok{\textquotesingle{}hot\textquotesingle{}}\NormalTok{, }\StringTok{\textquotesingle{}extra hot\textquotesingle{}}\NormalTok{), }\AttributeTok{ordered =} \ConstantTok{TRUE}\NormalTok{)}
\end{Highlighting}
\end{Shaded}

\begin{verbatim}
 [1] extra hot mild      hot       mild      medium    extra hot mild     
 [8] medium    extra hot medium    extra hot mild     
Levels: mild < medium < hot < extra hot
\end{verbatim}

\end{tcolorbox}

💻 \textbf{Hands-On}

Create an ordinal factor from the following character vector:

\begin{Shaded}
\begin{Highlighting}[]
\NormalTok{shirts }\OtherTok{\textless{}{-}} \FunctionTok{c}\NormalTok{(}\StringTok{\textquotesingle{}L\textquotesingle{}}\NormalTok{, }\StringTok{\textquotesingle{}XL\textquotesingle{}}\NormalTok{, }\StringTok{\textquotesingle{}XL\textquotesingle{}}\NormalTok{, }\StringTok{\textquotesingle{}L\textquotesingle{}}\NormalTok{, }\StringTok{\textquotesingle{}M\textquotesingle{}}\NormalTok{, }\StringTok{\textquotesingle{}S\textquotesingle{}}\NormalTok{, }\StringTok{\textquotesingle{}S\textquotesingle{}}\NormalTok{, }\StringTok{\textquotesingle{}XL\textquotesingle{}}\NormalTok{, }\StringTok{\textquotesingle{}M\textquotesingle{}}\NormalTok{, }\StringTok{\textquotesingle{}S\textquotesingle{}}\NormalTok{, }\StringTok{\textquotesingle{}L\textquotesingle{}}\NormalTok{, }\StringTok{\textquotesingle{}S\textquotesingle{}}\NormalTok{)}
\end{Highlighting}
\end{Shaded}

\begin{tcolorbox}[enhanced jigsaw, colframe=quarto-callout-tip-color-frame, coltitle=black, left=2mm, rightrule=.15mm, colback=white, opacityback=0, toprule=.15mm, bottomtitle=1mm, colbacktitle=quarto-callout-tip-color!10!white, breakable, titlerule=0mm, title=\textcolor{quarto-callout-tip-color}{\faLightbulb}\hspace{0.5em}{Answer}, toptitle=1mm, arc=.35mm, bottomrule=.15mm, leftrule=.75mm, opacitybacktitle=0.6]

\begin{Shaded}
\begin{Highlighting}[]
\FunctionTok{factor}\NormalTok{(shirts, }\AttributeTok{levels =} \FunctionTok{c}\NormalTok{(}\StringTok{\textquotesingle{}S\textquotesingle{}}\NormalTok{, }\StringTok{\textquotesingle{}M\textquotesingle{}}\NormalTok{, }\StringTok{\textquotesingle{}L\textquotesingle{}}\NormalTok{, }\StringTok{\textquotesingle{}XL\textquotesingle{}}\NormalTok{),}
       \AttributeTok{labels =} \FunctionTok{c}\NormalTok{(}\StringTok{\textquotesingle{}Small\textquotesingle{}}\NormalTok{, }\StringTok{\textquotesingle{}Medium\textquotesingle{}}\NormalTok{, }\StringTok{\textquotesingle{}Large\textquotesingle{}}\NormalTok{, }\StringTok{\textquotesingle{}Extra Large\textquotesingle{}}\NormalTok{))}
\end{Highlighting}
\end{Shaded}

\begin{verbatim}
 [1] Large       Extra Large Extra Large Large       Medium      Small      
 [7] Small       Extra Large Medium      Small       Large       Small      
Levels: Small Medium Large Extra Large
\end{verbatim}

\end{tcolorbox}




\end{document}
